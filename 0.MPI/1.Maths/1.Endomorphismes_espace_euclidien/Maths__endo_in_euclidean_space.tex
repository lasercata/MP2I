\documentclass[a4paper, 12pt, twoside]{article}


%------------------------------------------------------------------------
%
% Author                :   Lasercata
% Last modification     :   2023.04.14
%
%------------------------------------------------------------------------


%------ini
\usepackage[utf8]{inputenc}
\usepackage[T1]{fontenc}
%\usepackage[french]{babel}
%\usepackage[english]{babel}


%------geometry
\usepackage[textheight=700pt, textwidth=500pt]{geometry}


%------color
\usepackage{xcolor}
\definecolor{ff4500}{HTML}{ff4500}
\definecolor{00f}{HTML}{0000ff}
\definecolor{0ff}{HTML}{00ffff}
\definecolor{656565}{HTML}{656565}

%\renewcommand{\emph}{\textcolor{ff4500}}
%\renewcommand{\em}{\color{ff4500}}

\newcommand{\Emph}{\textcolor{ff4500}}

\newcommand{\strong}[1]{\textcolor{ff4500}{\bf #1}}
\newcommand{\st}{\color{ff4500}\bf}


%------Code highlighting
%---listings
\usepackage{listings}

\definecolor{cbg}{HTML}{272822}
\definecolor{cfg}{HTML}{ececec}
\definecolor{ccomment}{HTML}{686c58}
\definecolor{ckw}{HTML}{f92672}
\definecolor{cstring}{HTML}{e6db72}
\definecolor{cstringlight}{HTML}{98980f}
\definecolor{lightwhite}{HTML}{fafafa}

\lstdefinestyle{DarkCodeStyle}{
    backgroundcolor=\color{cbg},
    commentstyle=\itshape\color{ccomment},
    keywordstyle=\color{ckw},
    numberstyle=\tiny\color{cbg},
    stringstyle=\color{cstring},
    basicstyle=\ttfamily\footnotesize\color{cfg},
    breakatwhitespace=false,
    breaklines=true,
    captionpos=b,
    keepspaces=true,
    numbers=left,
    numbersep=5pt,
    showspaces=false,
    showstringspaces=false,
    showtabs=false,
    tabsize=4,
    xleftmargin=\leftskip
}

\lstdefinestyle{LightCodeStyle}{
    backgroundcolor=\color{lightwhite},
    commentstyle=\itshape\color{ccomment},
    keywordstyle=\color{ckw},
    numberstyle=\tiny\color{cbg},
    stringstyle=\color{cstringlight},
    basicstyle=\ttfamily\footnotesize\color{cbg},
    breakatwhitespace=false,
    breaklines=true,
    captionpos=b,
    keepspaces=true,
    numbers=left,
    numbersep=10pt,
    showspaces=false,
    showstringspaces=false,
    showtabs=false,
    tabsize=4,
    frame=L,
    xleftmargin=\leftskip
}

%\lstset{style=DarkCodeStyle}
\lstset{style=LightCodeStyle}
%Usage : \begin{lstlisting}[language=Caml, xleftmargin=xpt] ... \end{lstlisting}


%---Algorithm
\usepackage[linesnumbered,ruled,vlined]{algorithm2e}
\SetKwInput{KwInput}{Input}
\SetKwInput{KwOutput}{Output}

\SetKwProg{Fn}{Function}{:}{}
\SetKw{KwPrint}{Print}

\newcommand\commfont[1]{\textit{\texttt{\textcolor{656565}{#1}}}}
\SetCommentSty{commfont}
\SetProgSty{texttt}
\SetArgSty{textnormal}
\SetFuncArgSty{textnormal}
%\SetProgArgSty{texttt}

\newenvironment{indalgo}[2][H]{
    \begin{algoBox}
        \begin{algorithm}[#1]
            \caption{#2}
}
{
        \end{algorithm}
    \end{algoBox}
}


%---tcolorbox
\usepackage[many]{tcolorbox}
\DeclareTColorBox{emphBox}{O{black}O{lightwhite}}{
    breakable,
    outer arc=0pt,
    arc=0pt,
    top=0pt,
    toprule=-.5pt,
    right=0pt,
    rightrule=-.5pt,
    bottom=0pt,
    bottomrule=-.5pt,
    colframe=#1,
    colback=#2,
    enlarge left by=10pt,
    width=\linewidth-\leftskip-10pt,
}

\DeclareTColorBox{algoBox}{O{black}O{lightwhite}}{
    breakable,
    arc=0pt,
    top=0pt,
    toprule=-.5pt,
    right=0pt,
    rightrule=-.5pt,
    bottom=0pt,
    bottomrule=-.5pt,
    left=0pt,
    leftrule=-.5pt,
    colframe=#1,
    colback=#2,
    width=\linewidth-\leftskip-10pt,
}


%-------make the table of content clickable
\usepackage{hyperref}
\hypersetup{
    colorlinks,
    citecolor=black,
    filecolor=black,
    linkcolor=black,
    urlcolor=black
}


%------pictures
\usepackage{graphicx}
%\usepackage{wrapfig}

\usepackage{tikz}
%\usetikzlibrary{babel}             %Uncomment this to use circuitikz
%\usetikzlibrary{shapes.geometric}  % To draw triangles in trees
%\usepackage{circuitikz}            %Electrical circuits drawing


%------tabular
%\usepackage{color}
%\usepackage{colortbl}
%\usepackage{multirow}


%------Physics
%---Packages
%\usepackage[version=4]{mhchem} %$\ce{NO4^2-}$

%---Commands
\newcommand{\link}[2]{\mathrm{#1} \! - \! \mathrm{#2}}
\newcommand{\pt}[1]{\cdot 10^{#1}} % Power of ten
\newcommand{\dt}[2][t]{\dfrac{\mathrm d #2}{\mathrm d #1}} % Derivative


%------math
%---Packages
%\usepackage{textcomp}
%\usepackage{amsmath}
\usepackage{amssymb}
\usepackage{mathtools} % For abs
\usepackage{stmaryrd} %for \llbracket and \rrbracket
\usepackage{mathrsfs} %for \mathscr{x} (different from \mathcal{x})

%---Commands
%-Sets
\newcommand{\N}{\mathbb{N}} %set N
\newcommand{\Z}{\mathbb{Z}} %set Z
\newcommand{\Q}{\mathbb{Q}} %set Q
\newcommand{\R}{\mathbb{R}} %set R
\newcommand{\C}{\mathbb{C}} %set C
\newcommand{\U}{\mathbb{U}} %set U
\newcommand{\seg}[2]{\left[ #1\ ;\ #2 \right]}
\newcommand{\nset}[2]{\left\llbracket #1\ ;\ #2 \right\rrbracket}

%-Exponantial / complexs
\newcommand{\e}{\mathrm{e}}
\newcommand{\cj}[1]{\overline{#1}} %overline for the conjugate.

%-Vectors
\newcommand{\vect}{\overrightarrow}
\newcommand{\veco}[3]{\displaystyle \vect{#1}\binom{#2}{#3}} %vector + coord

%-Limits
\newcommand{\lm}[2][{}]{\lim\limits_{\substack{#2 \\ #1}}} %$\lm{x \to a} f$ or $\lm[x < a]{x \to a} f$
\newcommand{\Lm}[3][{}]{\lm[#1]{#2} \left( #3 \right)} %$\Lm{x \to a}{f}$ or $\Lm[x < a]{x \to a}{f}$
\newcommand{\tendsto}[1]{\xrightarrow[#1]{}}

%-Integral
\newcommand{\dint}[4][x]{\displaystyle \int_{#2}^{#3} #4 \mathrm{d} #1} %$\dint{a}{b}{f(x)}$ or $\dint[t]{a}{b}{f(t)}$

%-left right
\newcommand{\lr}[1]{\left( #1 \right)}
\newcommand{\lrb}[1]{\left[ #1 \right]}
\newcommand{\lrbb}[1]{\left\llbracket #1 \right\rrbracket}
\newcommand{\set}[1]{\left\{ #1 \right\}}
\newcommand{\abs}[1]{\left\lvert #1 \right\rvert}
\newcommand{\ceil}[1]{\left\lceil #1 \right\rceil}
\newcommand{\floor}[1]{\left\lfloor #1 \right\rfloor}
\newcommand{\lrangle}[1]{\left\langle #1 \right\rangle}

\newcommand{\norm}[1]{\left\lVert #1 \right\rVert}
\newcommand{\nnorm}[2][-2pt]{{\left\lvert \kern#1 \left\lvert \kern#1 \left\lvert #2 \right\rvert \kern#1 \right\rvert \kern#1 \right\rvert}}

%-Others
\newcommand{\para}{\ /\!/\ } %//
\newcommand{\ssi}{\ \Leftrightarrow \ }
\newcommand{\eqsys}[2]{\begin{cases} #1 \\ #2 \end{cases}}

\newcommand{\med}[2]{\mathrm{med} \left[ #1\ ;\ #2 \right]}  %$\med{A}{B} -> med[A ; B]$
\newcommand{\Circ}[2]{\mathscr{C}_{#1, #2}}

\renewcommand{\le}{\leqslant}
\renewcommand{\ge}{\geqslant}

\newcommand{\oboxed}[1]{\textcolor{ff4500}{\boxed{\textcolor{black}{#1}}}} %orange boxed

\newcommand{\rboxed}[1]{\begin{array}{|c} \hline #1 \\ \hline \end{array}} %boxed with right opened
\newcommand{\lboxed}[1]{\begin{array}{c|} \hline #1 \\ \hline \end{array}} %boxed with left opened

\newcommand{\orboxed}[1]{\textcolor{ff4500}{\rboxed{\textcolor{black}{#1}}}} %orange right boxed
\newcommand{\olboxed}[1]{\textcolor{ff4500}{\lboxed{\textcolor{black}{#1}}}} %orange left boxed

\newcommand{\ps}[2]{\lrangle{#1\ |\ #2}}

%------commands
%---to quote
\newcommand{\simplecit}[1]{\guillemotleft$\;$#1$\;$\guillemotright}
\newcommand{\cit}[1]{\simplecit{\textcolor{656565}{#1}}}
\newcommand{\quo}[1]{\cit{\it #1}}

%---to indent
\newcommand{\ind}[1][20pt]{\advance\leftskip + #1}
\newcommand{\deind}[1][20pt]{\advance\leftskip - #1}

%---to indent a text
\newcommand{\indented}[2][20pt]{\par \ind[#1] #2 \par \deind[#1]}
\newenvironment{indt}[2][20pt]{#2 \par \ind[#1]}{\par \deind} %Titled indented env

%---title
\newcommand{\thetitle}[2]{\begin{center}\textbf{{\LARGE \underline{\Emph{#1} :}} {\Large #2}}\end{center}}

%---Maths environments
%-Proofs
\newenvironment{proof}[1][{}]{\begin{indt}{$\square$ #1}}{$\blacksquare$ \end{indt}}

%-Maths parts (proposition, definition, ...)
\newenvironment{mathpart}[1]{\begin{indt}{\boxed{\text{\textbf{#1}}}}}{\end{indt}}
\newenvironment{mathbox}[1]{\boxed{\text{\textbf{#1}}}\begin{emphBox}}{\end{emphBox}}
\newenvironment{mathul}[1]{\begin{indt}{\underline{\textbf{#1}}}}{\end{indt}}

\newenvironment{theo}{\begin{mathpart}{Théorème}}{\end{mathpart}}
\newenvironment{Theo}{\begin{mathbox}{Théorème}}{\end{mathbox}}

\newenvironment{prop}{\begin{mathpart}{Proposition}}{\end{mathpart}}
\newenvironment{Prop}{\begin{mathbox}{Proposition}}{\end{mathbox}}
\newenvironment{props}{\begin{mathpart}{Propriétés}}{\end{mathpart}}

\newenvironment{defi}{\begin{mathpart}{Définition}}{\end{mathpart}}
\newenvironment{meth}{\begin{mathpart}{Méthode}}{\end{mathpart}}

\newenvironment{Rq}{\begin{mathul}{Remarque :}}{\end{mathul}}
\newenvironment{Rqs}{\begin{mathul}{Remarques :}}{\end{mathul}}

\newenvironment{Ex}{\begin{mathul}{Exemple :}}{\end{mathul}}
\newenvironment{Exs}{\begin{mathul}{Exemples :}}{\end{mathul}}


%------Sections
% To change section numbering :
% \renewcommand\thesection{\Roman{section}}
% \renewcommand\thesubsection{\arabic{subsection})}
% \renewcommand\thesubsubsection{\textit \alph{subsubsection})}

% To start numbering from 0
% \setcounter{section}{-1}


%------page style
\usepackage{fancyhdr}
\usepackage{lastpage}

\setlength{\headheight}{18pt}
\setlength{\footskip}{50pt}

\pagestyle{fancy}
\fancyhf{}
\fancyhead[LE, RO]{\textit{\textcolor{black}{\today}}}
\fancyhead[RE, LO]{\large{\textsl{\Emph{\texttt{\jobname}}}}}

\fancyfoot[RO, LE]{\textit{\texttt{\textcolor{black}{Page \thepage /}\pageref{LastPage}}}}
\fancyfoot[LO, RE]{\includegraphics[scale=0.12]{/home/lasercata/Pictures/1.images_profil/logo/mieux/lasercata_logo_fly_fond_blanc.png}}


%------init lengths
\setlength{\parindent}{0pt} %To avoid using \noindent everywhere.
\setlength{\parskip}{3pt}


%---------------------------------Begin Document
\begin{document}
    
    \thetitle{Maths}{Endomorphismes d'un espace euclidien}
    
    \tableofcontents
    \newpage
    
    Dans tout ce qui suit, on pose $[E, \ps \cdot \cdot]$ un espace euclidien.

    \begin{indt}{\section{Adjoint d'un endomorphisme}}
        \begin{indt}{\subsection{Adjoint}}
            \begin{indt}{\subsubsection{Définition (\textit{Endomorphisme adjoint})}}
                Soit $u \in \mathcal L_\R(E)$.
            
                Alors
                \[
                    \exists ! u^* \in E^E\ |\ \forall x, y \in E,\ \ps{u(x)}{y} = \ps{x}{u^*(y)}
                \]
            
                On appelle alors $u^*$ l'(endomorphisme) \emph{adjoint} de $u$.
            \end{indt}
            
            \vspace{12pt}
            
            \begin{indt}{\subsubsection{Propriété (\textit{Linéarité de l'adjoint})}}
                Soit $u \in \mathcal L_\R(E)$.
            
                Alors
                \[
                    u^* \in \mathcal L_\R(E)
                \]
            \end{indt}
            
            \vspace{12pt}
            
            \begin{indt}{\subsubsection{Propriétés}}
                $\bullet$ L'application
                \[
                    \begin{array}{ccc}
                        \mathcal L_\R(E) & \longrightarrow & \mathcal L_\R(E)
                        \\
                        u & \longmapsto & u^*
                    \end{array}
                \]
                est linéaire et involutive (\textit{i.e} $\forall u \in \mathcal L_\R(E),\ \lr{u^*}^* = u$)
            
                \vspace{6pt}
            
                $\bullet$ $\forall u, v \in \mathcal L_\R(E), \lr{v \circ u}^* = u^* \circ v^*$
            
                \vspace{6pt}
            
                $\bullet$ $\mathrm{id}_E^* = \mathrm{id}_E$
            
                \vspace{6pt}
            
                $\bullet$ $\forall u \in \mathrm{GL}_\R(E),\ u^* \in \mathrm{GL}_\R(E)$, et $\lr{u^*}^{-1} = \lr{u^{-1}}^*$
            \end{indt}
            
            \vspace{12pt}
            
            \begin{indt}{\subsubsection{Proposition (stabilité)}}
                Soit $F$ un sous espace vectoriel de $E$, et $u \in \mathcal L_\R(E)\ |\ u(F) \subset F$.
            
                Alors
                \[
                    u^*(F^\perp) \subset F^\perp
                \]
            \end{indt}
            
            \vspace{12pt}
            
            \begin{indt}{\subsubsection{Proposition (lien avec les matrices)}}
                Soit $\mathcal B$ une base orthonormée de $E$, et $u \in \mathcal L_\R(E)$.
            
                Alors :
                \[
                    \mathrm{Mat}_{\mathcal B}(u^*) = \lr{\mathrm{Mat}_{\mathcal B}(u)}^\top
                \]
            \end{indt}
        \end{indt}

        \vspace{12pt}
        
        \begin{indt}{\subsection{Endomorphisme autoadjoint}}
            \begin{indt}{\subsubsection{Définition (\textit{Endomorphisme autoadjoint})}}
                Soit $u \in \mathcal L_\R(E)$.

                Alors $u$ est dit \emph{autoadjoint} si et seulement si
                \[
                    u = u^*
                \]
                \textit{i.e} si et seulement si
                \[
                    \forall x, y \in E,\
                    \ps{u(x)} y = \ps x {u(x)}
                \]

                \vspace{6pt}
                
                On note alors
                \[
                    \mathscr S(E) = \set{u \in \mathcal L_\R(E)\ |\ u = u^*}
                \]
            \end{indt}

            \vspace{12pt}
            
            \begin{indt}{\subsubsection{Caractérisation matricielle}}
                Soit $\mathcal B$ une base orthonormée de $E$, et $u \in \mathcal L_\R(E)$.

                Alors
                \[
                    u \in \mathscr S(E)
                    \ssi
                    \mathrm{Mat}_{\mathcal B}(u) = \mathrm{Mat}_{\mathcal B}(u)^\top
                \]
            \end{indt}

            \vspace{12pt}
            
            \begin{indt}{\subsubsection{Corollaire}}
                $\mathscr S(E)$ est un sous espace vectoriel de $\mathcal L_\R(E)$, avec :
                \[
                    \dim\!\lr{\mathscr S(E)} = \dfrac{(\dim E)(1 + \dim E)}{2}
                \]
            \end{indt}

            \vspace{12pt}
            
            \begin{indt}{\subsubsection{Théorème spectral}}
                Soit $u \in \mathcal L_\R(E)$, et $n = \dim E$.

                Alors :
                \[
                    \begin{array}{cl}
                        & u \in \mathscr S(E)
                        \vspace{3pt}
                        \\
                        \ssi& \displaystyle
                        E = \bigoplus_{\lambda \in \mathrm{Sp}_\R(u)}^\perp E_\lambda(u)
                        \vspace{6pt}
                        \\
                        \ssi&
                        \exists \mathcal B = \lr{e_k}_{k \in \nset 1 n} \subset \mathrm{Sp}_\R(u)\ |\ \mathcal B\ \text{base orthonormée de}\ E
                    \end{array}
                \]
            \end{indt}

            \begin{indt}{\subsubsection{Définition (\textit{autoadjoint positif, défini positif})}}
                Soit $u \in \mathscr S(E)$.

                \begin{indt}{Alors $u$ est dit :}
                    $\bullet$ \emph{positif} si il vérifie
                    \[
                        \forall x \in E,\ \ps{u(x)} x \ge 0
                    \]

                    $\bullet$ \emph{défini positif} si il vérifie
                    \[
                        \begin{array}{cl}
                            & \forall x \in E \setminus \set 0,\
                            \ps{u(x)} x > 0
                            \vspace{6pt}
                            \\
                            \ssi& \forall x \in E,\
                            \begin{array}{|l}
                                \ps{u(x)} x \ge 0
                                \\
                                \ps{u(x)} x = 0 \Rightarrow x = 0
                            \end{array}
                        \end{array}
                    \]
                \end{indt}

                On définit alors les ensembles suivants :
                \[
                    \begin{array}{rcl}
                        \mathscr S^+(E)
                        &=& \set{u \in \mathscr S(E)\ |\ \forall x \in E,\ \ps{u(x)} x \ge 0}
                        \\
                        \mathscr S^{++}(E)
                        &=& \set{u \in \mathscr S(E)\ |\ \forall x \in E \setminus \set 0,\ \ps{u(x)} x > 0}
                    \end{array}
                \]
            \end{indt}

            \vspace{12pt}
            
            \begin{indt}{\subsubsection{Proposition}}
                Les ensembles $\mathscr S^+(E)$ et $\mathscr S^{++}(E)$ sont convexes (mais ne sont pas des espaces vectoriels).
            \end{indt}

            \vspace{12pt}
            
            \begin{indt}{\subsubsection{Caractérisation spectrale}}
                Soit $u \in \mathscr S(E)$.

                Alors :
                \[
                    \begin{array}{rcl}
                        u \in \mathscr S^+(E)
                        &\ssi&
                        \mathrm{Sp}_\R(u) \subset \R_+
                        \\
                        u \in \mathscr S^{++}(E)
                        &\ssi&
                        \mathrm{Sp}_\R(u) \subset \R_+^*
                    \end{array}
                \]
            \end{indt}
        \end{indt}
    \end{indt}

    \vspace{12pt}
    
    \begin{indt}{\section{Matrices orthogonales}}
        \begin{indt}{\subsection{Définitions}}
            \begin{indt}{\subsubsection{Définition}}
                Soit $n \in \N^*$.

                On note alors
                \[
                    \mathscr S_n(\R) = \set{A \in \mathcal M_n(\R)\ |\ A = A^\top}
                \]
                l'ensemble des matrices symétriques.
            \end{indt}

            \vspace{12pt}
            
            \begin{indt}{\subsubsection{Définition (\textit{matrice positive, définie positive})}}
                Soit $n \in \N^*$, et $A \in \mathscr S_n(\R)$.

                \begin{indt}{On dit alors que $A$ est :}
                    $\bullet$ \emph{positive} si elle vérifie
                    \[
                        \forall X \in \R^n \simeq \mathcal M_{n, 1}(\R),\
                        \underbrace{X^\top A X}_{\in \mathcal M_{1, 1}(\R) \simeq \R} \ge 0
                    \]

                    \vspace{6pt}
                    
                    $\bullet$ \emph{définie positive} si elle vérifie
                    \[
                        \forall X \in \R^n \setminus \set{0},\
                        X^\top A X > 0
                    \]
                \end{indt}

                On définit alors les ensembles suivants :
                \[
                    \begin{array}{rcl}
                        \mathscr S_n^+(\R)
                        &=& \set{A \in \mathscr S_n(\R) \ |\ A\ \text{positive}}
                        \vspace{3pt}
                        \\
                        \mathscr S_n^{++}(\R)
                        &=& \set{A \in \mathscr S_n(\R) \ |\ A\ \text{définie positive}}
                    \end{array}
                \]
            \end{indt}

            \vspace{12pt}
            
            \begin{indt}{\subsubsection{Propriété}}
                Soit $\mathcal B$ une base orthonormée de $E$, $n = \dim E$, et $u \in \mathscr S(E)$.

                Alors :
                \[
                    \begin{array}{rcl}
                        u \in \mathscr S^+(E)
                        &\ssi&
                        \mathrm{Mat}_{\mathcal B}(u) \in \mathscr S_n^+(\R)
                        \vspace{3pt}
                        \\
                        u \in \mathscr S^{++}(E)
                        &\ssi&
                        \mathrm{Mat}_{\mathcal B}(u) \in \mathscr S_n^{++}(\R)
                    \end{array}
                \]
            \end{indt}
        \end{indt}

        \vspace{12pt}
        
        \begin{indt}{\subsection{Matrices orthogonales}}
            \begin{indt}{\subsubsection{Définition (\textit{matrice orthogonale})}}
                Soit $n \in \N^*$, et $A \in \mathcal M_n(\R)$ (carrée !).

                Alors $A$ est dite \emph{orthogonale} si et seulement si
                \[
                    A^\top A = I_n
                \]

                On note alors
                \[
                    O_n(\R) = O(n) = \set{A \in \mathcal M_n(\R)\ |\ A^\top A = I_n}
                \]
            \end{indt}

            \vspace{12pt}
            
            \begin{indt}{\subsubsection{Caractérisation des matrices orthogonales}}
                Soit $n \in \N^*$, et $A \in \mathcal M_n(\R)$.

                On note $\lr{C_k}_{k \in \nset 1 n}$ les colonnes de $A$, et $\lr{L_k}_{k \in \nset 1 n}$ ses lignes.

                On a :
                \[
                    \begin{array}{rcl}
                        A \in O(n)
                        &\ssi&
                        A^\top A = I_n
                        \\
                        &\ssi& A A^\top = I_n
                        \\
                        &\ssi&
                        \begin{cases}
                            A \in \mathrm{GL}_n(\R)
                            \\
                            A^{-1} = A^\top
                        \end{cases}
                        \vspace{3pt}
                        \\
                        &\ssi&
                        \lr{C_k}_{k \in \nset 1 n}\ \text{famille (base) orthonormée de}\ \R^n \simeq \mathcal M_{n, 1}(\R)
                        \\
                        &\ssi&
                        \lr{L_k}_{k \in \nset 1 n}\ \text{famille (base) orthonormée de}\ \R^n \simeq \mathcal M_{1, n}(\R)
                    \end{array}
                \]
            \end{indt}

            \vspace{12pt}
            
            \begin{indt}{\subsubsection{Corollaire}}
                Soit $n \in \N^*$, et $A \in O(n)$.

                Alors
                \[
                    \det(A) \in \set{\pm 1}
                \]
            \end{indt}

            \vspace{12pt}
            
            \begin{indt}{\subsubsection{Caractérisation (par les matrices de passage)}}
                Soit $\mathcal B$ une base de $E$.

                Alors :
                \[
                    \forall A \in O(\dim E),\
                    \exists! \mathcal B'\ \text{base orthonormée de}\ E\ |\
                    A = \mathcal P_{\mathcal B}^{\mathcal B'}
                    = \mathrm{Mat}_{\mathcal B}(\mathcal B')
                \]
            \end{indt}
        \end{indt}

        \vspace{12pt}
        
        \begin{indt}{\subsection{Groupe orthogonal}}
            \begin{indt}{\subsubsection{Définition (\textit{groupe spécial orthogonal})}}
                On définit, $\forall n \in \N^*$,
                \[
                    SO(n) = SO_n(\R) = \set{A \in O_n(\R)\ |\ \det(A) = 1}
                \]
            \end{indt}

            \vspace{12pt}
            
            \begin{indt}{\subsubsection{Propriété}}
                Soit $n \in \N^*$.

                \begin{indt}{Alors :}
                    $\bullet$ $O(n)$ est un sous-groupe de $\mathrm{GL}_n(\R)$, appelé \emph{groupe orthogonal} ;

                    $\bullet$ $SO(n)$ est un sous-groupe de $O(n)$, appelé \emph{groupe spécial orthogonal}.
                \end{indt}
            \end{indt}

            \vspace{12pt}
            
            \begin{indt}{\subsubsection{Définition (\textit{matrice directe, indirecte})}}
                Soit $A \in O(n)$.

                \begin{indt}{Alors $A$ est dite :}
                    $\bullet$ \emph{directe} si et seulement si
                    \[
                        A \in SO(n)
                    \]

                    $\bullet$ \emph{indirecte} si et seulement si
                    \[
                        A \in O(n) \setminus SO(n)
                    \]
                \end{indt}
            \end{indt}

            \vspace{12pt}
            
            \begin{indt}{\subsubsection{Proposition (morphisme de $[\R, +]$ dans $[SO(2), \times]$)}}
                L'application
                \[
                    \begin{array}{ccccl}
                        \varphi & : & \R & \longrightarrow & SO(2)
                        \\
                        &&t & \longmapsto &
                        R_t =
                        \begin{pmatrix}
                            \cos t & -\sin t
                            \\
                            \sin t & \cos t
                        \end{pmatrix}
                    \end{array}
                \]
                est un morphisme de groupes de $[\R, +]$ dans $[SO(2), \times]$ surjectif, et
                \[
                    \ker \varphi = 2\pi \Z
                \]
            \end{indt}

            \vspace{12pt}
            
            \begin{indt}{\subsubsection{Corollaires}}
                $\bullet$ Le groupe $SO(2)$ est commutatif.

                \vspace{12pt}
                
                $\bullet$ L'application
                \[
                    \begin{array}{ccccl}
                        \psi & : & \U & \longrightarrow & SO(2)
                        \\
                         && z & \longmapsto &
                         \begin{pmatrix}
                             \Re(z) & -\Im(z)
                             \\
                             \Im(z) & \Re(z)
                         \end{pmatrix}
                    \end{array}
                \]
                est un isomorphisme de groupes de $[\U, \times]$ dans $[SO(2), \times]$.
            \end{indt}
        \end{indt}

        \vspace{12pt}
        
        \begin{indt}{\subsection{Orientation d'un espace vectoriel réel de dimension finie}}
            \begin{indt}{\subsubsection{Définition (cas général)}}
                Soit $E'$ un $\R$-espace vectoriel de dimension finie.

                Alors \emph{orienter} $E'$ signifie choisir une base $\mathcal B$ de $E'$ qui servira de référence : $\mathcal B$ définit le \emph{sens direct}.

                \begin{indt}{De plus, $\forall \mathcal B'$ base de $E'$,}
                    $\bullet$ soit $\mathcal B'$ est \emph{directe}, lorsqu'elle vérifie
                    \[
                        \det\!\lr{\mathcal P_{\mathcal B}^{\mathcal B'}} = \mathrm{det}_{\mathcal B}(\mathcal B') > 0
                    \]

                    $\bullet$ sinon, \textit{i.e} si
                    \[
                        \det\!\lr{\mathcal P_{\mathcal B}^{\mathcal B'}} = \mathrm{det}_{\mathcal B}(\mathcal B') < 0
                    \]
                    alors $\mathcal B'$ est dite \emph{indirecte}.
                \end{indt}
            \end{indt}

            \vspace{12pt}
            
            \begin{indt}{\subsubsection{Propriétés (orientation d'un espace euclidien)}}
                $\bullet$ Soient $\mathcal B$, $\mathcal B'$ deux bases de $E$ (espace euclidien), et $n = \dim E$.

                Alors $\mathcal B$ et $\mathcal B'$ définissent la même orientation de $E$ si et seulement si
                \[
                    \mathcal P_{\mathcal B}^{\mathcal B'} \in SO(n)
                \]

                \vspace{12pt}
                
                $\bullet$ Soient $\mathcal B$ et $\mathcal B'$ deux bases directes de $E$.

                Alors
                \[
                    \mathrm{det}_{\mathcal B} = \mathrm{det}_{\mathcal B'}
                \]

                On note parfois $\mathrm{Det}$ cette application.

                \vspace{12pt}
                
                $\bullet$ Soit $H$ un hyperplan de $E$.

                Alors on peut orienter $H$ en choisissant une normale $\vec n$ de $H$, \textit{i.e} un $\vec n \in E\ |\ \set{\vec n}^\perp = H$. Une base $\mathcal B_H$ de $H$ est directe si et seulement si la base $(\mathcal B_H, \vec n)$ de $E$ est directe. 
            \end{indt}
        \end{indt}

        \vspace{12pt}
        
        \begin{indt}{\subsection{Théorème spectral matriciel}}
            \begin{indt}{\subsubsection{Définition (\textit{matrices orthogonalement semblables})}}
                Soit $n \in \N^*$, et $A, B \in \mathcal M_n(\R)$.

                Alors $A$ et $B$ sont dites \emph{orthogonalement semblables} si, et seulement si
                \[
                    \exists P \in O(n) \mid A = P^{-1}BP = P^\top B P
                \]

                \vspace{6pt}
                
                \begin{indt}{Remarques :}
                    $-$ Comme $O(n)$ est un groupe, la relation \simplecit{être semblable} définit une relation d'équivalence.

                    $-$ Deux matrices sont orthogonalement semblables si et seulement si elles représentent le même endomorphisme dans deux bases orthonormées.
                \end{indt}
            \end{indt}

            \vspace{12pt}
            
            \begin{indt}{\subsubsection{Théorème spectral matriciel}}
                Soit $n \in \N^*$, et $A \in \mathcal M_n(\R)$.

                Alors :
                \[
                    A \in \mathscr S_n(\R)
                    \ssi
                    \exists P \in O(n)\ |\ P^{-1} A P = P^\top A P\ \text{diagonale}
                \]
            \end{indt}
        \end{indt}
    \end{indt}

    \vspace{12pt}
    
    \begin{indt}{\section{Isométries vectorielles}}
        \begin{indt}{\subsection{Définitions}}
            \begin{indt}{\subsubsection{Définition (\textit{isométrie vectorielle})}}
                Soit $f \in \mathcal L_\R(E)$.

                Alors $f$ est une \emph{isométrie vectorielle} si et seulement si elle conserve la norme, \textit{i.e} si et seulement si
                \[
                    \forall x \in E,\
                    \norm{f(x)} = \norm x
                \]
            \end{indt}

            \vspace{12pt}
            
            \begin{indt}{\subsubsection{Propriété (\textit{symétries orthogonales})}}
                Les symétries orthogonales de $E$, donc en particulier les réflexions de $E$ (symétries orthogonales par rapport à un hyperplan) sont des isométries vectorielles.
            \end{indt}
        \end{indt}

        \vspace{12pt}
        
        \begin{indt}{\subsection{Caractérisation}}
            Soit $u \in \mathcal L_\R(E)$, et $n = \dim E$.

            Alors :
            \[
                \begin{array}{rcl}
                    \forall x \in E,\ \norm{u(x)} = \norm x
                    &\ssi&
                    \forall x, y \in E,\ \ps{u(x)}{u(y)} = \ps x y
                    \vspace{3pt}
                    \\
                    &\ssi& \exists \mathcal B\ \text{base orthonormée de}\ E\ |\ u(\mathcal B)\ \text{base orthonormée}
                    \vspace{3pt}
                    \\
                    &\ssi& \forall \mathcal B\ \text{base orthonormée de}\ E,\ u(\mathcal B)\ \text{base orthonormée}
                    \vspace{3pt}
                    \\
                    &\ssi&
                    \begin{cases}
                        u \in \mathrm{GL}(E)
                        \\
                        u^{-1} = u^*
                    \end{cases}
                    \vspace{3pt}
                    \\
                    &\ssi&
                    u^* \circ u = \mathrm{id}_E
                    \vspace{3pt}
                    \\
                    &\ssi& u^* \circ u = u \circ u^* = \mathrm{id}_E
                    \vspace{3pt}
                    \\
                    &\ssi& \exists \mathcal B\ \text{base orthonormée de}\ E\ |\ \mathrm{Mat}_{\mathcal B}(u) \in O(n)
                    \vspace{3pt}
                    \\
                    &\ssi& \forall \mathcal B\ \text{base orthonormée de}\ E\ |\ \mathrm{Mat}_{\mathcal B}(u) \in O(n)
                \end{array}
            \]
        \end{indt}

        \vspace{12pt}
        
        \begin{indt}{\subsection{Groupe orthogonal}}
            \begin{indt}{\subsubsection{Propriété (\textit{déterminant d'une isométrie})}}
                Soit $u \in \mathcal L_\R(E)$ une isométrie.

                Alors
                \[
                    \det(f) \in \set{\pm 1}
                \]
            \end{indt}

            \vspace{12pt}
            
            \begin{indt}{\subsubsection{Définition}}
                On définit les ensembles :
                \[
                    \begin{array}{rcl}
                        O(E)
                        &=& \set{f \in \mathcal L_\R(E)\ |\ \forall x \in E,\ \norm{f(x)} = \norm x}
                        \\
                        SO(E)
                        &=& \set{f \in O(E)\ |\ \det(f) = 1}
                    \end{array}
                \]
            \end{indt}

            \vspace{12pt}
            
            \begin{indt}{\subsubsection{Propriété (structure de groupe)}}
                $\bullet$ L'ensemble $O(E)$ est un sous-groupe de $\mathrm{GL}(E)$, appelé \emph{groupe orthogonal} de $E$.

                $\bullet$ L'ensemble $SO(E)$ est un sous-groupe de $O(E)$, appelé \emph{groupe spécial orthogonal}.
            \end{indt}

            \vspace{12pt}
            
            \begin{indt}{\subsubsection{Définition (\textit{isométrie directe, indirecte})}}
                Soit $u \in O(E)$.

                \begin{indt}{Alors $u$ est dite :}
                    $\bullet$ \emph{directe} (on dit aussi que $u$ est une rotation) si et seulement si
                    \[
                        \det(u) = 1 \ssi u \in SO(E)
                    \]

                    $\bullet$ \emph{indirecte} sinon, \textit{i.e} si et seulement si
                    \[
                        \det(u) = -1 \ssi u \in O(E) \setminus SO(E)
                    \]
                \end{indt}
            \end{indt}

            \vspace{12pt}
            
            \begin{indt}{\subsubsection{Propriété}}
                Soit $\theta \in \R$, et $R_\theta, S_\theta \in \mathcal M_2(\R)$ définies par :
                \[
                    \begin{array}{rcl}
                        R_\theta
                        &=&
                        \begin{pmatrix}
                            \cos \theta & -\sin \theta
                            \\
                            \sin \theta & \cos \theta
                        \end{pmatrix}
                        \vspace{3pt}
                        \\
                        S_\theta
                        &=&
                        \begin{pmatrix}
                            \cos \theta & \sin \theta
                            \\
                            \sin \theta & -\cos \theta
                        \end{pmatrix}
                    \end{array}
                \]

                Alors on a :
                \[
                    \begin{array}{rcl}
                        SO(2)
                        &=& \set{R_\theta\ |\ \theta \in \R}
                        \\
                        O(2) \setminus SO(2)
                        &=& \set{S_\theta\ |\ \theta \in \R}
                    \end{array}
                \]
            \end{indt}

            \vspace{12pt}
            
            \begin{indt}{\subsubsection{Propriété (\textit{rotation vectorielle d'un plan euclidien})}}
                Soit $[P, \ps \cdot \cdot]$ un plan euclidien (donc $\dim P = 2$), et $r \in SO(P)$, \textit{i.e} $r$ est une rotation de $P$.

                Alors
                \[
                    \forall \mathcal B, \mathcal B'\ \text{bases orthonormées de}\ P,\
                    \mathrm{Mat}_{\mathcal B}(r) = \mathrm{Mat}_{\mathcal B'}(r)
                \]

                et, avec $\mathcal B$ une base orthonormée de $P$,
                \[
                    \exists \theta \in \R\ |\
                    \mathrm{Mat}_{\mathcal B}(r) =
                    \begin{pmatrix}
                        \cos \theta & -\sin \theta
                        \\
                        \sin \theta & \cos \theta
                    \end{pmatrix}
                \]
                avec $\theta$ unique modulo $2\pi$.
            \end{indt}
        \end{indt}

        \vspace{12pt}
        
        \begin{indt}{\subsection{Réduction}}
            \begin{indt}{\subsubsection{Propriété (stabilité)}}
                Soit $F$ un sous espace vectoriel de $E$, et $u \in O(E)$ une isométrie.
                
                Alors :
                \[
                    u(F) \subset F
                    \Rightarrow u(F^\perp) \subset F^\perp
                \]
            \end{indt}

            \vspace{12pt}
            
            \begin{indt}{\subsubsection{Théorème de réduction d'une matrice orthogonale}}
                Soit $n \in \N^*$.

                \[
                    \forall A \in O(n),\
                    \left|
                    \begin{array}{l}
                        \exists P \in O(n)
                        \vspace{3pt}
                        \\
                        \exists r, p, q \in \N
                        \vspace{3pt}
                        \\
                        \exists \lr{\theta_k}_{k \in \nset 1 q} \in \lr{\R \setminus \pi \Z}^q
                    \end{array}
                    \right.
                    \text{tels que}
                \]

                \[
                    A =
                    P^{-1}
                    \left(
                    \begin{array}{c|c|ccc}
                        I_r & 0 && 0
                        \\
                        \hline
                        0 & -I_p && 0
                        \\
                        \hline
                        && R_{\theta_1} && (0)
                        \\
                        0 & 0 && \ddots
                        \\
                        && (0) && R_{\theta_q}
                    \end{array}
                    \right)
                    P
                \]

                où $\forall \theta \in \R$,
                \[
                    R_\theta =
                    \begin{pmatrix}
                        \cos \theta & -\sin \theta
                        \\
                        \sin \theta & \cos \theta
                    \end{pmatrix}
                \]
            \end{indt}

            \vspace{12pt}
            
            \begin{indt}{\subsubsection{Théorème de réduction des isométries}}
                Soit $u \in O(E)$ une isométrie.

                Alors :
                \[
                    \exists \mathcal B\ \text{base orthonormée de}\ E\
                    \left|
                    \begin{array}{l}
                        \exists r, p, q \in \N
                        \vspace{3pt}
                        \\
                        \exists \lr{\theta_k}_{k \in \nset 1 q} \in \lr{\R \setminus \pi \Z}^q
                    \end{array}
                    \right.
                    \text{tels que}
                \]

                \[
                    \mathrm{Mat}_{\mathcal B}(u) =
                    \left(
                    \begin{array}{c|c|ccc}
                        I_r & 0 && 0
                        \\
                        \hline
                        0 & -I_p && 0
                        \\
                        \hline
                        && R_{\theta_1} && (0)
                        \\
                        0 & 0 && \ddots
                        \\
                        && (0) && R_{\theta_q}
                    \end{array}
                    \right)
                \]

                où $\forall \theta \in \R$,
                \[
                    R_\theta =
                    \begin{pmatrix}
                        \cos \theta & -\sin \theta
                        \\
                        \sin \theta & \cos \theta
                    \end{pmatrix}
                \]
            \end{indt}
        \end{indt}
    \end{indt}
    
\end{document}
%--------------------------------------------End
