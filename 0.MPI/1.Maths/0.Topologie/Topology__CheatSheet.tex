\documentclass[a4paper, 12pt, twoside]{article}


%------------------------------------------------------------------------
%
% Author                :   Lasercata
% Last modification     :   2023.11.03
%
%------------------------------------------------------------------------


%------ini
\usepackage[utf8]{inputenc}
\usepackage[T1]{fontenc}
%\usepackage[french]{babel}
%\usepackage[english]{babel}


%------geometry
\usepackage[textheight=700pt, textwidth=500pt]{geometry}


%------color
\usepackage{xcolor}
\definecolor{ff4500}{HTML}{ff4500}
\definecolor{00f}{HTML}{0000ff}
\definecolor{0ff}{HTML}{00ffff}
\definecolor{656565}{HTML}{656565}

%\renewcommand{\emph}{\textcolor{ff4500}}
%\renewcommand{\em}{\color{ff4500}}

\newcommand{\Emph}{\textcolor{ff4500}}

\newcommand{\strong}[1]{\textcolor{ff4500}{\bf #1}}
\newcommand{\st}{\color{ff4500}\bf}


%------Code highlighting
%---listings
\usepackage{listings}

\definecolor{cbg}{HTML}{272822}
\definecolor{cfg}{HTML}{ececec}
\definecolor{ccomment}{HTML}{686c58}
\definecolor{ckw}{HTML}{f92672}
\definecolor{cstring}{HTML}{e6db72}
\definecolor{cstringlight}{HTML}{98980f}
\definecolor{lightwhite}{HTML}{fafafa}

\lstdefinestyle{DarkCodeStyle}{
    backgroundcolor=\color{cbg},
    commentstyle=\itshape\color{ccomment},
    keywordstyle=\color{ckw},
    numberstyle=\tiny\color{cbg},
    stringstyle=\color{cstring},
    basicstyle=\ttfamily\footnotesize\color{cfg},
    breakatwhitespace=false,
    breaklines=true,
    captionpos=b,
    keepspaces=true,
    numbers=left,
    numbersep=5pt,
    showspaces=false,
    showstringspaces=false,
    showtabs=false,
    tabsize=4,
    xleftmargin=\leftskip
}

\lstdefinestyle{LightCodeStyle}{
    backgroundcolor=\color{lightwhite},
    commentstyle=\itshape\color{ccomment},
    keywordstyle=\color{ckw},
    numberstyle=\tiny\color{cbg},
    stringstyle=\color{cstringlight},
    basicstyle=\ttfamily\footnotesize\color{cbg},
    breakatwhitespace=false,
    breaklines=true,
    captionpos=b,
    keepspaces=true,
    numbers=left,
    numbersep=10pt,
    showspaces=false,
    showstringspaces=false,
    showtabs=false,
    tabsize=4,
    frame=L,
    xleftmargin=\leftskip
}

%\lstset{style=DarkCodeStyle}
\lstset{style=LightCodeStyle}
%Usage : \begin{lstlisting}[language=Caml, xleftmargin=xpt] ... \end{lstlisting}


%---Algorithm
\usepackage[linesnumbered,ruled,vlined]{algorithm2e}
\SetKwInput{KwInput}{Input}
\SetKwInput{KwOutput}{Output}

\SetKwProg{Fn}{Function}{:}{}
\SetKw{KwPrint}{Print}

\newcommand\commfont[1]{\textit{\texttt{\textcolor{656565}{#1}}}}
\SetCommentSty{commfont}
\SetProgSty{texttt}
\SetArgSty{textnormal}
\SetFuncArgSty{textnormal}
%\SetProgArgSty{texttt}

\newenvironment{indalgo}[2][H]{
    \begin{algoBox}
        \begin{algorithm}[#1]
            \caption{#2}
}
{
        \end{algorithm}
    \end{algoBox}
}


%---tcolorbox
\usepackage[many]{tcolorbox}
\DeclareTColorBox{emphBox}{O{black}O{lightwhite}}{
    breakable,
    outer arc=0pt,
    arc=0pt,
    top=0pt,
    toprule=-.5pt,
    right=0pt,
    rightrule=-.5pt,
    bottom=0pt,
    bottomrule=-.5pt,
    colframe=#1,
    colback=#2,
    enlarge left by=10pt,
    width=\linewidth-\leftskip-10pt,
}

\DeclareTColorBox{algoBox}{O{black}O{lightwhite}}{
    breakable,
    arc=0pt,
    top=0pt,
    toprule=-.5pt,
    right=0pt,
    rightrule=-.5pt,
    bottom=0pt,
    bottomrule=-.5pt,
    left=0pt,
    leftrule=-.5pt,
    colframe=#1,
    colback=#2,
    width=\linewidth-\leftskip-10pt,
}


%-------make the table of content clickable
\usepackage{hyperref}
\hypersetup{
    colorlinks,
    citecolor=black,
    filecolor=black,
    linkcolor=black,
    urlcolor=black
}


%------pictures
\usepackage{graphicx}
%\usepackage{wrapfig}

\usepackage{tikz}
%\usetikzlibrary{babel}             %Uncomment this to use circuitikz
%\usetikzlibrary{shapes.geometric}  % To draw triangles in trees
%\usepackage{circuitikz}            %Electrical circuits drawing


%------tabular
%\usepackage{color}
%\usepackage{colortbl}
%\usepackage{multirow}


%------Physics
%---Packages
%\usepackage[version=4]{mhchem} %$\ce{NO4^2-}$

%---Commands
\newcommand{\link}[2]{\mathrm{#1} \! - \! \mathrm{#2}}
\newcommand{\pt}[1]{\cdot 10^{#1}} % Power of ten
\newcommand{\dt}[2][t]{\dfrac{\mathrm d #2}{\mathrm d #1}} % Derivative


%------math
%---Packages
%\usepackage{textcomp}
%\usepackage{amsmath}
\usepackage{amssymb}
\usepackage{mathtools} % For abs
\usepackage{stmaryrd} %for \llbracket and \rrbracket
\usepackage{mathrsfs} %for \mathscr{x} (different from \mathcal{x})
\usepackage{yhmath} %For \widering

%---Commands
%-Sets
\newcommand{\N}{\mathbb{N}} %set N
\newcommand{\Z}{\mathbb{Z}} %set Z
\newcommand{\Q}{\mathbb{Q}} %set Q
\newcommand{\R}{\mathbb{R}} %set R
\newcommand{\C}{\mathbb{C}} %set C
\newcommand{\U}{\mathbb{U}} %set U
\newcommand{\K}{\mathbb K}
\newcommand{\seg}[2]{\left[ #1\ ;\ #2 \right]}
\newcommand{\nset}[2]{\left\llbracket #1\ ;\ #2 \right\rrbracket}

%-Exponantial / complexs
\newcommand{\e}{\mathrm{e}}
\newcommand{\cj}[1]{\overline{#1}} %overline for the conjugate.

%-Vectors
\newcommand{\vect}{\overrightarrow}
\newcommand{\veco}[3]{\displaystyle \vect{#1}\binom{#2}{#3}} %vector + coord

%-Limits
\newcommand{\lm}[2][{}]{\lim\limits_{\substack{#2 \\ #1}}} %$\lm{x \to a} f$ or $\lm[x < a]{x \to a} f$
\newcommand{\Lm}[3][{}]{\lm[#1]{#2} \left( #3 \right)} %$\Lm{x \to a}{f}$ or $\Lm[x < a]{x \to a}{f}$
\newcommand{\tendsto}[1]{\xrightarrow[#1]{}}

%-Integral
\newcommand{\dint}[4][x]{\displaystyle \int_{#2}^{#3} #4 \mathrm{d} #1} %$\dint{a}{b}{f(x)}$ or $\dint[t]{a}{b}{f(t)}$

%-left right
\newcommand{\lr}[1]{\left( #1 \right)}
\newcommand{\lrb}[1]{\left[ #1 \right]}
\newcommand{\lrbb}[1]{\left\llbracket #1 \right\rrbracket}
\newcommand{\set}[1]{\left\{ #1 \right\}}
\newcommand{\abs}[1]{\left\lvert #1 \right\rvert}
\newcommand{\norm}[1]{\left\lVert #1 \right\rVert}
\newcommand{\nnorm}[2][-2pt]{{\left\lvert \kern#1 \left\lvert \kern#1 \left\lvert #2 \right\rvert \kern#1 \right\rvert \kern#1 \right\rvert}}
\newcommand{\ceil}[1]{\left\lceil #1 \right\rceil}
\newcommand{\floor}[1]{\left\lfloor #1 \right\rfloor}
\newcommand{\lrangle}[1]{\left\langle #1 \right\rangle}

%-Others
\newcommand{\para}{\ /\!/\ } %//
\newcommand{\ssi}{\ \Leftrightarrow \ }
\newcommand{\eqsys}[2]{\begin{cases} #1 \\ #2 \end{cases}}

\newcommand{\med}[2]{\mathrm{med} \left[ #1\ ;\ #2 \right]}  %$\med{A}{B} -> med[A ; B]$
\newcommand{\Circ}[2]{\mathscr{C}_{#1, #2}}

\renewcommand{\le}{\leqslant}
\renewcommand{\ge}{\geqslant}

\newcommand{\oboxed}[1]{\textcolor{ff4500}{\boxed{\textcolor{black}{#1}}}} %orange boxed

\newcommand{\rboxed}[1]{\begin{array}{|c} \hline #1 \\ \hline \end{array}} %boxed with right opened
\newcommand{\lboxed}[1]{\begin{array}{c|} \hline #1 \\ \hline \end{array}} %boxed with left opened

\newcommand{\orboxed}[1]{\textcolor{ff4500}{\rboxed{\textcolor{black}{#1}}}} %orange right boxed
\newcommand{\olboxed}[1]{\textcolor{ff4500}{\lboxed{\textcolor{black}{#1}}}} %orange left boxed


%------commands
%---to quote
\newcommand{\simplecit}[1]{\guillemotleft$\;$#1$\;$\guillemotright}
\newcommand{\cit}[1]{\simplecit{\textcolor{656565}{#1}}}
\newcommand{\quo}[1]{\cit{\it #1}}

%---to indent
\newcommand{\ind}[1][20pt]{\advance\leftskip + #1}
\newcommand{\deind}[1][20pt]{\advance\leftskip - #1}

%---to indent a text
\newcommand{\indented}[2][20pt]{\par \ind[#1] #2 \par \deind[#1]}
\newenvironment{indt}[2][20pt]{#2 \par \ind[#1]}{\par \deind} %Titled indented env

%---title
\newcommand{\thetitle}[2]{\begin{center}\textbf{{\LARGE \underline{\Emph{#1} :}} {\Large #2}}\end{center}}

%---Maths environments
%-Proofs
\newenvironment{proof}[1][{}]{\begin{indt}{$\square$ #1}}{$\blacksquare$ \end{indt}}

%-Maths parts (proposition, definition, ...)
\newenvironment{mathpart}[1]{\begin{indt}{\boxed{\text{\textbf{#1}}}}}{\end{indt}}
\newenvironment{mathbox}[1]{\boxed{\text{\textbf{#1}}}\begin{emphBox}}{\end{emphBox}}
\newenvironment{mathul}[1]{\begin{indt}{\underline{\textbf{#1}}}}{\end{indt}}

\newenvironment{theo}{\begin{mathpart}{Théorème}}{\end{mathpart}}
\newenvironment{Theo}{\begin{mathbox}{Théorème}}{\end{mathbox}}

\newenvironment{prop}{\begin{mathpart}{Proposition}}{\end{mathpart}}
\newenvironment{Prop}{\begin{mathbox}{Proposition}}{\end{mathbox}}
\newenvironment{props}{\begin{mathpart}{Propriétés}}{\end{mathpart}}

\newenvironment{defi}{\begin{mathpart}{Définition}}{\end{mathpart}}
\newenvironment{meth}{\begin{mathpart}{Méthode}}{\end{mathpart}}

\newenvironment{Rq}{\begin{mathul}{Remarque :}}{\end{mathul}}
\newenvironment{Rqs}{\begin{mathul}{Remarques :}}{\end{mathul}}

\newenvironment{Ex}{\begin{mathul}{Exemple :}}{\end{mathul}}
\newenvironment{Exs}{\begin{mathul}{Exemples :}}{\end{mathul}}


%------Sections
% To change section numbering :
% \renewcommand\thesection{\Roman{section}}
% \renewcommand\thesubsection{\arabic{subsection})}
% \renewcommand\thesubsubsection{\textit \alph{subsubsection})}

% To start numbering from 0
% \setcounter{section}{-1}


%------page style
\usepackage{fancyhdr}
\usepackage{lastpage}

\setlength{\headheight}{18pt}
\setlength{\footskip}{50pt}

\pagestyle{fancy}
\fancyhf{}
\fancyhead[LE, RO]{\textit{\textcolor{black}{\today}}}
\fancyhead[RE, LO]{\large{\textsl{\Emph{\texttt{\jobname}}}}}

\fancyfoot[RO, LE]{\textit{\texttt{\textcolor{black}{Page \thepage /}\pageref{LastPage}}}}
\fancyfoot[LO, RE]{\includegraphics[scale=0.12]{/home/lasercata/Pictures/1.images_profil/logo/mieux/lasercata_logo_fly_fond_blanc.png}}


%------init lengths
\setlength{\parindent}{0pt} %To avoid using \noindent everywhere.
\setlength{\parskip}{3pt}


%---------------------------------Begin Document
\begin{document}
    
    \thetitle{Maths}{Topologie}
    
    \tableofcontents
    \newpage

    Dans tout ce qui suit, on note $\K$ le corps $\R$ ou $\C$.
    
    \begin{indt}{\section{Norme}}
        \begin{indt}{\subsection{Définition (\textit{norme})}}
            Soit $E$ un $\K$-espace vectoriel.

            Une \emph{norme} sur $E$ est une application
            \[
                N : E \longrightarrow \R
            \]

            \begin{indt}{vérifiant :}
                $\bullet$ la séparation :
                \[
                    \forall x \in E,\ N(x) = 0 \Rightarrow x = 0_E
                \]

                $\bullet$ l'homogénéité :
                \[
                    \forall (\lambda, x) \in \K \times E,\ N(\lambda x) = \abs \lambda N(x)
                \]

                $\bullet$ l'inégalité triangulaire :
                \[
                    \forall (x, y) \in E^2,\ N(x + y) \le N(x) + N(y)
                \]
            \end{indt}
        \end{indt}

        \vspace{12pt}
        
        \begin{indt}{\subsection{Proposition (\textit{inégalité triangulaire renversée})}}
            Soit $E$ un $\K$-espace vectoriel, et $N$ une norme sur $E$.

            On a :
            \[
                \forall (x, y) \in E^2,\ \abs{N(x) - N(y)} \le N(x + y)
            \]
        \end{indt}

        \vspace{12pt}
        
        \begin{indt}{\subsection{Convexité}}
            \begin{indt}{\subsubsection{Définition (\textit{segment})}}
                Soit $E$ un $\R$-espace vectoriel, et $A, B \in E$.
                
                Le \emph{segment} $[A, B]$ est définit par :
                \[
                    \begin{array}{rcl}
                        [A, B] &=& \set{A + \lambda(B - A) \ |\ \lambda \in \seg 0 1}
                        \\
                               &=& \set{\lambda A + (1 - \lambda)B\ |\ \lambda \in \seg 0 1}
                    \end{array}
                \]
            \end{indt}

            \vspace{12pt}
            
            \begin{indt}{\subsubsection{Définition (\textit{Partie convexe})}}
                Soit $E$ un $\R$-espace vectoriel, et $\mathcal A \in \mathcal P(E)$.

                Alors $\mathcal A$ est dite \emph{convexe} si, et seulement si
                \[
                    \forall (A, B) \in \mathcal A,\ [A, B] \in \mathcal A
                \]
            \end{indt}
        \end{indt}

        \vspace{12pt}
        
        \begin{indt}{\subsection{Normes usuelles}}
            \begin{indt}{\subsubsection{Sur $\K^n$}}
                Soit $n \in \N^*$.

                On définit, $\forall x = (x_1, \ldots, x_n) \in \K^n$ :
                \[
                    \begin{array}{lcl}
                        \norm{x}_1
                        &=& \displaystyle
                        \sum_{k = 1}^n \abs{x_k}
                        \vspace{6pt}
                        \\
                        \norm{x}_2
                        &=& \displaystyle
                        \sqrt{\sum_{k = 1}^n \abs{x_k}^2}
                        \vspace{6pt}
                        \\
                        \norm{x}_\infty
                        &=& \displaystyle
                        \max_{k \in \nset 1 n} \abs{x_k}
                    \end{array}
                \]

                Ces applications sont des normes sur $\K^n$.
            \end{indt}

            \vspace{12pt}
            
            \begin{indt}{\subsubsection{Sur $\mathcal C^0\!\lr{[a, b], \K}$}}
                Soient $a, b \in \R$.

                On peut munir le $\K$-espace vectoriel $\mathcal C^0\:\lr{[a, b], \K}$ de normes définies par, $\forall f \in \mathcal C^0\:\lr{[a, b], \K}$ :
                \[
                    \begin{array}{lcl}
                        \norm{x}_1
                        &=& \displaystyle
                        \dint a b {\abs{f(x)}}
                        \vspace{6pt}
                        \\
                        \norm{x}_2
                        &=& \displaystyle
                        \sqrt{\dint a b {\abs{f(x)}^2}}
                        \vspace{6pt}
                        \\
                        \norm{x}_\infty
                        &=& \displaystyle
                        \sup_{x \in [a, b]} \abs{f(x)}
                    \end{array}
                \]
            \end{indt}
        \end{indt}

        \vspace{12pt}
        
        \begin{indt}{\subsection{Fonctions lipschitziennes}}
            Soient $\lrb{E, \norm \cdot _E}, \lrb{F, \norm \cdot _F}$ des $\K$-espaces vectoriels normés, $A \in \mathcal P(E) \setminus \varnothing$, et $f \in F^A$.

            Alors $f$ est dite $k$-lipschitzienne si et seulement si
            \[
                \exists k \in \R_+\ |\ \forall x, y \in A,\
                \norm{f(y) - f(x)}_F \le k\norm{y - x}_E
            \]
        \end{indt}

        \vspace{12pt}
        
        \begin{indt}{\subsection{Définition (\textit{normes équivalentes})}}
            Soit $E$ un $\K$-espace vectoriel, et $N, \widetilde N$ deux normes sur $E$.

            Les normes $N$ et $\widetilde N$ sont \emph{équivalentes} si et seulement si
            \[
                \begin{array}{cl}
                    &
                    \exists k, \widetilde k \in \R^*_+\ |
                    \forall x \in E,\
                    \begin{array}{|l}
                        N(x) \le \widetilde k \widetilde N(x)
                        \\
                        \widetilde N(x) \le k N(x)
                    \end{array}
                    \vspace{6pt}
                    \\
                    \ssi&
                    \exists k, k' \in \R^*_+\ |\
                    \forall x \in E,\
                    k' \widetilde N(x) \le N(x) \le k \widetilde N(x)
                    \vspace{6pt}
                    \\
                    \ssi&
                    \exists k \in \R^*_+\ |\
                    \forall x \in E,\
                    \dfrac{1}{k} \widetilde N(x) \le N(x) \le k \widetilde N(x)
                \end{array}
            \]
        \end{indt}
    \end{indt}

    \vspace{12pt}
    
    \begin{indt}{\section{Topologie d'un espace vectoriel normé}}

        Dans cette section, on désigne par $\lrb{E, \norm \cdot}$ un $\K$-espace vectoriel.

        \begin{indt}{\subsection{Voisinages}}
            \begin{indt}{\subsubsection{Définition (\textit{voisinage})}}
                Soit $a \in E$.

                Un \emph{voisinage} de $a$ est une partie $A \in \mathcal P(E)$ vérifiant
                \[
                    \exists r \in \R^*_+\ |\ \mathscr B_o(a, r) \subset A
                \]

                On note $v(a)$ l'ensemble des voisinages de $a$ :
                \[
                    \forall a \in E,\
                    v(a) = \set{A \in \mathcal P(E)\ |\ \exists r \in \R^*_+\ |\ \mathscr B_o(a, r) \subset A}
                \]
            \end{indt}

            \vspace{12pt}
            
            \begin{indt}{\subsubsection{Propriétés}}
                On a :

                $\bullet$ $
                    \forall a \in E,\
                    \forall V \in v(a), a \in V
                $

                $\bullet$ $
                    \forall (a, W) \in E \times \mathcal P(E),\
                    (\exists V \in v(a)\ |\ V \subset W) \Rightarrow W \in v(a)
                $

                \vspace{6pt}
                
                Soit $I \neq \varnothing$ un ensemble d'indexation, et $a \in E$.
                \[
                    \forall \lr{V_i}_{i \in I} \subset v(a),\
                    \bigcup_{i \in I} V_i \in v(a)
                \]

                Soit $n \in \N$ et $a \in E$.
                \[
                    \forall \!\lr{V_k}_{k \in \nset 1 n} \subset v(a),\
                    \bigcap_{k = 1}^n V_k \in v(a)
                \]

                $\bullet$ $
                    \forall a, b \in E\ |\ a \neq b,\
                    \exists V, W \in v(a)\ |\ V \cap W = \varnothing
                $

                $\bullet$ Deux normes équivalentes définissent la même notion de voisinage.
            \end{indt}

            \vspace{12pt}
            
            \begin{indt}{\subsubsection{Extension à l'infini}}
                On appelle \emph{voisinage de $+\infty$} (respectivement de $-\infty$) toute partie $V \in \mathcal P(\R)$ vérifiant
                \[
                    \exists A \in \R\ |\ \ ]A\ ;\ +\infty[\ \subset V
                    \quad (\text{resp.}\ \ ] -\!\infty\ ;\ A[\ \subset V)
                \]

                On note
                \[
                    \begin{array}{rcl}
                        v(+\infty)
                        &=&
                        \set{V \in \mathcal P(\R)\ |\ \exists A \in \R\ |\ \ ]A\ ;\ +\infty[\ \subset V}
                        \vspace{6pt}
                        \\
                        v(-\infty)
                        &=&
                        \set{V \in \mathcal P(\R)\ |\ \exists A \in \R\ |\ \ ] -\!\infty\ ;\ A[\ \subset V}
                    \end{array}
                \]
            \end{indt}
        \end{indt}

        \vspace{12pt}
        
        \begin{indt}{\subsection{Ouverts, fermés}}
            \begin{indt}{\subsubsection{Définition (\textit{Ouvert})}}
                Un ouvert de $[E, \norm \cdot]$ est une partie de $E$ vérifiant :
                \[
                    \begin{array}{rcl}
                        \forall O \in \mathcal P(E),\
                        O\ \text{ouvert}
                        &\ssi&
                        \forall a \in O,\ O \in v(a)
                        \vspace{3pt}
                        \\
                        &\ssi&
                        \forall a \in O,\ \exists r \in \R^*_+\ |\ \mathscr B_o(a, r) \subset O
                    \end{array}
                \]
            \end{indt}

            \vspace{12pt}
            
            \begin{indt}{\subsubsection{Définition (\textit{Fermé})}}
                Un fermé de $[E, \norm \cdot]$ est une partie vérifiant :
                \[
                    \forall F \in \mathcal P(E),\
                    F\ \text{fermé}
                    \ssi
                    \complement_E(F) = E \setminus F\ \text{ouvert}
                \]
            \end{indt}

            \vspace{12pt}
            
            \begin{indt}{\subsubsection{Propriétés}}
                Soient $n \in \N^*$, et $I \neq \varnothing$ un ensemble d'indexation.

                \vspace{12pt}
                
                $\bullet$
                $
                    \displaystyle
                    \forall (O_i)_{i \in I}\ \text{ouverts},\
                    \bigcup_{i \in I} O_i\ \text{ouvert}
                $

                $\bullet$
                $
                    \displaystyle
                    \forall (O_k)_{k \in \nset 1 n}\ \text{ouverts},\
                    \bigcap_{k = 1}^n O_k\ \text{ouvert}
                $

                \vspace{6pt}
                
                $\bullet$
                $
                    \displaystyle
                    \forall (F_i)_{i \in I}\ \text{fermés},\
                    \bigcap_{i \in I} F_i\ \text{fermé}
                $

                $\bullet$
                $
                    \displaystyle
                    \forall (F_k)_{k \in \nset 1 n}\ \text{fermés},\
                    \bigcup_{k = 1}^n F_k\ \text{fermé}
                $

                \vspace{12pt}
                
                $\bullet$ Soient $(a, r) \in E \times \R^*_+$. On a :
                \[
                    \mathscr B_o(a, r) = \set{x \in E\ |\ \norm{x - a} < r}\ \text{ouvert}
                \]
                \[
                    \mathscr B_f(a, r) = \set{x \in E\ |\ \norm{x - a} \le r}\ \text{fermé}
                \]
                \[
                    \mathscr S(a, r) = \mathscr B_f(a, r) \setminus \mathscr B_o(a, r)\ \text{fermé}
                \]

                \vspace{12pt}
                
                Soient $\displaystyle \lr{[E_k, N_k]}_{k \in \nset 1 n}$ des espaces vectoriels normés.
                On note $E = \displaystyle \prod_{k = 1}^n E_k$.
                On munit $E$ de la norme produit $N$.

                Alors
                \[
                    \forall (O_k)_{k \in \nset 1 n}\ |\ \forall k \in \nset 1 n,\ O_k\ \text{ouvert de}\ E_k,\
                    \prod_{k = 1}^n O_k \ \text{ouvert de}\ [E, N]
                \]

                \[
                    \forall (F_k)_{k \in \nset 1 n}\ |\ \forall k \in \nset 1 n,\ F_k\ \text{fermé de}\ E_k,\
                    \prod_{k = 1}^n F_k \ \text{fermé de}\ [E, N]
                \]
            \end{indt}
        \end{indt}

        \vspace{12pt}
        
        \begin{indt}{\subsection{Adhérence}}
            \begin{indt}{\subsubsection{Définition (\textit{Adhérence})}}
                Soit $A \in \mathcal P(E)$, et $a \in E$.

                Alors $a$ est dit \emph{adhérent} à $A$ lorsque
                \[
                    \forall V \in v(a),\ V \cap A \neq \varnothing
                \]

                On note alors $\cj A$ l'ensemble des points adhérents à $A$, appelée \emph{adhérence de $A$} :
                \[
                    \cj A = \set{a \in E\ |\ \forall V \in v(a),\ V \cap A \neq \varnothing}
                \]
            \end{indt}

            \vspace{12pt}
            
            \begin{indt}{\subsubsection{Définition (\textit{densité})}}
                Soit $A \in \mathcal P(E)$.

                On dit que $A$ est \emph{dense} dans $E$ lorsque
                \[
                    \cj A = E
                \]
            \end{indt}

            \vspace{12pt}
            
            \begin{indt}{\subsubsection{Caractérisation de l'adhérence (intersection)}}
                Soit $A \in \mathcal P(E)$.

                Alors :
                \[
                    \cj A
                    =
                    \bigcap_{\substack{F \text{ fermé de } E \vspace{2pt} \\ A \subset F}} F
                \]

                Donc $\cj A$ est le plus petit fermé de $E$ contenant $A$.
            \end{indt}

            \vspace{12pt}
            
            \begin{indt}{\subsubsection{Propriétés}}
                Soient $A, B \in \mathcal P(E)$. On a :

                \vspace{6pt}
                
                $\bullet$ $A = \cj A \ssi A\ \text{fermé}$

                \vspace{3pt}

                $\bullet$ $\cj{\cj{A}} = \cj A$ (car $\cj A$ fermé)

                \vspace{3pt}

                $\bullet$ $A \subset B \Rightarrow \cj A \subset \cj B$

                \vspace{3pt}

                $\bullet$ $\cj{A \cup B} = \cj A \cup \cj B$

                \vspace{3pt}

                $\bullet$ $\cj{A \cap B} \subset \cj A \cap \cj B$
            \end{indt}

            \vspace{12pt}
            
            \begin{indt}{\subsubsection{Caractérisation séquentielle de l'adhérence}}
                Soit $A \in \mathcal P(E) \setminus \varnothing$, et $a \in E$.

                Alors :
                \[
                    a \in \cj A
                    \ssi
                    \exists (x_n)_{n \in \N} \in A^\N\ |\ x_n \tendsto{n \to +\infty} a
                \]
            \end{indt}

            \vspace{12pt}
            
            \begin{indt}{\subsubsection{Corollaire}}
                Soit $A \in \mathcal P(E) \setminus \varnothing$.

                On a :

                \vspace{6pt}
                
                $\bullet$ $\forall x \in E, x \in \cj A \ssi d(x, A) = 0$

                $\bullet$
                $
                    A\ \text{fermé dans}\ E
                    \ssi
                    \forall (x_n)_{n \in \N} \in A^\N,\
                    (\exists \ell \in E\ |\ x_n \tendsto{n \to +\infty} \ell) \Rightarrow \ell \in A
                $

                $\bullet$
                $
                    A\ \text{dense dans}\ E
                    \ssi
                    \forall x \in E,\
                    \exists (u_n)_{n \in \N} \in A^\N\ |\ u_n \tendsto{n \to +\infty} x
                $
            \end{indt}
        \end{indt}

        \vspace{12pt}
        
        \begin{indt}{\subsection{Intérieur}}
            \begin{indt}{\subsubsection{Définition (\textit{Intérieur})}}
                Soit $A \in \mathcal P(E)$, et $a \in E$.

                Alors $a$ est dit \emph{intérieur} à $A$ lorsque
                \[
                    \begin{array}{cl}
                        & \exists O\ \text{ouvert de}\ E\
                        \begin{array}{|l}
                            O \subset A
                            \\
                            a \in O
                        \end{array}
                        \vspace{3pt}
                        \\
                        \ssi&
                        \exists r \in \R^*_+\ |\ \mathscr B_o(a, r) \subset A
                        \vspace{3pt}
                        \\
                        \ssi&
                        A \in v(a)
                    \end{array}
                \]

                On note $\mathring A$ l'ensemble des points intérieurs à $A$, appelé \emph{intérieur} de $A$ :
                \[
                    \mathring A = \set{x \in E\ |\ A \in v(x)}
                \]
            \end{indt}

            \vspace{12pt}
            
            \begin{indt}{\subsubsection{Propriétés}}
                Soient $A, B \in \mathcal P(E)$.

                On a :

                $\bullet$ $\mathring A \subset A$

                $\bullet$ $A \subset B \Rightarrow \mathring A \subset \mathring B$
            \end{indt}

            \vspace{12pt}
            
            \begin{indt}{\subsubsection{Caractérisation (union)}}
                Soit $A \in \mathcal P(E)$.

                Alors
                \[
                    \mathring A =
                    \bigcup_{\substack{O \text{ ouvert de } E \vspace{2pt} \\ O \subset A}} O
                \]

                Donc $\mathring A$ est le plus grand ouvert de $E$ inclus $A$.
            \end{indt}

            \vspace{12pt}
            
            \begin{indt}{\subsubsection{Propriétés}}
                Soient $A, B \in \mathcal P(E)$.

                On a :

                \vspace{6pt}
                
                $\bullet$ $\widering{A \cap B} = \mathring A \cap \mathring B$

                $\bullet$ $\mathring A \cup \mathring B \subset \widering{A \cup B}$

                $\bullet$ $\complement_E(\cj A) = \widering{\complement_E(A)}$

                $\bullet$ $\complement_E\!\lr{\mathring A} = \cj{\complement_E(A)}$

                \vspace{6pt}
                
                $\bullet$ $A\ \text{ouvert} \ssi A = \mathring A$
            \end{indt}
        \end{indt}

        \vspace{12pt}
        
        \begin{indt}{\subsection{Frontière}}
            \begin{indt}{\subsubsection{Définition (\textit{Frontière})}}
                Soit $A \in \mathcal P(E)$.

                On appelle \emph{frontière} de $A$, et on le note souvent $Fr(A)$, l'ensemble
                \[
                    Fr(A) = \cj A \setminus \mathring A
                    = \complement_{\cj A}\!\lr{\mathring A}
                    = \cj A \cap \complement_E\!\lr{\mathring A}
                \]
            \end{indt}

            \vspace{12pt}
            
            \begin{indt}{\subsubsection{Définition (\textit{Extérieur})}}
                Soit $A \in \mathcal P(E)$.

                On appelle \emph{extérieur} de $A$ l'ouvert
                \[
                    \complement_E\!\lr{\bar A} = \widering{\complement_E(A)}
                \]
            \end{indt}

            \vspace{12pt}
            
            \begin{indt}{\subsubsection{Proposition}}
                Soit $A \in \mathcal P(E)$.

                Alors l'intérieur, l'extérieur et la frontière de $A$ forment une partition de $E$.
            \end{indt}
        \end{indt}

        \vspace{12pt}
        
        \begin{indt}{\subsection{Topologie relative à une partie}}
            \begin{indt}{\subsubsection{Définition (\textit{voisinage relatif})}}
                Soit $A \in \mathcal P(E) \setminus \varnothing$, $a \in A$, et $W \in \mathcal P(A)$.

                Alors $W$ est un \emph{voisinage relatif à $A$} de $a$ si
                \[
                    \exists W' \in v(a)\ |\ W = W' \cap A
                \]

                On note
                \[
                    v_A(a) = \set{W \in \mathcal P(A)\ |\ \exists W' \in v(a)\ |\ W = W' \cap A}
                \]
                l'ensemble des voisinages relatifs à $A$ de $a$.
            \end{indt}

            \vspace{12pt}
            
            \begin{indt}{\subsubsection{Définition (\textit{ouvert, fermé relatif})}}
                Soit $A \in \mathcal P(E)$, et $W \in \mathcal P(A)$.

                \vspace{6pt}
                
                $\bullet$ Alors $W$ est un \emph{ouvert relatif à $A$} de $E$ si
                \[
                    \forall a \in A,\ W \in v_A(a)
                \]

                \vspace{6pt}
                
                $\bullet$ $W$ est un \emph{fermé relatif à $A$} de $E$ si c'est le complémentaire d'un ouvert relatif à $A$.
            \end{indt}

            \vspace{12pt}
            
            \begin{indt}{\subsubsection{Caractérisation}}
                Soit $A \in \mathcal P(E)$.

                \vspace{6pt}
                
                $\bullet$ $\forall O \in \mathcal P(A)$, on a :
                \[
                    O\ \text{ouvert relatif à}\ A
                    \ssi
                    \exists O'\ \text{ouvert de}\ E\ |\ O = O' \cap A
                \]

                $\bullet$ $\forall W \in \mathcal P(A)$, on a :
                \[
                    W\ \text{fermé relatif à}\ A
                    \ssi
                    \exists W'\ \text{fermé de}\ E\ |\ W = W' \cap A
                \]
            \end{indt}

            \vspace{12pt}
            
            \begin{indt}{\subsubsection{Caractérisation séquentielle}}
                Soit $A \in \mathcal P(E)$.

                $\forall F \in \mathcal (A)$, on a :
                \[
                    F\ \text{fermé relatif à}\ A
                    \ssi
                    \forall (x_n)_{n \in \N} \in F^\N,\
                    (\exists \ell \in A\ |\ x_n \tendsto{n \to +\infty} \ell) \Rightarrow \ell \in F
                \]
            \end{indt}
        \end{indt}
    \end{indt}

    \begin{indt}{\section{Limites, continuité d'une application}}
        Dans toute cette section, on pose deux $\K$-espaces vectoriel normés $[E, \norm \cdot _E]$ et $[F, \norm \cdot _F]$.

        \begin{indt}{\subsection{Limites}}
            \begin{indt}{\subsubsection{Définition (\textit{Limite en un point})}}
                Soit $A \in \mathcal P(E) \setminus \varnothing$, $f \in F^A$, et $a \in \cj A$.

                On dit que $f$ \emph{admet une limite $b \in F$ en $a$} si et seulement si :
                \[
                    \begin{array}{cl}
                        &
                        \forall \varepsilon \in \R^*_+,\
                        \exists \delta_\varepsilon \in \R^*_+\ |\
                        \forall x \in A,\
                        \norm{x - a}_E \le \delta_\varepsilon \Rightarrow \norm{f(x) - b}_F \le \varepsilon
                        \vspace{6pt}
                        \\
                        \ssi&
                        \forall \varepsilon \in \R^*_+,\
                        \exists \delta_\varepsilon\ |\
                        f\!\lr{\mathscr B_f(a, \delta_\varepsilon) \cap A} \subset \mathscr B_f(b, \varepsilon)
                        \vspace{6pt}
                        \\
                        \overset{\text{si } a \in A}\ssi&
                        \forall V \in v_F(b),\ \exists W \in v_A(a)\ |\ f(W) \subset V
                        \vspace{6pt}
                        \\
                        \overset{\text{si } a \in A}\ssi&
                        \forall V \in v_F(b),\ f^{-1}(V) \in v_A(a)
                    \end{array}
                \]

                \vspace{12pt}
                
                Remarque : deux normes équivalentes définissent la même notion de voisinage et d'adhérence, donc la même notion de convergence : changer de norme pour une autre qui lui est équivalente ne change pas l'existence ni la valeur d'une limite.
            \end{indt}

            \vspace{12pt}
            
            \begin{indt}{\subsubsection{Caractérisation séquentielle de la limite}}
                Soient $A \in \mathcal P(E) \setminus \varnothing$, $f \in F^A$, et $a \in \cj A$.

                Alors $f$ admet une limite en $a$ si et seulement si
                \[
                    \forall (x_n)_{n \in \N} \in A^\N\ |\ x_n \tendsto{n \to +\infty} a,\
                    \exists \ell \in F\ |\ f(x_n) \tendsto{n \to +\infty} \ell
                \]

                Et dans ce cas, la limite de $f$ est $\ell$.

                \vspace{12pt}
                
                Remarque : peut être intéressant sans la valeur de la limite.
            \end{indt}

            \vspace{12pt}
            
            \begin{indt}{\subsubsection{Propriété}}
                Soient $n \in \N^*$, $(G_k)_{k \in \nset 1 n}$ des espaces vectoriels normés, $\displaystyle G = \prod_{k = 1}^n G_k$ que l'on munit de la norme produit, et $A \in \mathcal P(E) \setminus \varnothing$.

                Soient $\forall k \in \nset 1 n,\ f_k \in {G_k}^A$, et :
                \[
                    \begin{array}{ccccc}
                        f & : & A & \longrightarrow & G
                        \\
                          && x & \longmapsto & \lr{f_1(x), \ldots, f_n(x)}
                    \end{array}
                \]

                Alors pour $\ell = (\ell_1, \ldots, \ell_n) \in G$ et $a \in \cj A$, on a :
                \[
                    \lim_a f = \ell
                    \ssi
                    \forall k \in \nset 1 n,\
                    \lim_a f_k = \ell_k
                \]
            \end{indt}
        \end{indt}

        \vspace{12pt}
        
        \begin{indt}{\subsection{Continuité d'une application}}
            \begin{indt}{\subsubsection{Définition}}
                Soit $A \in \mathcal P(E) \setminus \varnothing$, et $f \in F^A$.

                \vspace{6pt}
                
                $\bullet$ La fonction $f$ est \emph{continue en $a \in A$} si et seulement si
                \[
                    \exists \ell \in F\ |\ f(x) \tendsto{x \to a} \ell
                \]
                et dans ce cas, $\ell = f(a)$, \textit{i.e} si et seulement si
                \[
                    \forall \varepsilon \in \R^*_+,\
                    \exists \delta_\varepsilon\ |\
                    \forall x \in A,\
                    \norm{x - a}_E \le \delta_\varepsilon \Rightarrow \norm{f(x) - f(a)}_F \le \varepsilon
                \]

                \vspace{6pt}
                
                $\bullet$ La fonction $f$ est \emph{continue} (sur $A$) si et seulement si elle est continue en tout point de $A$.

                Dans ce cas, on note
                \[
                    f \in \mathcal C^0(A, F)
                \]
            \end{indt}

            \vspace{12pt}
            
            \begin{indt}{\subsubsection{Propriétés}}
                $\bullet$ Soient $n \in \N^*$, $(G_k)_{k \in \nset 1 n}$ des espaces vectoriels normés, $\displaystyle G = \prod_{k = 1}^n G_k$ que l'on munit de la norme produit, et $A \in \mathcal P(E) \setminus \varnothing$.

                Soient $\forall k \in \nset 1 n,\ f_k \in {G_k}^A$, et :
                \[
                    \begin{array}{ccccc}
                        f & : & A & \longrightarrow & G
                        \\
                          && x & \longmapsto & \lr{f_1(x), \ldots, f_n(x)}
                    \end{array}
                \]

                Alors $f$ est continue en $a \in A$ si et seulement si
                \[
                    \forall k \in \nset 1 n,\
                    f_k \in \mathcal C^0(A, G_k)
                \]

                \vspace{12pt}
                
                $\bullet$ Soit $A \in \mathcal P(E) \setminus \varnothing$, $a \in A$, et $f \in F^A$.

                Alors si $f$ est continue en $a$, alors elle est bornée au voisinage de $a$.

                \vspace{12pt}
                
                $\bullet$ Soit $A \in \mathcal P(E) \setminus \varnothing$. Alors $\mathcal C^0(A, F)$ est un $\K$-sous espace vectoriel de $F^A$.
            \end{indt}

            \vspace{12pt}
            
            \begin{indt}{\subsubsection{Caractérisation séquentielle de la continuité}}
                Soit $A \in \mathcal P(E)$, $a \in A$, et $f \in F^A$.

                Alors $f$ est continue en $a$ si et seulement si
                \[
                    \forall \lr{x_n}_{n \in \N}\ |\ x_n \tendsto{n \to +\infty} a,\
                    f(x_n) \tendsto{n \to +\infty} f(a)
                \]
            \end{indt}

            \vspace{12pt}
            
            \begin{indt}{\subsubsection{Propriétés (exemples d'applications continues)}}
                $\bullet$ Une application lipschitzienne est continue.

                \vspace{12pt}
                
                $\bullet$ On suppose $E$ de dimension finie $n \in \N$, et on note $\mathcal B = \lr{e_k}_{k \in \nset 1 n}$ une de ses $\K$-bases.

                On pose $I \in \mathcal P(\N^n) \setminus \varnothing$ finie, et $\lr{\lambda_i}_{i \in I} \subset \K$.

                \vspace{6pt}
                
                Alors l'application polynômiale (en les coordonnées de $x$ dans $\mathcal B$)
                \[
                    \begin{array}{ccccl}
                        f & : & E & \longrightarrow & \K
                        \\
                          && x & \longmapsto &
                          \displaystyle
                          \sum_{\substack{i \in I \\ i = (i_1, \ldots, i_n)}} \lambda_i \prod_{k = 1}^n \lr{e_k^*(x)}^{i_k}
                    \end{array}
                \]
                est continue.
            \end{indt}

            \vspace{12pt}
            
            \begin{indt}{\subsubsection{Lien avec les ouverts}}
                Soit $A \in \mathcal P(E) \setminus \varnothing$, et $f \in F^A$.

                Alors on a :
                \[
                    \begin{array}{rcl}
                        f \in \mathcal C^0(A, F)
                        &\ssi&
                        \forall O\ \text{ouvert de}\ F,\
                        f^{-1}(O)\ \text{ouvert relatif à}\ A
                        \\
                        &\ssi&
                        \forall F\ \text{fermé de}\ F,\
                        f^{-1}(F)\ \text{fermé relatif à}\ A
                    \end{array}
                \]
            \end{indt}

            \vspace{12pt}
            
            \begin{indt}{\subsubsection{Propriété}}
                Soit $A \in \mathcal P(E) \setminus \varnothing$, $B \in \mathcal P(A)\ |\ \cj B = A$, et $f, g \in \mathcal C^0(A, F)$

                Alors
                \[
                    f_{|B} = g_{|B}\
                    \Rightarrow\
                    f = g
                \]
            \end{indt}
        \end{indt}

        \vspace{12pt}
        
        \begin{indt}{\subsection{Cas des applications linéaires}}
            \begin{indt}{\subsubsection{Théorème (CNS de continuité pour des applications linéaires)}}
                Soit $f \in \mathcal L_\K(E, F)$.

                On a :
                \[
                    \begin{array}{rcl}
                        f \in \mathcal C^0(E, F)
                        &\ssi&
                        f\ \text{continue en}\ 0_E
                        \vspace{6pt}
                        \\
                        &\ssi&
                        \exists \alpha \in \R\ |\ \forall x \in E,\ \norm{f(x)}_F \le \alpha \norm x _E
                        \vspace{6pt}
                        \\
                        &\ssi&
                        f\ \text{lipschitzienne}
                        \vspace{6pt}
                        \\
                        &\ssi&
                        \exists M \in \R\ |\ \forall x \in E\ |\
                        \norm x _E \le 1 \Rightarrow \norm{f(x)}_F \le M
                        %\vspace{6pt}
                        %\\
                        %&\ssi&
                        %\exists M \in \R\ |\ \forall x \in \mathscr B_f(0, 1),\ f(x) \in \mahtscr B_f(0, M)
                    \end{array}
                \]

                \vspace{12pt}
                
                On note
                \[
                    \mathcal L_c(E, F) = \mathcal C^0(E, F) \cap \mathcal L(E, F)
                \]

                C'est un $\K$-sous espace vectoriel de $\mathcal L_\K(E, F)$
            \end{indt}

            \vspace{12pt}
            
            \begin{indt}{\subsubsection{Définition (\textit{norme d'opérateurs})}}
                On peut munir le $\K$-espace vectoriel $\mathcal L_c(E, F)$ d'une norme dite \emph{subordonnée} (à $\norm \cdot _E$ et $\norm \cdot _F$), ou \emph{norme d'opérateurs}, par :
                \[
                    \begin{array}{rcl}
                        \forall u \in \mathcal L_c(E, F),\
                        \nnorm u
                        =
                        \norm u _{\rm op}
                        &=&
                        \sup\!\set{
                            \norm{u(x)}_F\
                            \begin{array}{|l}
                                x \in E
                                \\
                                \norm x _E = 1
                            \end{array}
                            \vphantom{\dfrac a {\tfrac a a}}
                        }
                        \vspace{6pt}
                        \\
                        &=&
                        \sup\!\set{
                            \norm{u(x)}_F\
                            \begin{array}{|l}
                                x \in E
                                \\
                                \norm x _E \le 1
                            \end{array}
                            \vphantom{\dfrac a {\tfrac a a}}
                        }
                        \vspace{6pt}
                        \\
                        &=&
                        \sup\!\set{
                            \dfrac{\norm{u(x)}_F}{\norm x _E}\
                            \left| \vphantom{\dfrac a a}\right.
                            x \in E \setminus \set 0
                            \vphantom{\dfrac a {\tfrac a a}}
                        }
                    \end{array}
                \]

                \vspace{6pt}
                
                Remarque : une norme subordonnée dépend des normes choisies sur $E$ et $F$.
            \end{indt}

            \vspace{12pt}
            
            \begin{indt}{\subsubsection{Propriété (\textit{sous-multiplicabilité})}}
                Soient $[E, \norm \cdot _E],\ [F, \norm \cdot _F],\ [G, \norm \cdot _G]$ des $\K$-espaces vectoriels normés.

                Soient
                $
                    \begin{array}{|l}
                        u \in \mathcal L_c(E, F)
                        \\
                        v \in \mathcal L_c(F, G)
                    \end{array}
                $.

                \vspace{6pt}
                
                Alors
                \[
                    v \circ u \in \mathcal L_c(E, G)
                \]
                et
                \[
                    \nnorm{v \circ u} \le \nnorm{v} \cdot \nnorm u
                \]
            \end{indt}
            
            \vspace{12pt}
            
            \begin{indt}{\subsubsection{Continuité d'une application multilinéaire}}
                Soient $n \in \N^*$, $\lr{\lrb{A_k, \norm \cdot _{A_k}}}_{k \in \nset 1 n}$ des espaces vectoriels normés, $\displaystyle A = \prod_{k = 1}^n A_k$ munit de la norme produit, et $f \in F^A$ une application $n$-linéaire.

                Alors
                \[
                    f \in \mathcal C^0(A, F)
                    \ssi
                    \exists \alpha \in \R_+\ |\
                    \forall (x_1, \ldots, x_n) \in A,\
                    \norm{f(x_1, \ldots, x_n)}_F \le \alpha \prod_{k = 1}^n \norm{x_k}_{A_k}
                \]
            \end{indt}
        \end{indt}
    \end{indt}

    \vspace{12pt}
    
    \begin{indt}{\section{Compacité}}
        On note, dans cette section, $[E, \norm \cdot _E]$ un $\K$-espace vectoriel normé.

        \begin{indt}{\subsection{Définition (\textit{compact})}}
            Soit $A \in \mathcal P(E)$.

            Alors la partie $A$ est dite \emph{compacte} lorsque toute suite de points de $A$ admet au moins une valeur d'adhérence dans $A$, \textit{i.e} lorque
            \[
                \forall \lr{x_n}_{n \in \N} \in A^\N,\
                \exists \varphi \in \N^\N\ \text{strictement croissante},\
                \exists \ell \in A\ |\
                x_{\varphi(n)} \tendsto{n \to +\infty} \ell
            \]

            \vspace{6pt}
            
            Remarque : deux normes équivalentes définissent la même notion de convergence et de limite, donc \textit{a fortiori} la même notion de compacité.
        \end{indt}
    \end{indt}
    
\end{document}
%--------------------------------------------End
