\documentclass[a4paper, 12pt, twoside]{article}


%------------------------------------------------------------------------
%
% Author                :   Lasercata
% Last modification     :   2023.01.12
%
%------------------------------------------------------------------------


%------ini
\usepackage[utf8]{inputenc}
\usepackage[T1]{fontenc}
\usepackage[french]{babel}
%\usepackage[english]{babel}


%------geometry
\usepackage[textheight=700pt, textwidth=500pt]{geometry}


%------color
\usepackage{xcolor}
\definecolor{ff4500}{HTML}{ff4500}
\definecolor{00f}{HTML}{0000ff}
\definecolor{0ff}{HTML}{00ffff}
\definecolor{656565}{HTML}{656565}

%\renewcommand{\emph}{\textcolor{ff4500}}
%\renewcommand{\em}{\color{ff4500}}

\newcommand{\Emph}{\textcolor{ff4500}}

\newcommand{\strong}[1]{\textcolor{ff4500}{\bf #1}}
\newcommand{\st}{\color{ff4500}\bf}


%------Code highlighting
%---listings
\usepackage{listings}

\definecolor{cbg}{HTML}{272822}
\definecolor{cfg}{HTML}{ececec}
\definecolor{ccomment}{HTML}{686c58}
\definecolor{ckw}{HTML}{f92672}
\definecolor{cstring}{HTML}{e6db72}
\definecolor{cstringlight}{HTML}{98980f}
\definecolor{lightwhite}{HTML}{fafafa}

\lstdefinestyle{DarkCodeStyle}{
    backgroundcolor=\color{cbg},
    commentstyle=\itshape\color{ccomment},
    keywordstyle=\color{ckw},
    numberstyle=\tiny\color{cbg},
    stringstyle=\color{cstring},
    basicstyle=\ttfamily\footnotesize\color{cfg},
    breakatwhitespace=false,
    breaklines=true,
    captionpos=b,
    keepspaces=true,
    numbers=left,
    numbersep=5pt,
    showspaces=false,
    showstringspaces=false,
    showtabs=false,
    tabsize=4,
    xleftmargin=\leftskip
}

\lstdefinestyle{LightCodeStyle}{
    backgroundcolor=\color{lightwhite},
    commentstyle=\itshape\color{ccomment},
    keywordstyle=\color{ckw},
    numberstyle=\tiny\color{cbg},
    stringstyle=\color{cstringlight},
    basicstyle=\ttfamily\footnotesize\color{cbg},
    breakatwhitespace=false,
    breaklines=true,
    captionpos=b,
    keepspaces=true,
    numbers=left,
    numbersep=10pt,
    showspaces=false,
    showstringspaces=false,
    showtabs=false,
    tabsize=4,
    frame=L,
    xleftmargin=\leftskip
}

%\lstset{style=DarkCodeStyle}
\lstset{style=LightCodeStyle}
%Usage : \begin{lstlisting}[language=Caml, xleftmargin=xpt] ... \end{lstlisting}


%---Algorithm
\usepackage[linesnumbered,ruled,vlined]{algorithm2e}
\SetKwInput{KwInput}{Input}
\SetKwInput{KwOutput}{Output}

\SetKwProg{Fn}{Function}{:}{}
\SetKw{KwPrint}{Print}

\newcommand\commfont[1]{\textit{\texttt{\textcolor{656565}{#1}}}}
\SetCommentSty{commfont}
\SetProgSty{texttt}
\SetArgSty{textnormal}
\SetFuncArgSty{textnormal}
%\SetProgArgSty{texttt}

\newenvironment{indalgo}[2][H]{
    \begin{algoBox}
        \begin{algorithm}[#1]
            \caption{#2}
}
{
        \end{algorithm}
    \end{algoBox}
}


%---tcolorbox
\usepackage[many]{tcolorbox}
\DeclareTColorBox{emphBox}{O{black}O{lightwhite}}{
    breakable,
    outer arc=0pt,
    arc=0pt,
    top=0pt,
    toprule=-.5pt,
    right=0pt,
    rightrule=-.5pt,
    bottom=0pt,
    bottomrule=-.5pt,
    colframe=#1,
    colback=#2,
    enlarge left by=10pt,
    width=\linewidth-\leftskip-10pt,
}

\DeclareTColorBox{algoBox}{O{black}O{lightwhite}}{
    breakable,
    arc=0pt,
    top=0pt,
    toprule=-.5pt,
    right=0pt,
    rightrule=-.5pt,
    bottom=0pt,
    bottomrule=-.5pt,
    left=0pt,
    leftrule=-.5pt,
    colframe=#1,
    colback=#2,
    width=\linewidth-\leftskip-10pt,
}


%-------make the table of content clickable
\usepackage{hyperref}
\hypersetup{
    colorlinks,
    citecolor=black,
    filecolor=black,
    linkcolor=black,
    urlcolor=black
}


%------pictures
\usepackage{graphicx}
%\usepackage{wrapfig}

\usepackage{tikz}
%\usetikzlibrary{babel}             %Uncomment this to use circuitikz
%\usetikzlibrary{shapes.geometric}  % To draw triangles in trees
%\usepackage{circuitikz}            %Electrical circuits drawing


%------tabular
%\usepackage{color}
%\usepackage{colortbl}
%\usepackage{multirow}


%------Physics
%---Packages
%\usepackage[version=4]{mhchem} %$\ce{NO4^2-}$

%---Commands
\newcommand{\link}[2]{\mathrm{#1} \! - \! \mathrm{#2}}
\newcommand{\pt}[1]{\cdot 10^{#1}} % Power of ten
\newcommand{\dt}[2][t]{\dfrac{\mathrm d #2}{\mathrm d #1}} % Derivative


%------math
%---Packages
%\usepackage{textcomp}
%\usepackage{amsmath}
\usepackage{amssymb}
\usepackage{mathtools} % For abs
\usepackage{stmaryrd} %for \llbracket and \rrbracket
\usepackage{mathrsfs} %for \mathscr{x} (different from \mathcal{x})

%---Commands
%-Sets
\newcommand{\N}{\mathbb{N}} %set N
\newcommand{\Z}{\mathbb{Z}} %set Z
\newcommand{\Q}{\mathbb{Q}} %set Q
\newcommand{\R}{\mathbb{R}} %set R
\newcommand{\C}{\mathbb{C}} %set C
\newcommand{\U}{\mathbb{U}} %set U
\newcommand{\seg}[2]{\left[ #1\ ;\ #2 \right]}
\newcommand{\nset}[2]{\left\llbracket #1\ ;\ #2 \right\rrbracket}

%-Exponantial / complexs
\newcommand{\e}{\mathrm{e}}
\newcommand{\cj}[1]{\overline{#1}} %overline for the conjugate.

%-Vectors
\newcommand{\vect}{\overrightarrow}
\newcommand{\veco}[3]{\displaystyle \vect{#1}\binom{#2}{#3}} %vector + coord

%-Limits
\newcommand{\lm}[2][{}]{\lim\limits_{\substack{#2 \\ #1}}} %$\lm{x \to a} f$ or $\lm[x < a]{x \to a} f$
\newcommand{\Lm}[3][{}]{\lm[#1]{#2} \left( #3 \right)} %$\Lm{x \to a}{f}$ or $\Lm[x < a]{x \to a}{f}$
\newcommand{\tendsto}[1]{\xrightarrow[#1]{}}

%-Integral
\newcommand{\dint}[4][x]{\displaystyle \int_{#2}^{#3} #4 \mathrm{d} #1} %$\dint{a}{b}{f(x)}$ or $\dint[t]{a}{b}{f(t)}$

%-left right
\newcommand{\lr}[1]{\left( #1 \right)}
\newcommand{\lrb}[1]{\left[ #1 \right]}
\newcommand{\lrbb}[1]{\left\llbracket #1 \right\rrbracket}
\newcommand{\set}[1]{\left\{ #1 \right\}}
\newcommand{\abs}[1]{\left\lvert #1 \right\rvert}
\newcommand{\ceil}[1]{\left\lceil #1 \right\rceil}
\newcommand{\floor}[1]{\left\lfloor #1 \right\rfloor}
\newcommand{\lrangle}[1]{\left\langle #1 \right\rangle}

%-Others
\newcommand{\para}{\ /\!/\ } %//
\newcommand{\ssi}{\ \Leftrightarrow \ }
\newcommand{\eqsys}[2]{\begin{cases} #1 \\ #2 \end{cases}}

\newcommand{\med}[2]{\mathrm{med} \left[ #1\ ;\ #2 \right]}  %$\med{A}{B} -> med[A ; B]$
\newcommand{\Circ}[2]{\mathscr{C}_{#1, #2}}

\renewcommand{\le}{\leqslant}
\renewcommand{\ge}{\geqslant}

\newcommand{\oboxed}[1]{\textcolor{ff4500}{\boxed{\textcolor{black}{#1}}}} %orange boxed

\newcommand{\rboxed}[1]{\begin{array}{|c} \hline #1 \\ \hline \end{array}} %boxed with right opened
\newcommand{\lboxed}[1]{\begin{array}{c|} \hline #1 \\ \hline \end{array}} %boxed with left opened

\newcommand{\orboxed}[1]{\textcolor{ff4500}{\rboxed{\textcolor{black}{#1}}}} %orange right boxed
\newcommand{\olboxed}[1]{\textcolor{ff4500}{\lboxed{\textcolor{black}{#1}}}} %orange left boxed


%------commands
%---to quote
\newcommand{\simplecit}[1]{\guillemotleft$\;$#1$\;$\guillemotright}
\newcommand{\cit}[1]{\simplecit{\textcolor{656565}{#1}}}
\newcommand{\quo}[1]{\cit{\it #1}}

%---to indent
\newcommand{\ind}[1][20pt]{\advance\leftskip + #1}
\newcommand{\deind}[1][20pt]{\advance\leftskip - #1}

%---to indent a text
\newcommand{\indented}[2][20pt]{\par \ind[#1] #2 \par \deind[#1]}
\newenvironment{indt}[2][20pt]{#2 \par \ind[#1]}{\par \deind} %Titled indented env

%---title
\newcommand{\thetitle}[2]{\begin{center}\textbf{{\LARGE \underline{\Emph{#1} :}} {\Large #2}}\end{center}}

%---Maths environments
%-Proofs
\newenvironment{proof}[1][{}]{\begin{indt}{$\square$ #1}}{$\blacksquare$ \end{indt}}

%-Maths parts (proposition, definition, ...)
\newenvironment{mathpart}[1]{\begin{indt}{\boxed{\text{\textbf{#1}}}}}{\end{indt}}
\newenvironment{mathbox}[1]{\boxed{\text{\textbf{#1}}}\begin{emphBox}}{\end{emphBox}}
\newenvironment{mathul}[1]{\begin{indt}{\underline{\textbf{#1}}}}{\end{indt}}

\newenvironment{theo}{\begin{mathpart}{Théorème}}{\end{mathpart}}
\newenvironment{Theo}{\begin{mathbox}{Théorème}}{\end{mathbox}}

\newenvironment{prop}{\begin{mathpart}{Proposition}}{\end{mathpart}}
\newenvironment{Prop}{\begin{mathbox}{Proposition}}{\end{mathbox}}
\newenvironment{props}{\begin{mathpart}{Propriétés}}{\end{mathpart}}

\newenvironment{defi}{\begin{mathpart}{Définition}}{\end{mathpart}}
\newenvironment{meth}{\begin{mathpart}{Méthode}}{\end{mathpart}}

\newenvironment{Rq}{\begin{mathul}{Remarque :}}{\end{mathul}}
\newenvironment{Rqs}{\begin{mathul}{Remarques :}}{\end{mathul}}

\newenvironment{Ex}{\begin{mathul}{Exemple :}}{\end{mathul}}
\newenvironment{Exs}{\begin{mathul}{Exemples :}}{\end{mathul}}


%------Sections
% To change section numbering :
% \renewcommand\thesection{\Roman{section}}
% \renewcommand\thesubsection{\arabic{subsection})}
% \renewcommand\thesubsubsection{\textit \alph{subsubsection})}

% To start numbering from 0
% \setcounter{section}{-1}


%------page style
\usepackage{fancyhdr}
\usepackage{lastpage}

\setlength{\headheight}{18pt}
\setlength{\footskip}{50pt}

\pagestyle{fancy}
\fancyhf{}
\fancyhead[LE, RO]{\textit{\textcolor{black}{\today}}}
\fancyhead[RE, LO]{\large{\textsl{\Emph{\texttt{\jobname}}}}}

\fancyfoot[RO, LE]{\textit{\texttt{\textcolor{black}{Page \thepage /}\pageref{LastPage}}}}
\fancyfoot[LO, RE]{\includegraphics[scale=0.12]{/home/lasercata/Pictures/1.images_profil/logo/mieux/lasercata_logo_fly_fond_blanc.png}}


%------init lengths
\setlength{\parindent}{0pt} %To avoid using \noindent everywhere.
\setlength{\parskip}{3pt}

\usepackage{bbm}
\newcommand{\1}{\mathbbm 1}


%---------------------------------Begin Document
\begin{document}
    
    \thetitle{Chapitre 17}{Complément d'algorithmique}
    
    \tableofcontents
    \newpage
    
    \begin{indt}{\section{Optimisation}}
        \begin{indt}{\subsection{Optimisation exacte}}
            \begin{indt}{\subsubsection{Introduction}}
                On s'intéresse ici à la résolution de problèmes d'optimisation au sens de la définition du chapitre 16, en 1.2.2 : on cherche un algorithme calculant une solution optimale \textbf{pour toute instance}.
            \end{indt}

            \vspace{12pt}
            
            \begin{indt}{\subsubsection{Exemple : le problème du sac à dos}}
                \label{1.1.2}

                Le problème est le suivant : on dispose d'objets de poids respectifs $w_0, \ldots, w_{n - 1}$ et de valeurs respectives $p_0, \ldots, p_{n - 1}$ et d'un sac à dos capable de supporter un poids $W$. On souhaite sélectionner des objets de sorte à maximiser la valeur totale sans dépasser la capacité du sac à dos.

                Dans la version en variables réelles, on suppose que l'on peut prendre des fractions des objets. Le problème d'optimisation se formule ainsi :

                Maximiser $\displaystyle \sum_{i = 0}^{n - 1} x_i p_i$ sous les contraintes :
                \[
                    \begin{cases}
                        \displaystyle
                        \sum_{i = 0}^{n - 1} x_i w_i \le W
                        \\
                        \forall i \in \nset 0 {n - 1},\ x_i \in \seg 0 1
                    \end{cases}
                \]

                Ce problème est résolu par un algorithme glouton :

                \begin{indalgo}{Solution du problème du sac à dos, version en variables réelles}
                    \label{alg:1}
                    Trier les objets par $\dfrac{p_i}{w_i}$ décroissant\;

                    \While{que possible en considérant les objets dans cet ordre}{
                        Fixer $x_i$ à 1\;
                    }

                    Lorsque cela n'est plus possible, prendre la fraction de l'objet courant permettant de remplir le sac\;
                \end{indalgo}

                Cet algorithme calcule bien une solution optimale : si on note $i$ l'objet de $\dfrac{p_i}{w_i}$ maximal non encore sélectionné et si une solution optimale coïncidant avec l'algorithme sur les objets avant $i$, et ne sélectionne pas cet objet dans son intégralité, alors $\exists j$ tel que la solution optimale sélectionne une fraction de l'objet $j$ qui est $> x_j$.

                Dans ce cas, il existe $\delta_j > 0$ tel que l'on peut ajouter une quantité $\dfrac{\delta_j}{w_i}$ de l'objet $i$ et retirer une quantité $\dfrac{\delta_j}{w_j}$ de l'objet $j$ à la solution optimale.

                La variation de poids est $\dfrac{\delta_j}{w_i}w_i - \dfrac{\delta_j}{w_j}w_j = 0$, donc on a toujours une solution.

                La variation de valeur est
                \[
                    \dfrac{\delta_j}{w_i} p_i - \dfrac{\delta_j}{w_j} p_j
                    = \underbrace{\delta_j}_{> 0}\underbrace{\lr{\dfrac{p_i}{w_i} - \dfrac{p_j}{w_j}}}_{\ge 0}
                \]

                Donc la solution reste optimale.

                On peut donc modifier la solution optimale jusqu'à l'obtention d'une solution optimale ayant choisi l'objet $i$ dans son intégralité.

                L'invariant \simplecit{il existe une solution optimale ayant fait les mêmes choix que l'algorithme glouton} est vrai.

                \vspace{12pt}
                
                Remarque : le problème du sac à dos, dans sa version entière (les $x_i \in \set{0, 1}$) ne peut pas être résolu par l'algorithme glouton (algorithme n°\ref{alg:1}) auquel on retire la dernière étape ne prenant qu'une fraction du dernier objet.

                \begin{center}
                    \begin{tabular}{l|ccc}
                        Poids & 5 & 5 & 7
                        \\
                        \hline
                        Valeur & 5 & 5 & 8
                    \end{tabular}
                    $\qquad W = 10$
                \end{center}

                \textit{Cf} l'exemple ci-dessus : l'algorithme glouton donne une solution de valeur $8$ en prenant l'objet de poids 7 alors qu'une solution optimale est de valeur 10 : on prend les deux objets de poids 5.
            \end{indt}

            \vspace{12pt}
            
            \begin{indt}{\subsubsection{Séparation et évaluation (\textit{branch and bound})}}
                $\bullet$ Une technique de résolution des problèmes d'optimisation consiste à effectuer une exploration exhaustive de l'ensemble des solutions et à conserver la meilleure solution.

                Cependant, on se heurte à des problèmes de complexité (exemple : pour le sac à dos en variables entières, il y a $2^n$ solutions potentielles à tester).

                On peut parfois accélérer la recherche grâce à l'heuristique du retour sur trace (\textit{cf} chapitre 8, 4.3).

                Par exemple, pour le problème du sac à dos, il y a surement de nombreuses combinaisons d'objets qui dépassent la capacité du sac à dos. On peut donc sélectionner les objets un à un et lorsque l'on s'aperçoit que la capacité du sac à dos est dépassée, on revient sur le dernier choix.

                En pratique, cela revient à construire un arbre binaire dans lequel tous les n\oe uds de même profondeur correspondent à un même objet et pour ces n\oe uds, le fils gauche correspond au cas où l'on a sélectionné l'objet et le fils droit au cas où l'objet n'est pas sélectionné.

                On élague les branches correspondant à des sélections dépassant la capacité du sac à dos.

                Si $w_1 + w_2 > W$,

                \begin{center}
                    \begin{tikzpicture}
                        \node (0) at (0, 0) {$\bullet$}
                            child [xshift=-90pt] {node {$\bullet$}
                                child [xshift=-30pt] {node {$\bullet$}
                                    child {node {} edge from parent [dashed]}
                                    edge from parent [above left] node {$x_2 = 1$}
                                }
                                child [xshift=30pt] {node {$\bullet$}
                                    child {node {$\bullet$}
                                        child {node {} edge from parent [dashed]}
                                        edge from parent [above left] node {$x_3 = 1$}
                                    }
                                    child {node {$\bullet$}
                                        child {node {} edge from parent [dashed]}
                                        edge from parent [above right] node {$x_3 = 0$}
                                    }
                                    edge from parent [above right] node {$x_2 = 0$}
                                }
                                edge from parent [above left] node {$x_1 = 1$}
                            }
                            child [xshift=90pt] {node {$\bullet$}
                                child [xshift=-30pt] {node {$\bullet$}
                                    child {node {$\bullet$}
                                        child {node {} edge from parent [dashed]}
                                        edge from parent [above left] node {$x_3 = 1$}
                                    }
                                    child {node {$\bullet$}
                                        child {node {} edge from parent [dashed]}
                                        edge from parent [above right] node {$x_3 = 0$}
                                    }
                                    edge from parent [above left] node {$x_2 = 1$}
                                }
                                child [xshift=30pt] {node {$\bullet$}
                                    child {node {$\bullet$}
                                        child {node {} edge from parent [dashed]}
                                        edge from parent [above left] node {$x_3 = 1$}
                                    }
                                    child {node {$\bullet$}
                                        child {node {} edge from parent [dashed]}
                                        edge from parent [above right] node {$x_3 = 0$}
                                    }
                                    edge from parent [above right] node {$x_2 = 0$}
                                }
                                edge from parent [above right] node {$x_1 = 0$}
                            }
                        ;

                        \node (1) at (-6, -6) {
                            \begin{tabular}{c}
                                Sous-arbre non parcouru
                                \\
                                car dépasse la capacité
                            \end{tabular}
                        };
                        \draw[->] (1) to (-5.7, -4.5);
                    \end{tikzpicture}
                \end{center}

                Dans le cadre de la résolution d'un problème d'optimisation, on peut parfois élaguer encore plus l'arbre de recherche en considérant le coût des solutions construites : si on sait évaluer une borne du meilleur possible pour une série de choix sans parcourir l'intégralité du sous-arbre correspondant, on peut parfois élaguer ce sous-arbre si on connaît déjà une solution de coût meilleur que cette borne.

                \begin{indt}{Cette méthode consiste à concevoir un algorithme par séparation et évaluation :}
                    $-$ La séparation consiste à diviser le problème en sous-problèmes, donc à créer un branchement dans l'arbre de reverche.

                    Exemple : pour le problème du sac à dos, on a deux sous-problèmes, selon que l'objet $i$ est sélectionné ou non.

                    \vspace{6pt}
                    
                    $-$ L'évaluation consiste à déterminer une borne sur le coût d'une solution optimale \textbf{réalisable avec les choix déjà faits} et à le comparer avec une borne connue pour savoir s'il est nécessaire de poursuivre l'exploration du sous-arbre.
                \end{indt}

                \vspace{12pt}
                
                \begin{indt}{Pour que cette méthode soit efficace, on a besoin de bonnes heuristiques pour :}
                    $-$ La séparation : si les choix initiaux convergent rapidement vers une \simplecit{bonne solution}, on élaguera plus de branches dans la suite de l'exploration.

                    \vspace{6pt}
                    
                    $-$ L'évaluation : on doit pouvoir calculer \emph{efficacement} une borne \emph{la plus juste possible} pour avoir de bonnes chances d'élaguer des branches.
                \end{indt}
            \end{indt}

            \vspace{12pt}
            
            \begin{indt}{\subsubsection{Exemple : le problème du sac à dos}}
                $\bullet$ Pour la séparation, on sélectionne ou pas un objet, avec l'heuristique suivante : on considère les objets par $\dfrac{p_i}{w_i}$ décroissant.

                \vspace{6pt}
                
                $\bullet$ Pour l'évaluation, on utilise l'heuristique de relaxation : on relâche certaines contraintes, ce qui élargit le domaine des solutions donc permet potentiellement d'atteindre un meilleur coût.
                Si le problème relâché est plus simple à résoudre, le coût d'une solution optimale est donc la borne recherchée.

                Ici, on effectue une relaxation continue : on n'impose plus aux $x_i$ d'être des entiers, ce qui nous ramène au problème vu en \ref{1.1.2} (page \pageref{1.1.2}), que l'on sait résoudre efficacement (en $\mathcal O(n)$ car les objets seront triés une seule fois au début de l'algorithme pour l'heuristique de séparation).

                Remarque : si l'algorithme nous donne une solution entière : on a trouvé la solution optimale (selon les choix qui sont déjà fait).

                Si l'algorithme nous donne une solution avec un terme dans $]0, 1[$, la partie entière de la valeur de cette solution est une borne supérieure sur la solution optimale du problème entier.

                \vspace{12pt}
                
                Exemple : on considère l'instance suivante :

                \begin{center}
                    \begin{tabular}{r|cccc}
                        $i$ & 1 & 2 & 3 & 4
                        \\
                        \hline
                        $p_i$ & 1 & 1 & 5 & 3
                        \\
                        \hline
                        $w_i$ & 3 & 2 & 4 & 2
                    \end{tabular}
                    $\qquad W = 5$
                \end{center}

                Le tri des objets indique qu'on doit les traiter dans l'ordre suivant : 4, 3, 2, 1.

                On note au cours de l'exécution, $W$ la capacité courante du sac à dos, et $V$ la valeur totale courante des objets sélectionnés.

                \begin{center}
                    \begin{tikzpicture}[scale=1.6]
                        \node (0) at (0, 0) [rectangle, draw] {$W = 5, V = 0, \sup = \floor{3 + \dfrac 3 4 5} = 6$}
                            child [xshift=-60pt] {node [rectangle, draw] {$W = 3, V = 3, \sup = 6$}
                                child [xshift=-20pt] {node [rectangle, draw] {$W = -1$}
                                edge from parent [above left] node {$x_3 = 1$}}
                                child [xshift=20pt] {node [rectangle, draw] {$
                                        \begin{array}{c}
                                            W = 3, V = 3
                                            \\
                                            \sup = 3 + \floor{1 + \frac 1 3} = 4
                                        \end{array}
                                $}
                                    child [xshift=-20pt] {node [rectangle, draw] {$
                                            \begin{array}{c}
                                                W = 1, V = 4
                                                \\
                                                \sup = 4 + \floor{\frac 1 3} = 4 = V
                                            \end{array}
                                    $} edge from parent [left] node {$x_2 = 1$}}
                                    child [xshift=20pt] {node [rectangle, draw] {$
                                            \begin{array}{c}
                                                W = 3, V = 3
                                                \\
                                                \sup = 4 = \mathrm{inf}
                                            \end{array}
                                    $} edge from parent [right] node {$x_2 = 0$}}
                                    edge from parent [above right] node {$x_3 = 0$}
                                }
                                edge from parent [above left] node {$x_4 = 1$}
                            }
                            child [xshift=60pt] {node [rectangle, draw] {$
                                    \begin{array}{c}
                                        W = 5, V = 0
                                        \\
                                        \sup = \floor{5 + \frac 1 2} = 5
                                    \end{array}
                            $}
                                child [xshift=-20pt] {node [rectangle, draw] {$
                                        \begin{array}{c}
                                            W = 1, V = 5
                                            \\
                                            \sup = 5 = V > \mathrm{inf}
                                        \end{array}
                                $} edge from parent [left] node {$x_3 = 1$}}
                                edge from parent [above right] node {$x_4 = 0$}
                            }
                        ;
                        
                        \node (1) at (-4.4, -3.5) {\huge{$\times$}};
                        \node (1) at (.3, -5.2) {\huge{$\times$}};
                    \end{tikzpicture}
                \end{center}

                Solution $(0, 1, 0, 1)$, et $\inf = 4$

                Solution $(0, 0, 1, 0)$, et $\inf = 5$

                \vspace{12pt}
                
                Conclusion : sélectionner uniquement l'objet $3$ est une solution optimale.

                \vspace{6pt}
                
                \boxed{\rm Exo} écrire l'exécution de l'algorithme si l'heuristique de séparation consiste à prendre les objets par ordre d'indice ou par ordre de poids décroissant ou de valeur décroissante.

                Tester aussi l'heuristique d'évaluation qui consiste à prendre la borne des valeurs des objets comme borne supérieure.
            \end{indt}
        \end{indt}

        \vspace{12pt}
        
        \begin{indt}{\subsection{Optimisation et \textbf{NP}-complétude}}
            \begin{indt}{\subsubsection{Introduction}}
                On considère un problème d'optimisation caractérisé par une relation $\mathcal R \subseteq A \times B$ et une fonction de coût $c : B \longrightarrow \R^+$.
                On rappelle que le problème de décision associé à ce problème d'optimisation s'énonce ainsi : étant donné une instance $a \in A$ et un entier $k \in \N$, existe-t-il une solution $b \in B$ telle que
                \[
                    \begin{cases}
                        a \mathcal R b
                        \\
                        c(b) \le k
                    \end{cases}
                \]

                Remarque : on peut aussi considérer des problèmes de maximisation et la condition à satisfaire est alors $c(k) \ge k$.

                \vspace{6pt}
                
                Fait : s'il existe un algorithme de complexité polynomiale qui résout le problème d'optimisation, alors le problème de décision associé appartient à la classe $\mathrm{P}$.
                En effet, il suffit pour une instance $a$ d'exécuter l'algorithme qui résout le problème d'optimisation sur $a$ (complexité polynomiale en la taille de $a$) puis de comparer le coût de la solution optimale obtenue et $k$ (complexité $\mathcal O(\log k)$).
                Cela donne un algorithme polynomial qui résout le problème de décision.

                \vspace{12pt}
                
                Conséquence : si le problème de décision associé à un problème d'optimisation est \textbf{NP}-complet, alors on a peu d'espoir de trouver un algorithme polynomial qui résout le problème d'optimisation.
            \end{indt}

            \vspace{12pt}
            
            \begin{indt}{\subsubsection{Retour au problème du sac à dos}}
                On sait que le problème en variables réelles peut être résolu en temps $\mathcal O(n\log n)$ par un algorithme glouton et que cet algorithme ne fonctionne pas pour le problème en variables entières.

                \vspace{12pt}
                
                $\bullet$ Résolution par programmation dynamique : on considère une instance
                \[
                    (w_1, \ldots, w_n, p_1, \ldots, p_n, W)
                \]
                du problème du sac à dos en variables entières.

                $\forall i \in \nset 0 n,\ \forall w \in \nset 0 W$, on note $V(i, w)$ la valeur maximale que l'on peut atteindre en sélectionnant des objets parmi ceux d'indices 1 à $i$ dans un sac à dos de capacité $w$.

                On cherche à obtenir $V(n, W)$.

                $
                    \forall i \in \nset 0 {n - 1},\
                    \forall w \in \nset 0 W,\
                $
                \[
                    V(i + 1, w) =
                    \begin{cases}
                        \vspace{-32pt}
                        \\
                        V(i, w)
                        & \text{si}\ \overbrace{w_{i + 1} > w}^{\substack{\text{on ne peut pas} \\ \text{sélectionner l'objet $i + 1$}}}
                        \vspace{6pt}
                        \\
                        \max(\!\!\!\!\!\! \underbrace{V(i, w)}_{\substack{\text{on ne sélectionne} \\ \text{pas l'objet $i + 1$}}}\!\!\!\!\!\!\!,\ \, \underbrace{p_{i + 1} + V(i, w - w_{i + 1})}_{\substack{\text{on sélectionne} \\ \text{l'objet $i + 1$}}})
                        & \text{sinon}
                        \vspace{-21pt}
                    \end{cases}
                \]

                \vspace{12pt}

                $\forall w \in \nset 0 W,\ V(0, W) = 0$

                On peut donc remplir la matrice $V$ ligne par ligne (par $i$ croissant) et retrouver une solution réalisant $V(n, W)$ en temps $\mathcal O(W_n)$.

                Conclusion ? Aucune car c'est un algorithme de complexité exponentielle en la taille de l'instance ($W$ est de taille $\mathcal O(\log w)$).

                \vspace{12pt}
                
                $\bullet$ Proposition

                \begin{emphBox}
                    On considère le problème de décision suivant :

                    \textsc{Sac\_a\_dos} : étant donné $n$ poids $w_1, \ldots, w_n \in \N$, $n$ valeurs $x_1, \ldots, x_n \in \set{0, 1}$ telles que
                    \[
                        \begin{cases}
                            \displaystyle
                            \sum_{i = 1}^n x_i p_i \ge k
                            \\
                            \displaystyle
                            \sum_{i = 1}^n x_i w_i \le W
                        \end{cases}
                        \qquad
                        \text{?}
                    \]

                    \vspace{6pt}
                    
                    \textsc{Sac\_a\_dos} est \textbf{NP}-complet.
                \end{emphBox}

                \vspace{6pt}
                
                \begin{proof}
                    $-$ \textsc{Sac\_a\_dos} $\in \mathbf{NP}$ : $x_1, \ldots, x_n$ est un certificat vérifiable en temps polynomial.

                    \vspace{6pt}
                    
                    $-$ \textsc{Sac\_a\_dos} est \textbf{NP}-difficile : on procède par réduction de $2$-\textsc{Partition} (\textit{cf} TD$_{46}$) : étant donné $S = \set{a_1, \ldots, a_n} \subseteq \N$, existe-t-il $I \subseteq \nset 1 n$ tel que
                    \[
                        \sum_{i \in I} a_i = \sum_{i \in \nset 1 n \setminus I} a_i
                        \qquad \text{?}
                    \]

                    Soit $S = \set{a_1, \ldots, a_n}$ une instance de 2-\textsc{Partition}.

                    On construit l'instance suivante de \textsc{Sac\_a\_dos} :
                    \[
                        \begin{cases}
                            \forall i \in \nset 1 n,\ w_i = p_i = a_i
                            \\
                            \displaystyle
                            W = k = \dfrac 1 2 \sum_{i = 1}^n a_i
                        \end{cases}
                    \]

                    \begin{indt}{Cette instance est calculable en temps polynomial en la taille de $S$ et cela constitue une réduction :}
                        $+$ Si $\exists I \subseteq \nset 1 n\ |\ \displaystyle \sum_{i \in I} a_i = \sum_{i \in \nset 1 n \setminus I} a_i$, alors
                        \[
                            \sum_{i = 1}^n a_i
                            = \sum_{i \in I} a_i + \sum_{i \in \nset 1 n \setminus I} a_i
                            = 2\sum_{i \in I} a_i
                        \]

                        donc en notant $\forall i \in \nset 1 n,\ x_i = \1_I(i)$, on obtient
                        \[
                            \sum_{i = 1}^n x_i p_i
                            = \sum_{i = 1}^n x_i w_i
                            = \sum_{i = 1}^n x_i a_i
                            = \sum_{i = 1}^n \1_I(i) a_i
                            = \sum_{i \in I} a_i
                            = \dfrac 1 2 \sum_{i = 1}^n a_i
                            = k
                            = W
                        \]

                        Donc les $x_i$ sont une solution de l'instance de \textsc{Sac\_a\_dos} associée.

                        \vspace{6pt}
                        
                        $+$ Réciproquement, si
                        \[
                            \exists (x_1, \ldots, x_n) \in \set{0, 1}^n\ |\
                            \begin{cases}
                                \displaystyle
                                \sum_{i = 1}^n x_i w_i \le W
                                \\
                                \displaystyle
                                \sum_{i = 1}^n x_i p_i \ge k
                            \end{cases}
                        \]

                        Alors
                        \[
                            \dfrac 1 2 \sum_{i = 1}^n a_i
                            = k
                            \le \sum_{i = 1}^n x_i p_i
                            = \sum_{i = 1}^n x_i a_i
                            = \sum_{i = 1}^n x_i w_i
                            \le W
                            = \dfrac 1 2 \sum_{i = 1}^n a_i
                        \]

                        Donc
                        \[
                            \begin{array}{rcl}
                                \displaystyle
                                \sum_{i = 1}^n x_i a_i
                                &=& \displaystyle
                                \dfrac 1 2 \sum_{i = 1}^n a_i
                                \vspace{3pt}
                                \\
                                &=& \displaystyle
                                \dfrac 1 2 \lr{\sum_{\substack{i = 1 \\ x_i = 1}}^n a_i + \sum_{\substack{i = 1 \\ x_i = 0}}^n a_i}
                                \vspace{3pt}
                                \\
                                &=& \displaystyle
                                \dfrac 1 2 \lr{\sum_{\substack{i = 1 \\ x_i = 1}}^n x_i a_i + \sum_{\substack{i = 1 \\ x_i = 0}}^n a_i}
                                \vspace{3pt}
                                \\
                                &=& \displaystyle
                                \dfrac 1 2 \lr{\sum_{\substack{i = 1}}^n x_i a_i + \sum_{\substack{i = 1 \\ x_i = 0}}^n a_i}
                            \end{array}
                        \]

                        Donc $\displaystyle \sum_{\substack{i = 1 \\ x_i = 1}}^n a_i = \sum_{i = 1}^n x_i a_i = \sum_{\substack{i = 0 \\ x_i = 0}}^n a_i$
                    \end{indt}

                    Donc $I = \set{i \in \nset 1 n\ |\ x_i = 1}$ est une solution à l'instance de 2-\textsc{Partition}
                \end{proof}
            \end{indt}

            \vspace{12pt}
            
            \begin{indt}{\subsubsection{Remarque}}
                Comme on a peu d'espoir de trouver efficacement une solution optimale à un problème d'optimisation dont le problème de décision associé est $\mathbf{NP}$-complet, on va plutôt chercher efficacement une solution \simplecit{pas trop mauvaise}.
                On cherche donc à concevoir un algorithme de complexité polynomiale qui fournit des solutions pour lesquelles on peut estimer la \simplecit{distance} à l'optimum.

                \vspace{12pt}
                
                $\bullet$ Exemple : l'algorithme glouton vu en \ref{1.1.2} (page \pageref{1.1.2}) pour le problème du sac à dos est très mauvais vis à vis du problème en variables entières : si $k \in \N^*$ et $W \in \N \setminus \set{0, 1}$, on considère deux objets tels que
                \[
                    \begin{cases}
                        p_1 = 1
                        \\
                        w_1 = \dfrac{W - 1}{k}
                    \end{cases}
                    \quad
                    \text{et}
                    \quad
                    \begin{cases}
                        p_2 = k
                        \\
                        w_2 = W
                    \end{cases}
                \]

                L'algorithme glouton sélectionne l'objet $1$ car $\dfrac{1}{\dfrac{W - 1}{k}}  = \dfrac{k}{W - 1} > \dfrac k W$, ce qui donne une solution de valeur 1 alors que la solution optimale consiste à prendre l'objet 2 pour une valeur $k$.

                Ainsi $\forall k \in \N^*$, il existe une instance telle que la solution optimale est $k$ fois meilleure que celle calculée par l'algorithme glouton.

                \vspace{12pt}
                
                $\bullet$ On peut faire mieux en modifiant légèrement l'algorithme : on garde la meilleure solution entre celle de l'algorithme glouton et celle qui consiste à ne prendre que l'objet de valeur maximale.

                \vspace{6pt}
                
                Proposition :
                \begin{emphBox}
                    On note pour toute instance $e$, $V^*(e)$ la valeur d'une solution optimale, et $V(e)$ la valeur de la solution calculée par l'algorithme glouton modifié.

                    Alors
                    \[
                        \forall e,\ V^*(e) \le 2V(e)
                    \]
                \end{emphBox}

                \vspace{6pt}
                
                \begin{proof}
                    On note $e = (w_1, \ldots, w_n, p_1, \ldots, p_n, W)$.

                    Quitte à renuméroter, on peut supposer que les objets sont tirés par $\dfrac{p_i}{w_i}$ décroissant.

                    On sait que toute solution entière donne une valeur inférieure à celle d'une solution optimale réelle.
                    De plus, une telle solution est calculée \textit{via} l'algorithme glouton.

                    Alors, en notant $j \in \nset 1 n$ l'indice du premier objet que l'on ne peut pas placer intégralement dans le sac à dos, on a
                    \[
                        \begin{array}{rcl}
                            V^*(e)
                            &\le& \displaystyle
                            \sum_{i = 1}^{j - 1} p_i
                            + \underbrace{\dfrac{\displaystyle W - \sum_{i = 1}^{j - 1} w_i}{w_j}}_{< 1} p_j
                            \\
                            &\le& \displaystyle
                            \sum_{i = 1}^{j - 1} p_i + p_i
                            \vspace{3pt}
                            \\
                            &\le& \displaystyle
                            \sum_{i = 1}^{j - 1} p_i + \max_{i \in \nset 1 n} p_i
                            \vspace{3pt}
                            \\
                            &\le& \displaystyle
                            2 \max\!\lr{\sum_{i = 1}^{j - 1} a_i,\ \max_{i \in \nset 1 n} p_i}
                        \end{array}
                    \]
                    (L'algorithme glouton prend tous les objets de 1 à $j - 1$ puis la fraction de l'objet $j$ qui permet de remplir le sac à dos)
                \end{proof}

                \vspace{12pt}
                
                Remarque : on a donc un algorithme de même complexité que l'algorithme glouton et tel que la solution calculée est toujours de valeur supérieure à la moitié de la valeur optimale.

                \vspace{12pt}
                
                \boxed{\rm H.P} $\forall \varepsilon \in \R^*_+$, il existe un algorithme de complexité $\mathcal O\!\lr{\dfrac{1 + \varepsilon}{\varepsilon} n^3}$ déterminant une solution entière au problème du sac à dos et tel que $\forall e$ instance, la valeur $V(e)$ de la solution calculée vérifie $V^*(e) \le (1 + \varepsilon)V(e)$.

                Idée : par programmation dynamique : $\forall i \in \nset 0 n,\ \forall p \in \nset{0}{\displaystyle \sum_{i = 1}^n p_i}$, on calcule le poids minimal $W(i, p)$ réalisable en sélectionnant des objets parmi ceux d'indices de 1 à $i$ de sorte que la valeur obtenue vaille $p$ (et le poids $\le W$).

                On le fait avec des objets de valeur modifiée en fonction de $n$, $\varepsilon$, $p_{\max} = \max_{i \in \nset 1 n} p_i$

                $\displaystyle \forall i \in \nset 1 n$, on note $\floor{\dfrac{p_i}{2^t}}$, où
                \[
                    t = \floor{\log\!\lr{\dfrac{\varepsilon}{1 + \varepsilon} \dfrac{p_{\max}}{n}}}
                \]
            \end{indt}
        \end{indt}
    \end{indt}
    
\end{document}
%--------------------------------------------End
