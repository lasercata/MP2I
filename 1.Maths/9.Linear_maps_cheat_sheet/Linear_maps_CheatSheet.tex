\documentclass[a4paper, 12pt, twoside]{article}


%------------------------------------------------------------------------
%
% Author                :   Lasercata
% Last modification     :   2022.08.28
%
%------------------------------------------------------------------------


%------ini
\usepackage[utf8]{inputenc}
\usepackage[T1]{fontenc}
%\usepackage[french]{babel}
%\usepackage[english]{babel}


%------geometry
\usepackage[textheight=700pt, textwidth=500pt]{geometry}


%------color
\usepackage{xcolor}
\definecolor{ff4500}{HTML}{ff4500}
\definecolor{00f}{HTML}{0000ff}
\definecolor{0ff}{HTML}{00ffff}
\definecolor{656565}{HTML}{656565}

\renewcommand{\emph}{\textcolor{ff4500}}
\renewcommand{\em}{\color{ff4500}}

\newcommand{\strong}[1]{\textcolor{ff4500}{\bf #1}}
\newcommand{\st}{\color{ff4500}\bf}


%------Code highlighting
%---listings
\usepackage{listings}

\definecolor{cbg}{HTML}{272822}
\definecolor{cfg}{HTML}{ececec}
\definecolor{ccomment}{HTML}{686c58}
\definecolor{ckw}{HTML}{f92672}
\definecolor{cstring}{HTML}{e6db72}
\definecolor{cstringlight}{HTML}{98980f}
\definecolor{lightwhite}{HTML}{fafafa}

\lstdefinestyle{DarkCodeStyle}{
    backgroundcolor=\color{cbg},
    commentstyle=\itshape\color{ccomment},
    keywordstyle=\color{ckw},
    numberstyle=\tiny\color{cbg},
    stringstyle=\color{cstring},
    basicstyle=\ttfamily\footnotesize\color{cfg},
    breakatwhitespace=false,
    breaklines=true,
    captionpos=b,
    keepspaces=true,
    numbers=left,
    numbersep=5pt,
    showspaces=false,
    showstringspaces=false,
    showtabs=false,
    tabsize=4,
    xleftmargin=\leftskip
}

\lstdefinestyle{LightCodeStyle}{
    backgroundcolor=\color{lightwhite},
    commentstyle=\itshape\color{ccomment},
    keywordstyle=\color{ckw},
    numberstyle=\tiny\color{cbg},
    stringstyle=\color{cstringlight},
    basicstyle=\ttfamily\footnotesize\color{cbg},
    breakatwhitespace=false,
    breaklines=true,
    captionpos=b,
    keepspaces=true,
    numbers=left,
    numbersep=10pt,
    showspaces=false,
    showstringspaces=false,
    showtabs=false,
    tabsize=4,
    frame=L,
    xleftmargin=\leftskip
}

%\lstset{style=DarkCodeStyle}
\lstset{style=LightCodeStyle}
%Usage : \begin{lstlisting}[language=Caml] ... \end{lstlisting}

%---tcolorbox
\usepackage[many]{tcolorbox}
\DeclareTColorBox{pseudocode}{O{black}O{lightwhite}}{
    breakable,
    outer arc=0pt,
    arc=0pt,
    top=0pt,
    toprule=-.5pt,
    right=0pt,
    rightrule=-.5pt,
    bottom=0pt,
    bottomrule=-.5pt,
    colframe=#1,
    colback=#2,
    enlarge left by=10pt,
    width=\linewidth-\leftskip-10pt,
}


%-------make the table of content clickable
\usepackage{hyperref}
\hypersetup{
    colorlinks,
    citecolor=black,
    filecolor=black,
    linkcolor=black,
    urlcolor=black
}
%Uncomment this and comment above for dark mode
% \hypersetup{
%     colorlinks,
%     citecolor=white,
%     filecolor=white,
%     linkcolor=white,
%     urlcolor=white
% }


%------pictures
\usepackage{graphicx}
%\usepackage{wrapfig}

\usepackage{tikz}
%\usetikzlibrary{babel}             %Uncomment this to use circuitikz
%\usetikzlibrary{shapes.geometric}  % To draw triangles in trees
%\usepackage{circuitikz}            %Electrical circuits drawing


%------tabular
%\usepackage{color}
%\usepackage{colortbl}
%\usepackage{multirow}


%------Physics
%---Packages
%\usepackage[version=4]{mhchem} %$\ce{NO4^2-}$

%---Commands
\newcommand{\link}[2]{\mathrm{#1} \! - \! \mathrm{#2}}
\newcommand{\pt}[1]{\cdot 10^{#1}} % Power of ten
\newcommand{\dt}[2][t]{\dfrac{\mathrm d #2}{\mathrm d #1}} % Derivative


%------math
%---Packages
%\usepackage{textcomp}
%\usepackage{amsmath}
\usepackage{amssymb}
\usepackage{mathtools} % For abs
\usepackage{stmaryrd} %for \llbracket and \rrbracket
\usepackage{mathrsfs} %for \mathscr{x} (different from \mathcal{x})

%---Commands
%-Sets
\newcommand{\N}{\mathbb{N}} %set N
\newcommand{\Z}{\mathbb{Z}} %set Z
\newcommand{\Q}{\mathbb{Q}} %set Q
\newcommand{\R}{\mathbb{R}} %set R
\newcommand{\C}{\mathbb{C}} %set C
\newcommand{\U}{\mathbb{U}} %set U
\newcommand{\seg}[2]{\left[ #1\ ;\ #2 \right]}
\newcommand{\nset}[2]{\left\llbracket #1\ ;\ #2 \right\rrbracket}

%-Exponantial / complexs
\newcommand{\e}{\mathrm{e}}
\newcommand{\cj}[1]{\overline{#1}} %overline for the conjugate.

%-Vectors
\newcommand{\vect}{\overrightarrow}
\newcommand{\veco}[3]{\displaystyle \vect{#1}\binom{#2}{#3}} %vector + coord

%-Limits
\newcommand{\lm}[2][{}]{\lim\limits_{\substack{#2 \\ #1}}} %$\lm{x \to a} f$ or $\lm[x < a]{x \to a} f$
\newcommand{\Lm}[3][{}]{\lm[#1]{#2} \left( #3 \right)} %$\Lm{x \to a}{f}$ or $\Lm[x < a]{x \to a}{f}$
\newcommand{\tendsto}[1]{\xrightarrow[#1]{}}

%-Integral
\newcommand{\dint}[4][x]{\displaystyle \int_{#2}^{#3} #4 \mathrm{d} #1} %$\dint{a}{b}{f(x)}$ or $\dint[t]{a}{b}{f(t)}$

%-left right
\newcommand{\lr}[1]{\left( #1 \right)}
\newcommand{\lrb}[1]{\left[ #1 \right]}
\newcommand{\lrbb}[1]{\left\llbracket #1 \right\rrbracket}
\newcommand{\set}[1]{\left\{ #1 \right\}}
\newcommand{\abs}[1]{\left\lvert #1 \right\rvert}
\newcommand{\ceil}[1]{\left\lceil #1 \right\rceil}
\newcommand{\floor}[1]{\left\lfloor #1 \right\rfloor}
\newcommand{\lrangle}[1]{\left\langle #1 \right\rangle}

%-Others
\newcommand{\para}{\ /\!/\ } %//
\newcommand{\ssi}{\ \Leftrightarrow \ }
\newcommand{\eqsys}[2]{\begin{cases} #1 \\ #2 \end{cases}}

\newcommand{\med}[2]{\mathrm{med} \left[ #1\ ;\ #2 \right]}  %$\med{A}{B} -> med[A ; B]$
\newcommand{\Circ}[2]{\mathscr{C}_{#1, #2}}

\renewcommand{\le}{\leqslant}
\renewcommand{\ge}{\geqslant}

\newcommand{\oboxed}[1]{\textcolor{ff4500}{\boxed{\textcolor{black}{#1}}}} %orange boxed


%------commands
%---to quote french text
\newcommand{\simplecit}[1]{\guillemotleft$\;$#1$\;$\guillemotright}
\newcommand{\cit}[1]{\simplecit{\textcolor{656565}{#1}}}
\newcommand{\quo}[1]{\cit{\it #1}}

%---to indent
\newcommand{\ind}[1][20pt]{\advance\leftskip + #1}
\newcommand{\deind}[1][20pt]{\advance\leftskip - #1}

%---to indent a text
\newcommand{\indented}[2][20pt]{\par \ind[#1] #2 \par \deind[#1]}
\newenvironment{indt}[2][20pt]{#2 \par \ind[#1]}{\par \deind} %Titled indented env

%---title
\newcommand{\thetitle}[2]{\begin{center}\textbf{{\LARGE \underline{\emph{#1} :}} {\Large #2}}\end{center}}


%------Sections
% To change section numbering :
% \renewcommand\thesection{\Roman{section}}
% \renewcommand\thesubsection{\arabic{subsection}}
% \renewcommand\thesubsubsection{\textit \alph{subsubsection}}

% To start numbering from 0
% \setcounter{section}{-1}


%------page style
\usepackage{fancyhdr}
\usepackage{lastpage}

\setlength{\headheight}{18pt}
\setlength{\footskip}{50pt}

\pagestyle{fancy}
\fancyhf{}
\fancyhead[LE, RO]{\textit{\textcolor{black}{\today}}}
\fancyhead[RE, LO]{\large{\textsl{\emph{\texttt{\jobname}}}}}

\fancyfoot[RO, LE]{\textit{\texttt{\textcolor{black}{Page \thepage /}\pageref{LastPage}}}} %Change 'black' to 'white' for dark mode
\fancyfoot[LO, RE]{\includegraphics[scale=0.12]{/home/lasercata/Pictures/1.images_profil/logo/mieux/lasercata_logo_fly_fond_blanc.png}}

% For dark mode :
%/home/lasercata/Pictures/1.images_profil/logo/mieux/lasercata_logo_fly.png


%------init lengths
\setlength{\parindent}{0pt} %To avoid using \noindent everywhere.
\setlength{\parskip}{3pt}


%---------------------------------Begin Document
\begin{document}
    
    %For dark mode :
    % \pagecolor{black}
    % \color{white}
    
    \thetitle{Maths}{Applications linéaires}
    
    \tableofcontents
    \newpage


    Dans tout ce qui suit, $K$ désigne un corps, et $E, F$ désignent des $K$-espaces vectoriels.
    
    
    \begin{indt}{\section{Applications linéaires}}
        
        \begin{indt}{\subsection{Premières définitions et propriétés}}
            \begin{indt}{\subsubsection{Définition (\textit{Application linéaire})}}
                Une \textit{application linéaire} de $E$ vers $F$ est une application $f \in F^E$ telle que :
                \[
                    \forall (x, y) \in E^2, \forall \lambda \in K,\
                    \begin{cases}
                        f(x + y) = f(x) + f(y)
                        \\
                        f(\lambda x) = \lambda f(x)
                    \end{cases}
                \]

                On note $\mathcal L(E, F)$ l'ensemble des applications linéaires de $F^E$.
            \end{indt}

            \vspace{12pt}
            
            \begin{indt}{\subsubsection{Définition (\textit{endomorphisme})}}
                Un \textit{endomorphisme} de $E$ est une application linéaire de $E^E$.

                On note $\mathcal L(E) = \mathcal L(E, E)$ l'ensemble des endomorphismes de $E$.
            \end{indt}

            \vspace{12pt}
            
            \begin{indt}{\subsubsection{Définition (\textit{forme linéaire, dual})}}
                Une \textit{forme linéaire} sur $E$ est une application linéaire de $K^E$.

                On note $E^* = \mathcal L(E, K)$ l'ensemble des formes linéaires sur $E$, aussi appelé le \textit{dual} de $E$.
            \end{indt}

            \vspace{12pt}
            
            \begin{indt}{\subsubsection{Caractérisation des applications linéaires}}
                Soit $f \in F^E$.

                On a :
                \[
                    f \in \mathcal L(E, F)
                    \ssi
                    \forall ((x, y), (\lambda, \mu)) \in E^2 \times K^2,\
                    f(\lambda x + \mu y) = \lambda f(x) + \mu f(y)
                \]

                Ou plus simplement :
                \[
                    f \in \mathcal L(E, F)
                    \ssi
                    \forall ((x, y), \lambda) \in E^2 \times K,\
                    f(\lambda x + y) = \lambda f(x) + f(y)
                \]
            \end{indt}

            \vspace{12pt}
            
            \begin{indt}{\subsubsection{Propriétés}}
                Soit $f \in \mathcal L(E, F)$.

                $\bullet$ On a :
                \[
                    f(0) = 0.
                \]

                $\bullet$ Soit $n \in \N$, $(x_k)_{k \in \nset 1 n} \subset E$, et $(\lambda_k)_{k \in \nset 1 n} \subset K$.

                Alors :
                \[
                    f\!\lr{\sum_{k = 1}^n \lambda_k x_k} = \sum_{k = 1}^n \lambda_k f(x_k)
                \]
            \end{indt}
        \end{indt}

        \vspace{12pt}
        
        \begin{indt}{\subsection{Combinaisons linéaires et composition d'applications linéaire}}
            \begin{indt}{\subsubsection{Propriété}}
                L'ensemble $\mathcal L(E, F)$ est un sous-espace vectoriel de $F^E$, \textit{i.e} :
                \[
                    \forall ((f, g), (\lambda, \mu)) \in \lr{\mathcal L(E, F)}^2 \times K^2,\
                    \lambda f + \mu g \in \mathcal L(E, F)
                \]
            \end{indt}

            \vspace{12pt}
            
            \begin{indt}{\subsubsection{Composition d'applications linéaires}}
                Soient $E, F, G$ des $K$-espaces vectoriels, et
                $
                    \begin{array}{|l}
                        f \in \mathcal L(E, F)
                        \\
                        g \in \mathcal L(F, G)
                    \end{array}
                $

                Alors $g \circ f \in \mathcal L(E, G)$.
            \end{indt}

            \vspace{12pt}
            
            \begin{indt}{\subsubsection{Propriété (\textit{anneau des endomorphismes})}}
                $(\mathcal L(E), +, \circ)$ est un anneau.

                \vspace{6pt}
                
                \begin{indt}{Remarques :}
                    $-$ On note, pour $n \in \N^*$, %$f^n = \underbrace{f \circ f \circ \cdots \circ f}_{n\ \text{facteurs}}$.
                    $f^n = f^{n - 1} \circ f$, avec $f^0 = \mathrm{id}$.

                    $-$ $\lr{E^E, +, \circ}$ n'est pas un anneau : pour la distributivité à gauche, il faut la linéarité des fonctions.
                \end{indt}
            \end{indt}
        \end{indt}

        \vspace{12pt}
        
        \begin{indt}{\subsection{Isomorphismes, automorphismes}}
            \begin{indt}{\subsubsection{Définition (\textit{isomorphisme})}}
                Soit $f \in \mathcal L(E, F)$.

                Alors $f$ est un \textit{isomorphisme} de $E$ vers $F$ si elle est bijective.
            \end{indt}

            \vspace{12pt}
            
            \begin{indt}{\subsubsection{Définition (\textit{automorphisme})}}
                Soit $f \in \mathcal L(E)$.

                Alors $f$ est un \textit{automorphisme} de $E$ si elle est bijective, \textit{i.e} un automorphisme est un endomorphisme bijectif.

                On note $GL(E)$ l'ensemble des automorphismes de $E$.
            \end{indt}

            \vspace{12pt}
            
            \begin{indt}{\subsubsection{Définition (\textit{Espaces isomorphes})}}
                Soient $E, F$ deux $K$-espaces vectoriels.

                Alors $E$ et $F$ sont isomorphes si et seulement si
                \[
                    \exists f \in F^E\ |\ f\ \text{isomorphe}
                \]
            \end{indt}

            \vspace{12pt}
            
            \begin{indt}{\subsubsection{Propriété (\textit{linéarité de la réciproque d'un isomorphisme})}}
                Soit $f \in \mathcal L(E, F)$ un isomorphisme.

                Alors :
                \[
                    f^{-1} \in \mathcal L(F, E)
                \]
            \end{indt}

            \vspace{12pt}
            
            \begin{indt}{\subsubsection{Propriété}}
                Soit $E$ un $K$-espace vectoriel.

                Alors $GL(E) \subset \mathcal L(E)$, et $(GL(E), \circ)$ est un sous-groupe de $(S_E, \circ)$.
            \end{indt}
        \end{indt}
        
    \end{indt}

    \vspace{12pt}
    
    \begin{indt}{\section{Noyau et image}}
        \begin{indt}{\subsection{Définitions}}
            \begin{indt}{\subsubsection{Définition (\textit{Noyau})}}
                Soit $f \in \mathcal L(E, F)$.

                Le \textit{noyau} de $f$ est l'ensemble
                \[
                    \mathrm{Ker}(f) = \set{x \in E\ \left| \vphantom{\frac a a} \right.\ f(x) = 0_F}
                    = f^{-1}\!\lr{\set{0_F}}
                \]
            \end{indt}

            \vspace{12pt}
            
            \begin{indt}{\subsubsection{Définition (\textit{Image})}}
                Soit $f \in \mathcal L(E, F)$.

                L'\textit{image} de $f$ est l'ensemble
                \[
                    \mathrm{Im}(f) = f(E)
                    = \set{y \in F\ \left| \vphantom{\frac a a} \right.\ \exists x \in E\ |\ y = f(x)}
                \]
            \end{indt}

            \vspace{12pt}
            
            \begin{indt}{\subsubsection{Propriétés}}
                Soit $f \in \mathcal L(E, F)$.

                $\bullet$ On a :

                \[ 0_E \in \mathrm{Ker}(f) \]
                \[ 0_F \in \mathrm{Im}(f) \]

                $\bullet$ $\mathrm{Ker}(f)$ est un sous-espace vectoriel de $E$.

                $\bullet$ $\mathrm{Im}(f)$ est un sous-espace vectoriel de $F$.
            \end{indt}

            \vspace{12pt}
            
            \begin{indt}{\subsubsection{Théorème (\textit{CNS d'injectivité et de surjectivité})}}
                Soit $f \in \mathcal L(E, F)$.

                Alors :
                \[
                    \begin{array}{clcl}
                        f & \text{injective} & \ssi & \mathrm{Ker}(f) = \set 0
                        \\
                        f & \text{surjective} & \ssi & \mathrm{Im}(f) = F
                    \end{array}
                \]
            \end{indt}
        \end{indt}

        \vspace{12pt}
        
        \begin{indt}{\subsection{Propriétés sur les noyaux et les images}}
            \begin{indt}{\subsubsection{Propriété (\textit{composition})}}
                Soient $E, F, G$ des $K$-espaces vectoriels, et
                $
                    \begin{array}{|l}
                        f \in \mathcal L(E, F)
                        \\
                        g \in \mathcal L(F, G)
                    \end{array}
                $.

                Alors :
                \[
                    \mathrm{Im}(g \circ f)
                    = g\!\lr{f(E)}
                    = g\!\lr{\mathrm{Im}(f)}
                    = \mathrm{Im}\!\lr{g_{| \mathrm{Im}(f)}}
                \]
            \end{indt}

            \vspace{12pt}
            
            \begin{indt}{\subsubsection{Propriétés}}
                Soit $f \in \mathcal L(E)$.

                $\bullet$ On a :
                \[
                    \mathrm{Im}\!\lr{f^2} \subset \mathrm{Im}(f)
                \]
                \[
                    \mathrm{Ker}(f) \subset \mathrm{Ker}\!\lr{f^2}
                \]

                \vspace{6pt}
                
                $\bullet$ On a :
                \[
                    f^2 = 0 \ssi \mathrm{Im}(f) \subset \mathrm{Ker}(f)
                \]
            \end{indt}

            \vspace{12pt}
            
            \begin{indt}{\subsubsection{Propriété (\textit{noyau d'une restriction})}}
                Soit $f \in \mathcal L(E, F)$, et soit $G$ un sous-espace vectoriel de $E$.

                On a :
                \[
                    \mathrm{Ker}(f_{| G}) = \mathrm{Ker}(f) \cap G
                \]
            \end{indt}

            \vspace{12pt}
            
            \begin{indt}{\subsubsection{Propriété (\textit{antécédents par une application linéaire})}}
                Soit $f \in \mathcal L(E, F)$, et $y_0 \in \mathrm{Im}(f)$.

                Soit alors $x_0 \in E\ |\ y_0 = f(x_0)$.

                Alors :
                \[
                    \set{x \in E\ |\ y_0 = f(x)}
                    = x_0 + \mathrm{Ker}(f)
                    = \set{x_0 + v\ |\ v \in \mathrm{Ker}(f)}
                \]
                (c'est l'ensemble des antécédents de $y_0$ par $f$).

                \textit{I.e} pour $x \in E$, $y_0 = f(x) \ssi f(x) - y_0 = 0 \ssi f(x) - f(x_0) = 0 \ssi f(x - x_0) = 0 \ssi x - x_0 \in \mathrm{Ker}(f)$
            \end{indt}
        \end{indt}
    \end{indt}

    \vspace{12pt}
    
    \begin{indt}{\section{Image de familles par une application linéaire}}
        Dans ce paragraphe, $I$ désigne un ensemble d'indexation.

        \begin{indt}{\subsection{Propriété}}
            Soit $(e_k)_{k \in I} \subset E$ une famille de $E$, et $u \in \mathcal L(E, F)$.

            Alors
            \[
                u\!\lr{\mathrm{vect}(e_k)_{k \in I }} = \mathrm{vect}\!\lr{u\!\lr{e_k}}_{k \in I}
            \]
        \end{indt}

        \vspace{12pt}
        
        \begin{indt}{\subsection{Corollaire}}
            Soit $(e_k)_{k \in I} \subset E\ |\ E = \mathrm{vect}(e_k)_{k \in I}$ (\textit{i.e} une famille génératrice de $E$), et soit $u \in \mathcal L(E, F)$.

            Alors :

            $\bullet$ La famille $(u(e_k))_{k \in I}$ est une famille génératrice de $\mathrm{Im}(u)$, \textit{i.e} :
            \[
                \mathrm{Im}(u) = \mathrm{vect}(u(e_k))_{k \in I}
            \]

            $\bullet$ $u$ surjective $\ssi F = \mathrm{vect}(u(e_k))_{k \in I}$
        \end{indt}

        \vspace{12pt}
        
        \begin{indt}{\subsection{Propriétés}}
            Soit $\mathcal F = (e_k)_{k \in I} \subset E$, et $u \in \mathcal L(E, F)$.

            \vspace{6pt}
            
            $\bullet$ Si
            $
                \begin{cases}
                    \mathcal F\ \text{est libre}
                    \\
                    u\ \text{est injective}
                \end{cases}
            $, alors la famille $(u(e_k))_{k \in I}$ est libre.

            \vspace{6pt}
            
            $\bullet$ Si $\mathcal F$ est une base de $E$, alors :
            \[
                u\ \text{injective} \ssi (u(e_k))_{k \in I}\ \text{libre}
            \]
        \end{indt}

        \vspace{12pt}
        
        \begin{indt}{\subsection{Corollaire}}
            Soit $(e_k)_{k \in I} \subset E$ une base de $E$, et $u \in \mathcal L(E, F)$.

            Alors $u$ est un isomorphisme $\ssi$ $(u(e_k))_{k \in I}$ est une base de $F$.
        \end{indt}

        \vspace{12pt}
        
        \begin{indt}{\subsection{Théorème de prolongement par linéarité}}
            Soit $(e_k)_{k \in I} \subset E$ une base de $E$.

            Alors :
            \[
                \forall (f_k)_{k \in I} \subset F,\
                \exists!\: u \in \mathcal L(E, F)\ |\
                \forall k \in I,\ u(e_k) = f_k
            \]
        \end{indt}
    \end{indt}

    \vspace{12pt}
    
    \begin{indt}{\section{Applications linéaires en dimension finie}}
        Dans ce paragraphe, on suppose que le $K$-espace vectoriel $E$ est de dimension finie non nulle, et on note $n = \dim(E) \in \N^*$. De plus, $I$ désigne un ensemble d'indexation.

        \begin{indt}{\subsection{Image de familles libres, génératrices, bases}}
            \begin{indt}{\subsubsection{Propriétés}}
                Soit $\mathcal F = (e_k)_{k \in I} \subset E$, et $u \in \mathcal L(E, F)$.

                \vspace{6pt}
                
                $\bullet$ Si $u$ est injective, alors
                \[
                    \dim(E) \le \dim(F)
                \]
                De plus, si $\mathcal F$ est libre, alors $u(\mathcal F) = (u(e_k))_{k \in I}$ est libre.

                \vspace{12pt}
                
                $\bullet$ Si $u$ est surjective, alors
                \[
                    \dim(E) \ge \dim(F)
                \]
                De plus, si $E = \mathrm{vect}(\mathcal F)$, alors $F = \mathrm{vect}(u(\mathcal F))$.

                \vspace{12pt}
                
                $\bullet$ Si $u$ est bijective, alors
                \[
                    \dim(E) = \dim(F)
                \]
                De plus, si $\mathcal F$ est une base de $E$, alors $u(\mathcal F)$ est une base de $F$.
            \end{indt}

            \vspace{12pt}
            
            \begin{indt}{\subsubsection{Théorème (\textit{CNS d'injectivité, de surjectivité, et de bijection})}}
                Soit $\mathcal B = (e_k)_{k \in \nset 1 n} \subset E$ une base de $E$, et $u \in \mathcal L(E, F)$.

                \vspace{6pt}
                
                $\bullet$ La fonction $u$ est injective $\ssi$ $u(\mathcal B)$ est libre.

                $\bullet$ La fonction $u$ est surjective $\ssi$ $F = \mathrm{vect}(u(\mathcal B))$.

                $\bullet$ La fonction $u$ est bijective $\ssi$ $u(\mathcal B)$ est une base de $F$.
            \end{indt}

            \vspace{12pt}
            
            \begin{indt}{\subsubsection{Théorème de prolongement par linéarité}}
                Soit $\mathcal B = (e_k)_{k \in \nset 1 n} \subset E$ une base de $E$.

                Alors
                \[
                    \forall (v_k)_{k \in \nset 1 n} \subset F,\
                    \exists!\: \varphi \in \mathcal L(E, F)\ |\
                    \forall k \in \nset 1 n,\
                    v_k = \varphi(e_k)
                \]

                De plus, pour $\displaystyle x = \sum_{k = 1}^n x_k e_k \in E$ (où $(x_k)_{k \in \nset 1 n} \subset K$), on a :
                \[
                    \varphi(x)
                    = \varphi\!\lr{\sum_{k = 1}^n x_k e_k}
                    = \sum_{k = 1}^n x_k \varphi(e_k)
                    = \sum_{k = 1}^n x_k v_k
                \]

                En particulier, pour $f, g \in \mathcal L(E, F)$, on a :
                \[
                    f = g \ssi \forall k \in \nset 1 n,\ f(e_k) = g(e_k)
                \]
            \end{indt}

            \vspace{12pt}
            
            \begin{indt}{\subsubsection{Corollaire}}
                Soit $G$ un $K$-espace vectoriel.

                Alors $G$ est isomorphe à $E$ $\ssi \dim(G) = n < +\infty$
            \end{indt}
        \end{indt}

        \vspace{12pt}
        
        \begin{indt}{\subsection{Théorème du rang}}
            \begin{indt}{\subsubsection{Définition (\textit{rang})}}
                Soit $f \in \mathcal L(E, F)$.

                Le \textit{rang} de $f$, noté $\mathrm{rang}(f)$ ou $\mathrm{rg}(f)$, est :
                \[
                    \mathrm{rang}(f) = \dim(\mathrm{Im}(f))
                \]
            \end{indt}

            \vspace{12pt}
            
            \begin{indt}{\subsubsection{Propriétés}}
                Soit $f \in \mathcal L(E, F)$, et $(e_k)_{k \in \nset 1 n} \subset E$ une base de $E$.

                $\bullet$ On a :
                \[
                    \mathrm{rang}(f) = \mathrm{rang}\!\lr{f(e_k)}_{k \in \nset 1 n}
                \]

                $\bullet$ On a :
                \[
                    \mathrm{rang}(f) \le \min\!\lr{\dim(E), \dim(F)}
                \]
            \end{indt}

            \vspace{12pt}
            
            \begin{indt}{\subsubsection{Propriété (\textit{CNS d'injectivité et de surjectivité avec le rang})}}
                Soit $f \in \mathcal L(E, F)$.
                
                On a :

                $\bullet$ $f$ est injective $\ssi \mathrm{rang}(f) = \dim(E)$

                $\bullet$ $f$ est surjective $\ssi \mathrm{rang}(f) = \dim(F)$
            \end{indt}

            \vspace{12pt}
            
            \begin{indt}{\subsubsection{Propriétés (\textit{conservation du rang par les injections / surjections})}}
                Soient $E, F, G$ trois $K$-espaces vectoriels de dimension finie, et
                $
                    \begin{array}{|l}
                        f \in \mathcal L(E, F)
                        \\
                        g \in \mathcal L(F, G)
                    \end{array}
                $.

                Alors :

                $\bullet$ Si $f$ est surjective, on a :
                \[
                    \mathrm{rang}(g \circ f) = \mathrm{rang}(g)
                \]

                $\bullet$ Si $g$ est injective, on a :
                \[
                    \mathrm{rang}(g \circ f) = \mathrm{rang}(f)
                \]
            \end{indt}

            \vspace{12pt}
            
            \begin{indt}{\subsubsection{Théorème du rang}}
                Soit $f \in \mathcal L(E, F)$.

                On a :
                \[
                    \dim(E) = \dim(\mathrm{Ker}(f)) + \mathrm{rang}(f)
                \]
            \end{indt}

            \vspace{12pt}
            
            \begin{indt}{\subsubsection{Théorème}}
                Soient $E, F$ deux $K$-espaces vectoriels de dimension finie tels que
                \[
                    \dim(E) = \dim(F),
                \]
                et soit $f \in \mathcal L(E, F)$.

                \vspace{6pt}
                
                On a alors :
                \[
                    f\ \text{injective}
                    \ssi
                    f\ \text{surjective}
                    \ssi
                    f\ \text{bijective}
                \]
            \end{indt}
        \end{indt}
    \end{indt}
    
    
    
\end{document}
%--------------------------------------------End
