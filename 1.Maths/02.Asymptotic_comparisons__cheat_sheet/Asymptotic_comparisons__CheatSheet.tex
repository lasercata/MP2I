\documentclass[a4paper, 12pt, twoside]{article}


%------------------------------------------------------------------------
%
% Author                :   Lasercata
% Last modification     :   2022.06.12
%
%------------------------------------------------------------------------


%------ini
\usepackage[utf8]{inputenc}
\usepackage[T1]{fontenc}
%\usepackage[french]{babel}
\usepackage[english]{babel}


%------geometry
\usepackage[textheight=700pt, textwidth=500pt]{geometry}


%------color
\usepackage{xcolor}
\definecolor{ff4500}{HTML}{ff4500}
\definecolor{00f}{HTML}{0000ff}
\definecolor{0ff}{HTML}{00ffff}
\definecolor{656565}{HTML}{656565}

\renewcommand{\emph}{\textcolor{ff4500}}
\renewcommand{\em}{\color{ff4500}}

\newcommand{\strong}[1]{\textcolor{ff4500}{\bf #1}}
\newcommand{\st}{\color{ff4500}\bf}


%------Code highlighting
\usepackage{listings}

\definecolor{cbg}{HTML}{272822}
\definecolor{cfg}{HTML}{ececec}
\definecolor{ccomment}{HTML}{686c58}
\definecolor{ckw}{HTML}{f92672}
\definecolor{cstring}{HTML}{e6db72}
\definecolor{cstringlight}{HTML}{98980f}
\definecolor{lightwhite}{HTML}{fafafa}

\lstdefinestyle{DarkCodeStyle}{
    backgroundcolor=\color{cbg},
    commentstyle=\itshape\color{ccomment},
    keywordstyle=\color{ckw},
    numberstyle=\tiny\color{cbg},
    stringstyle=\color{cstring},
    basicstyle=\ttfamily\footnotesize\color{cfg},
    breakatwhitespace=false,
    breaklines=true,
    captionpos=b,
    keepspaces=true,
    numbers=left,
    numbersep=5pt,
    showspaces=false,
    showstringspaces=false,
    showtabs=false,
    tabsize=4,
    xleftmargin=\leftskip
}

\lstdefinestyle{LightCodeStyle}{
    backgroundcolor=\color{lightwhite},
    commentstyle=\itshape\color{ccomment},
    keywordstyle=\color{ckw},
    numberstyle=\tiny\color{cbg},
    stringstyle=\color{cstringlight},
    basicstyle=\ttfamily\footnotesize\color{cbg},
    breakatwhitespace=false,
    breaklines=true,
    captionpos=b,
    keepspaces=true,
    numbers=left,
    numbersep=10pt,
    showspaces=false,
    showstringspaces=false,
    showtabs=false,
    tabsize=4,
    frame=L,
    xleftmargin=\leftskip
}

%\lstset{style=DarkCodeStyle}
\lstset{style=LightCodeStyle}

%Usage : \begin{lstlisting}[language=Caml] ... \end{lstlisting}


%-------make the table of content clickable
\usepackage{hyperref}
\hypersetup{
    colorlinks,
    citecolor=black,
    filecolor=black,
    linkcolor=black,
    urlcolor=black
}


%------pictures
\usepackage{graphicx}
%\usepackage{wrapfig}


%------tabular
%\usepackage{color}
%\usepackage{colortbl}
%\usepackage{multirow}


%------Physics
%---Packages
%\usepackage[version=4]{mhchem} %$\ce{NO4^2-}$

%---Commands
\newcommand{\link}[2]{\mathrm{#1} \! - \! \mathrm{#2}}
\newcommand{\pt}[1]{\cdot 10^{#1}} % Power of ten
\newcommand{\dt}[2][t]{\dfrac{d#2}{d#1}} % Derivative


%------math
%---Packages
%\usepackage{textcomp}
%\usepackage{amsmath}
\usepackage{amssymb}
\usepackage{mathtools} % For abs
\usepackage{stmaryrd} %for \llbracket and \rrbracket
\usepackage{mathrsfs} %for \mathscr{x} (different from \mathcal{x})

%---Commands
%-Sets
\newcommand{\N}{\mathbb{N}} %set N
\newcommand{\R}{\mathbb{R}} %set R
\newcommand{\C}{\mathbb{C}} %set C
\newcommand{\U}{\mathbb{U}} %set U
\newcommand{\set}[2]{\left[ #1\ ;\ #2 \right]}
\newcommand{\nset}[2]{\left\llbracket #1\ ;\ #2 \right\rrbracket}

%-Exponantial / complexs
\newcommand{\e}[1]{\mathrm{e}^{#1}}
\newcommand{\ex}{\e{x}}
\newcommand{\cj}[1]{\overline{#1}} %overline for the conjugate.

%-Vectors
\newcommand{\vect}{\overrightarrow}
\newcommand{\veco}[3]{\displaystyle \vect{#1}\binom{#2}{#3}} %vector + coord

%-Limits
\newcommand{\lm}[2][{}]{\lim\limits_{\substack{#2 \\ #1}}} %$\lm{x \to a} f$ or $\lm[x < a]{x \to a} f$
\newcommand{\Lm}[3][{}]{\lm[#1]{#2} \left( #3 \right)} %$\Lm{x \to a}{f}$ or $\Lm[x < a]{x \to a}{f}$
\newcommand{\tendsto}[1]{\xrightarrow[#1]{}}

%-Integral
\newcommand{\dint}[4][x]{\displaystyle \int_{#2}^{#3} #4 \mathrm{d} #1} %$\dint{a}{b}{f(x)}$ or $\dint[t]{a}{b}{f(t)}$

%-Others
\newcommand{\para}{\ /\!/\ } %//
\newcommand{\ssi}{\ \Leftrightarrow \ }
\newcommand{\abs}[1]{\left\lvert #1 \right\rvert} % abs{x} -> |x|
\newcommand{\eqsys}[2]{\begin{cases} #1 \\ #2 \end{cases}}

\newcommand{\med}[2]{\mathrm{med} \left[ #1\ ;\ #2 \right]}  %$\med{A}{B} -> med[A ; B]$
\newcommand{\Circ}[2]{\mathscr{C}_{#1, #2}}

\newcommand{\lr}[1]{\left( #1 \right)}
\newcommand{\lrb}[1]{\left[ #1 \right]}


%------commands
%---to quote french text
\newcommand{\simplecit}[1]{\guillemotleft$\;$#1$\;$\guillemotright}
\newcommand{\cit}[1]{\simplecit{\textcolor{656565}{#1}}}
\newcommand{\quo}[1]{\cit{\it #1}}

%---to indent
\newcommand{\ind}[1][20pt]{\advance\leftskip + #1}
\newcommand{\deind}[1][20pt]{\advance\leftskip - #1}

%---to indent a text
\newcommand{\indented}[2][20pt]{\par \ind[#1] #2 \par \deind[#1]}
\newenvironment{indentedenv}[1][20pt]{\par \ind[#1]}{\par \deind}
\newenvironment{indt}[2][20pt]{#2 \begin{indentedenv}[#1]}{\end{indentedenv}} %Titled indented env

%---title
\newcommand{\thetitle}[2]{\begin{center}\textbf{{\LARGE \underline{\emph{#1} :}} {\Large #2}}\end{center}}

%---parts
%-I
\newcommand{\mainpart}[2][$\!\!$]{\underline{\large \textbf{\emph{\textit{#1} #2}}}}
\newcommand{\bmainpart}[2][$\!\!$]{\underline{\large \textbf{\textit{#1} #2}}}
%-A
\newcommand{\subpart}[2][$\!\!$]{\underline{\bf \textit{#1} #2}}
%-1
\newcommand{\subsubpart}[2][$\!\!$]{\underline{\textsl{#1} #2}}
%-a
\newcommand{\subsubsubpart}[2][$\!\!$]{\underline{\it #1 #2}}

%math part
\newcommand{\secpart}[1]{.\underline{\it #1 :}}

\newenvironment{mathdef}[2][20pt]{
    \secpart{#2} \begin{indentedenv}[#1]}
    {\end{indentedenv}}


%------page style
\usepackage{fancyhdr}
\usepackage{lastpage}

\setlength{\headheight}{18pt}
\setlength{\footskip}{50pt}

\pagestyle{fancy}
\fancyhf{}
\fancyhead[LE, RO]{\textit{\today}}
\fancyhead[RE, LO]{\large{\textsl{\emph{\texttt{\jobname}}}}}

\fancyfoot[RO, LE]{\textit{\texttt{Page \thepage /\pageref{LastPage}}}}
\fancyfoot[LO, RE]{\includegraphics[scale=0.12]{/home/lasercata/Pictures/1.images_profil/logo/mieux/lasercata_logo_fly_fond_blanc.png}}


%------init lengths
\setlength{\parindent}{0pt} %no \noindent needed !!!
\setlength{\parskip}{3pt}


%---------------------------------Begin Document
\begin{document}

    \thetitle{Maths}{Asymptotic comparison}    
    
    
    \begin{indt}{\section{Definitions :}}
        
        $(u_n)$ est \emph{dominée} par $(v_n)$ :
        \vspace{-12pt}
        \[u_n = O(v_n) \ssi \exists (M, n_0) \in \R \times \N\ |\ \forall n \in \N,\ n \ge n_0 \Rightarrow \abs{\dfrac{u_n}{v_n}} \le M \]
        
        \vspace{6pt}
        
        $(u_n)$ est \emph{négligeable} devant $(v_n)$ :
        \vspace{-12pt}
        \[u_n = o(v_n) \ssi \dfrac{u_n}{v_n} \tendsto{n \to +\infty} 0\]
        
        \vspace{6pt}
        
        $(u_n)$ est \emph{équivalente} à $(v_n)$ :
        \vspace{-12pt}
        \[ u_n \sim v_n \ssi \dfrac{u_n}{v_n} \tendsto{n \to +\infty} 1 \]
    
    \end{indt}
    
    \begin{indt}{\section{Proprieties :}}
        
        \begin{indt}{\subsection{}}
            $\forall l \in \R^*,\quad u_n \sim l \ssi u_n \tendsto{n \to +\infty} l, \quad (u_n) \in \R^\N$
            
            \vspace{6pt}
            
            $u_n \sim v_n$ et $u_n' \sim v_n' \Rightarrow u_n u_n' \sim v_n v_n'$
            
            \vspace{6pt}
            
            $u_n \sim v_n \ssi u_n - v_n = o(v_n)$
        \end{indt}
        
        \begin{indt}{\subsection{Sums}}
            
            $u_n = o(v_n) \Rightarrow u_n + v_n \sim v_n
            \begin{array}{rcl}
                \square\
                u_n = o(v_n)
                &\Rightarrow& \dfrac{u_n}{v_n} \tendsto{n \to +\infty} 0
                \\\\
                &\Rightarrow& \dfrac{u_n}{v_n} + 1 \tendsto{n \to +\infty} 1
                \\\\
                &\Rightarrow& \dfrac{u_n + v_n}{v_n} \tendsto{n \to +\infty} 1
                \\\\
                &\Rightarrow& u_n + v_n \sim v_n
                \quad \blacksquare
            \end{array}$
            
            \vspace{12pt}
            
            $\begin{cases}
                u_n \sim \lambda w_n
                \\
                v_n \sim \mu w_n
                \\
                \lambda + \mu \neq 0
            \end{cases}
            \Rightarrow
            u_n + v_n \sim (\lambda + \mu)w_n
            $
            
            $\begin{array}{rcl}
                \square
                \begin{cases}
                    u_n \sim \lambda w_n
                    \\
                    v_n \sim \mu w_n
                    \\
                    \lambda + \mu \neq 0
                \end{cases}
                &\Rightarrow&
                \begin{cases}
                    u_n = \lambda w_n + o(w_n)
                    \\
                    v_n = \mu w_n + o(w_n)
                    \\
                    \lambda + \mu \neq 0
                \end{cases}
                \\
                &\Rightarrow&
                \begin{cases}
                    u_n + v_n = (\lambda + \mu)w_n + o(w_n)
                    \\
                    \lambda + \mu \neq 0
                \end{cases}
                \\
                &\Rightarrow&
                u_n + v_n \sim (\lambda + \mu)w_n
                \quad \blacksquare
            \end{array}$
            
        \end{indt}
        
    \end{indt}
    

    
    
\end{document}
%--------------------------------------------End
