\documentclass[a4paper, 12pt, twoside]{article}


%------------------------------------------------------------------------
%
% Author                :   Lasercata
% Last modification     :   2022.03.05
%
%------------------------------------------------------------------------


%------ini
\usepackage[utf8]{inputenc}
\usepackage[T1]{fontenc}
%\usepackage[french]{babel}
%\usepackage[english]{babel}


%------geometry
\usepackage[textheight=700pt, textwidth=500pt]{geometry}


%------color
\usepackage{xcolor}
\definecolor{ff4500}{HTML}{ff4500}
\definecolor{00f}{HTML}{0000ff}
\definecolor{0ff}{HTML}{00ffff}
\definecolor{656565}{HTML}{656565}

\renewcommand{\emph}{\textcolor{ff4500}}
\renewcommand{\em}{\color{ff4500}}

\newcommand{\strong}[1]{\textcolor{ff4500}{\bf #1}}
\newcommand{\st}{\color{ff4500}\bf}


%------Code highlighting
\usepackage{listings}

\definecolor{cbg}{HTML}{272822}
\definecolor{cfg}{HTML}{ececec}
\definecolor{ccomment}{HTML}{686c58}
\definecolor{ckw}{HTML}{f92672}
\definecolor{cstring}{HTML}{e6db72}
\definecolor{cstringlight}{HTML}{98980f}
\definecolor{lightwhite}{HTML}{fafafa}

\lstdefinestyle{DarkCodeStyle}{
    backgroundcolor=\color{cbg},
    commentstyle=\itshape\color{ccomment},
    keywordstyle=\color{ckw},
    numberstyle=\tiny\color{cbg},
    stringstyle=\color{cstring},
    basicstyle=\ttfamily\footnotesize\color{cfg},
    breakatwhitespace=false,
    breaklines=true,
    captionpos=b,
    keepspaces=true,
    numbers=left,
    numbersep=5pt,
    showspaces=false,
    showstringspaces=false,
    showtabs=false,
    tabsize=4,
    xleftmargin=\leftskip
}

\lstdefinestyle{LightCodeStyle}{
    backgroundcolor=\color{lightwhite},
    commentstyle=\itshape\color{ccomment},
    keywordstyle=\color{ckw},
    numberstyle=\tiny\color{cbg},
    stringstyle=\color{cstringlight},
    basicstyle=\ttfamily\footnotesize\color{cbg},
    breakatwhitespace=false,
    breaklines=true,
    captionpos=b,
    keepspaces=true,
    numbers=left,
    numbersep=10pt,
    showspaces=false,
    showstringspaces=false,
    showtabs=false,
    tabsize=4,
    frame=L,
    xleftmargin=\leftskip
}

%\lstset{style=DarkCodeStyle}
\lstset{style=LightCodeStyle}

%Usage : \begin{lstlisting}[language=Caml] ... \end{lstlisting}


%-------make the table of content clickable
\usepackage{hyperref}
\hypersetup{
    colorlinks,
    citecolor=black,
    filecolor=black,
    linkcolor=black,
    urlcolor=black
}


%------pictures
\usepackage{graphicx}
%\usepackage{wrapfig}


%------tabular
%\usepackage{color}
%\usepackage{colortbl}
%\usepackage{multirow}


%------Physics
%---Packages
%\usepackage[version=4]{mhchem} %$\ce{NO4^2-}$

%---Commands
\newcommand{\link}[2]{\mathrm{#1} \! - \! \mathrm{#2}}
\newcommand{\pt}[1]{\cdot 10^{#1}} % Power of ten
\newcommand{\dt}[2][t]{\dfrac{d#2}{d#1}} % Derivative


%------math
%---Packages
%\usepackage{textcomp}
%\usepackage{amsmath}
\usepackage{amssymb}
\usepackage{mathtools} % For abs
\usepackage{stmaryrd} %for \llbracket and \rrbracket
\usepackage{mathrsfs} %for \mathscr{x} (different from \mathcal{x})

%---Commands
%-Sets
\newcommand{\N}{\mathbb{N}} %set N
\newcommand{\Z}{\mathbb{Z}} %set Z
\newcommand{\Q}{\mathbb{Q}} %set Q
\newcommand{\R}{\mathbb{R}} %set R
\newcommand{\C}{\mathbb{C}} %set C
\newcommand{\U}{\mathbb{U}} %set U
\newcommand{\seg}[2]{\left[ #1\ ;\ #2 \right]}
\newcommand{\nset}[2]{\left\llbracket #1\ ;\ #2 \right\rrbracket}

%-Exponantial / complexs
\newcommand{\e}{\mathrm{e}}
\newcommand{\cj}[1]{\overline{#1}} %overline for the conjugate.

%-Vectors
\newcommand{\vect}{\overrightarrow}
\newcommand{\veco}[3]{\displaystyle \vect{#1}\binom{#2}{#3}} %vector + coord

%-Limits
\newcommand{\lm}[2][{}]{\lim\limits_{\substack{#2 \\ #1}}} %$\lm{x \to a} f$ or $\lm[x < a]{x \to a} f$
\newcommand{\Lm}[3][{}]{\lm[#1]{#2} \left( #3 \right)} %$\Lm{x \to a}{f}$ or $\Lm[x < a]{x \to a}{f}$
\newcommand{\tendsto}[1]{\xrightarrow[#1]{}}

%-Integral
\newcommand{\dint}[4][x]{\displaystyle \int_{#2}^{#3} #4 \mathrm{d} #1} %$\dint{a}{b}{f(x)}$ or $\dint[t]{a}{b}{f(t)}$

%-Others
\newcommand{\para}{\ /\!/\ } %//
\newcommand{\ssi}{\ \Leftrightarrow \ }
\newcommand{\abs}[1]{\left\lvert #1 \right\rvert} % abs{x} -> |x|
\newcommand{\eqsys}[2]{\begin{cases} #1 \\ #2 \end{cases}}

\newcommand{\med}[2]{\mathrm{med} \left[ #1\ ;\ #2 \right]}  %$\med{A}{B} -> med[A ; B]$
\newcommand{\Circ}[2]{\mathscr{C}_{#1, #2}}

\newcommand{\lr}[1]{\left( #1 \right)}
\newcommand{\lrb}[1]{\left[ #1 \right]}
\newcommand{\set}[1]{\left\{ #1 \right\}}

\newcommand{\lrangle}[1]{\left\langle #1 \right\rangle}

\renewcommand{\le}{\leqslant}
\renewcommand{\ge}{\geqslant}

\newcommand{\dom}{\mathrm{dom}}


%------commands
%---to quote french text
\newcommand{\simplecit}[1]{\guillemotleft$\;$#1$\;$\guillemotright}
\newcommand{\cit}[1]{\simplecit{\textcolor{656565}{#1}}}
\newcommand{\quo}[1]{\cit{\it #1}}

%---to indent
\newcommand{\ind}[1][20pt]{\advance\leftskip + #1}
\newcommand{\deind}[1][20pt]{\advance\leftskip - #1}

%---to indent a text
\newcommand{\indented}[2][20pt]{\par \ind[#1] #2 \par \deind[#1]}
\newenvironment{indt}[2][20pt]{#2 \par \ind[#1]}{\par \deind} %Titled indented env

%---title
\newcommand{\thetitle}[2]{\begin{center}\textbf{{\LARGE \underline{\emph{#1} :}} {\Large #2}}\end{center}}

%---parts
%-I
\newcommand{\mainpart}[2][$\!\!$]{\underline{\large \textbf{\emph{\textit{#1} #2}}}}
\newcommand{\bmainpart}[2][$\!\!$]{\underline{\large \textbf{\textit{#1} #2}}}
%-A
\newcommand{\subpart}[2][$\!\!$]{\underline{\bf \textit{#1} #2}}
%-1
\newcommand{\subsubpart}[2][$\!\!$]{\underline{\textsl{#1} #2}}
%-a
\newcommand{\subsubsubpart}[2][$\!\!$]{\underline{\it #1 #2}}


%------page style
\usepackage{fancyhdr}
\usepackage{lastpage}

\setlength{\headheight}{18pt}
\setlength{\footskip}{50pt}

\pagestyle{fancy}
\fancyhf{}
\fancyhead[LE, RO]{\textit{\today}}
\fancyhead[RE, LO]{\large{\textsl{\emph{\texttt{\jobname}}}}}

\fancyfoot[RO, LE]{\textit{\texttt{Page \thepage /\pageref{LastPage}}}}
\fancyfoot[LO, RE]{\includegraphics[scale=0.12]{/home/lasercata/Pictures/1.images_profil/logo/mieux/lasercata_logo_fly_fond_blanc.png}}


%------init lengths
\setlength{\parindent}{0pt} %To avoid using \noindent everywhere.
\setlength{\parskip}{3pt}


%---------------------------------Begin Document
\begin{document}

    \thetitle{Maths}{Polynomial}    
    
    \tableofcontents
    \newpage
    
    Dans tout ce qui suit, $K$ est un corps abélien, et $n, m \in \N$.
    
    \begin{indt}{\section{Définitions}}
        
        \begin{indt}{\subsection{Polynôme à une indéterminée}}
            Un polynôme à une indéterminée $X$ à coefficients dans $K$ est une somme formelle
            \[ \sum_{k = 0}^n a_k X^k \quad \text{avec} \ \forall k \in \nset{0}{n},\ a_k \in K \]
            
            Par convention, $\forall k > n,\ a_k = 0$.
            
            On note $K[X]$ l'ensemble des polynômes à une indéterminée à coefficients dans $K$.
        \end{indt}
        
        \vspace{6pt}
        
        \begin{indt}{\subsection{Degré d'un polynôme}}
            Soit $\displaystyle P = \sum_{k = 0}^n a_k X^k$.
            
            \vspace{6pt}
            
            Si $P = 0$, alors $\deg(P) = -\infty$.
            
            Sinon, $\deg(P) = \max \set{k \in \nset{0}{n}\ |\ a_k \neq 0,\ \forall j \in \nset{k + 1}{n},\ a_j = 0}$
            
            \vspace{12pt}
            
            On note $\dom(P) = a_{\deg(P)}$ (Attention, notation non universelle, à définir avant de l'utiliser)
            
            \vspace{6pt}
            
            $P$ est \textit{unitaire} $\ssi \dom(P) = 1$
            
            \vspace{6pt}
            
            On note $K_n[X] = \set{P \in K[X]\ |\ \deg(P) \le n}$
        \end{indt}
        
        \vspace{12pt}
        
        \begin{indt}{\subsection{Propriétés}}
            Soit $\displaystyle P = \sum_{k = 0}^n a_k X^k$. Alors :
            
            \vspace{6pt}
            
            \begin{indt}{}
                $\deg(P) \le n$
                
                $\deg(P) = n \ssi a_n \neq 0$
                
                $\exists p \in \nset{0}{n}\ |\ a_p \neq 0 \Rightarrow \deg(P) \ge p$
            \end{indt}
        \end{indt}
        
    \end{indt}
    
    \vspace{12pt}
    
    \begin{indt}{\section{Somme, produit}}
        
        Soient $\displaystyle P = \sum_{k = 0}^n a_k X^k$ et $\displaystyle Q = \sum_{k = 0}^m b_k X^k$.
        
        \begin{indt}{\subsection{Somme}}
            On définit la somme $P + Q$ par :
                \[ P + Q = \sum_{k = 0}^{\max(n, m)} (a_k + b_k)X^k \]
        \end{indt}
        
        %\vspace{6pt}
        
        \begin{indt}{\subsection{Produit}}
            Soit $\lambda \in K$.
            
            On définit $\lambda P$ par :
                \[\lambda P = \sum_{k = 0}^n \lambda a_k X^k  \]
            
            On définit le produit $PQ$ par :
                \[ PQ = \sum_{k = 0}^{n + m} \lr{\sum_{i = 0}^k a_i b_{k - i}} X^k = \sum_{k = 0}^{n + m} \lr{\sum_{i + j = k} a_i b_j} X^k \]
            
            On peut aussi écrire le produit comme suit :
                \[ PQ = \sum_{i = 0}^n \sum_{j = 0}^m a_i b_j X^{i + j} = \sum_{\substack{0 \le i \le n \\ 0 \le j \le m}} a_i b_j X^{i + j} \]
        \end{indt}
        
        \vspace{12pt}
        
        \begin{indt}{\subsection{Degré d'une sommme, produit}}
            Soient $P, Q \in K[X]$. On a :
            
            \vspace{6pt}
            
            \begin{indt}{$\bullet$ $\deg(P + Q) \le \max(\deg(P), \deg(Q))$, et plus précisément :}
                $-$ Si $\deg(P) \neq \deg(Q)$, alors $\deg(P + Q) = \max(\deg(P), \deg(Q))$
                
                $-$ Sinon, $\deg(P) = \deg(Q)$, et $\deg(P + Q) = \deg(P) \ssi \dom(P) \neq - \dom(Q)$
            \end{indt}
            
            \vspace{6pt}
            
            $\bullet$ $\deg(PQ) = \deg(P) + \deg(Q)$.
            
            \vspace{12pt}
            
            Soient $P \in K[X] \setminus \{0\},\ n \in \N$. On a $\deg(P^n) = n\deg(P)$
            
            \vspace{12pt}
            
            Soient $n \in \N^*$ et $(P_k)_{k \in \nset{1}{n}} \subset K[X]$. Alors :
                \[ \deg\lr{\sum_{k = 0}^n P_k} \le \max\set{\deg(P_k)\ |\ k \in \nset 1 n} \]
        \end{indt}
        
        \vspace{12pt}
        
        \begin{indt}{\subsection{Anneau $K[X]$}}
            $(K[X], +, \cdot)$ est un anneau abélien intègre.
            
            $K[X]^* = \set{P \in K[X]\ |\ \deg(P) = 0} = K^*$ (polynômes constants non nuls)
        \end{indt}
        
    \end{indt}
    
    \vspace{12pt}
    
    \begin{indt}{\section{Composée}}
        
        \begin{indt}{\subsection{Définition}}
            Soient $P, Q \in K[X]$, avec $\displaystyle P = \sum_k a_k X^k$
            
            On définit le polynôme composé $P \circ Q$ par
                \[ P \circ Q = P(Q(X)) = \sum_k a_k Q^k \]
        \end{indt}
        
        \vspace{6pt}
        
        \begin{indt}{\subsection{Propriétés}}
            Soient $A, B, R \in K[X]$. Alors :
                \[ (A \circ R) \cdot (B \circ R) = (AB) \circ R \]
                \[ A \circ R + B \circ R = (A + B) \circ R \]
            
            \vspace{6pt}
            
            Si $A \neq 0$ ou $B \neq 0$, alors
                \[ \deg(A \circ B) = \deg(A) \deg(B) \]
        \end{indt}
        
    \end{indt}
    
    \vspace{12pt}
    
    \begin{indt}{\section{Division euclidienne}}
        
        \begin{indt}{\subsection{Divisibilité}}
            Soient $A, B \in K[X]$.
                \[ A|B \ssi \exists D \in K[X]\ |\ B = AD \]
        \end{indt}
        
        \vspace{6pt}
        
        \begin{indt}{\subsection{Définition (association)}}
            Soient $A, B \in K[X]$.
            On note $A$ et $B$ sont \textit{associés} par $A \sim B$, et 
                \[ A \sim B \ssi \exists \lambda \in K^* \ |\ A = \lambda B \]
        \end{indt}
        
        \vspace{6pt}
        
        \begin{indt}{\subsection{Propriétés}}
            Soient $A, B \in K[X]$. On a :
                \[ \eqsys{A|B}{B \neq 0} \Rightarrow \deg(A) \le \deg(B) \]
                \[ A \sim B \ssi \eqsys{A|B}{B|A} \]
        \end{indt}
        
        \vspace{6pt}
        
        \begin{indt}{\subsection{Division euclidienne}}
            $\forall (A, B) \in K[X] \times (K[X] \setminus \set 0),\ \exists! (Q, R) \in (K[X])^2\ |\ \eqsys{A = BQ + R}{\deg(R) < \deg(B)}$
        \end{indt}
        
        \vspace{6pt}
        
        \begin{indt}{\subsection{Propriété}}
            $\forall (a, P) \in K \times K[X],\ \exists (b, Q) \in K \times K[X]\ |\ P = (X - a)Q + b$, et on a $b = P(a)$.
        \end{indt}
        
    \end{indt}
    
    \vspace{12pt}
    
    \begin{indt}{\section{Polynôme dérivé}}
        
        \begin{indt}{\subsection{Définition}}
            Soit $\displaystyle P = \sum_{k = 0}^n a_k X^k$.
            
            \vspace{6pt}
            
            On définit le polynôme dérivé de $P$ par :
                \[ P' = \sum_{k = 1}^n k a_k X^{k - 1} = \sum_{k = 0}^{n - 1} (k + 1) a_{k + 1} X^k \]
            
            Et par récurrence, le polynôme dérivé d'ordre $n \in \N^*$ est :
                \[ P^{(n)} = \lr{P^{(n - 1)}}' \]
            avec $P^{(0)} = P$.
        \end{indt}
        
        \vspace{6pt}
        
        \begin{indt}{\subsection{Propriétés}}
            Soient $P, Q \in K[X]$, et $\lambda, \mu \in K$. On a :
                \[ (\lambda P + \mu Q)' = \lambda P' + \mu Q' \]
                \[ (PQ)' = P'Q + Q'P \]
            
            \vspace{6pt}
            
            $\forall n, p \in \N$, on a :
                \[
                (X^n)^{(p)} =
                \left\{ \!\!
                \begin{array}{ll}
                    \dfrac{n!}{(n - p)!} X^{n - p}
                    & \text{si}\ n \ge p
                    \\\\
                    0
                    & \text{sinon.}
                \end{array}
                \right.
                \]
            
            \vspace{6pt}
            
            Soient $P, Q \in K[X]$. On a :
                \[\begin{array}{l}
                    P' = 0 \ssi \deg(P) \le 0
                    \vspace{6pt}
                    \\
                    \deg(P) > 0 \Rightarrow \eqsys{\deg(P') = \deg(P) - 1}{\dom(P') = \deg(P) \cdot \dom(P)}
                \end{array}\]
            
            Si $P \neq 0$, alors :
                \[\begin{array}{l}
                    \forall n \le \deg(P),\ \deg(P^{(n)}) = \deg(P) - n
                    \vspace{6pt}
                    \\
                    P^{(\deg(P))} = (\deg(P))! \cdot \dom(P)
                    \vspace{6pt}
                    \\
                    P^{(\deg(P) + 1)} = 0
                \end{array}\]
        \end{indt}
        
        \vspace{12pt}
        
        \begin{indt}{\subsection{Formule de Leibniz}}
            Soient $P, Q \in K[X]$. On a :
                \[ (PQ)^{(n)} = \sum_{k = 0}^n \binom{n}{k} P^{(k)} Q^{(n - k)} \]
        \end{indt}
        
    \end{indt}
    
    \vspace{12pt}
    
    \begin{indt}{\section{Racines d'un polynôme}}
        
        \begin{indt}{\subsection{Fonction polynomiale, formule de Taylor}}
            \begin{indt}{\subsubsection{Défintion (fonction polynomiale)}}
                Soit $\displaystyle P = \sum_{k = 0}^n a_k X^k \in K[X]$.
                
                \vspace{6pt}
                
                La fonction polynomiale associée à $P$ est la fonction
                    \[
                        \begin{array}{rcccc}
                            \widetilde P
                            & :
                            & K
                            & \longrightarrow
                            & K
                            \\
                            && x
                            & \longmapsto
                            & \displaystyle \sum_{k = 0}^n a_k x^k
                        \end{array}
                    \]
                
                Pour $\alpha \in K$, on note
                    \[ P(\alpha) = \widetilde P(\alpha) = \sum_{k = 0}^n a_k \alpha^k \]
            \end{indt}
            
            \vspace{6pt}
            
            \begin{indt}{\subsubsection{Formule de Taylor}}
                Soient $P \in K[X]$, et $a \in K$. On a :
                    \[ P(X) = \sum_{k = 0}^{+\infty} \dfrac{P^{(k)}(a)}{k!}(X - a)^k \]
                    \[ P(X + a) = \sum_{k = 0}^{+\infty} \dfrac{P^{(k)}(a)}{k!} X^k \]
            \end{indt}
            
            \vspace{6pt}
            
            \begin{indt}{\subsubsection{Propriétés}}
                Soient $P \in K[X]$, et $a \in K$. On a :
                    \[ \forall m \ge \deg(P),\ P(X) = \sum_{k = 0}^m \dfrac{P^{(k)}(a)}{k!} (X - a)^k \]
                
                \vspace{6pt}
                
                Soient $n \in \N,\ a \in K$, et $(a_k)_{k \in \nset 0 n} \subset K$. Alors
                    \[ \sum_{k = 0}^n a_k(X - a)^k = 0\ \Rightarrow\ \forall k \in \nset 0 n,\ a_k = 0 \]
                
                \vspace{6pt}
                
                Soient $(P, a, n) \in K[X] \times K \times \N^*$. On a : %$P \in K[X],\ n \in \N^*,\ a \in K$. On a :
                    \[\begin{array}{rcl}
                        P(X)
                        &=& \displaystyle \sum_{k = 0}^{+\infty} \dfrac{P^{(k)}(a)}{k!} (X - a)^k
                        \\
                        &=& \displaystyle
                        (X - a)^n \underbrace{\sum_{k = n}^{+\infty} \dfrac{P^{(k)}(a)}{k!} (X - a)^{k - n}}_{Q}
                        +
                        \underbrace{\sum_{k = 0}^{n - 1} \dfrac{P^{(k)}(a)}{k!} (X - a)^k}_{R}
                    \end{array}\]
                avec $Q$ et $R$ respectivement le quotient et le reste de la division euclidienne de $P$ par $(X - a)^n$
            \end{indt}
        \end{indt}
        
        \vspace{12pt}
        
        \begin{indt}{\subsection{Racines simples}}
            \begin{indt}{\subsubsection{Définition}}
                Soient $P \in K[X]$, et $a \in K$.
                
                $a$ est \textit{racine} de $P$ $\ssi P(a) = 0$.
            \end{indt}
            
            \vspace{6pt}
            
            \begin{indt}{\subsubsection{Propriétés}}
                Soient $A, B \in K[X],\ a \in K$. Alors
                    \[ \eqsys{A(a) = 0}{A|B} \Rightarrow B(a) = 0 \]
                
                \vspace{6pt}
                
                Soient $P \in K[X],\ \alpha \in K$. On a :
                    \[ P(\alpha) = 0 \ssi X - \alpha | P \]
                
                \vspace{6pt}
                
                Soient
                $
                    \left|
                    \begin{array}{l}
                        P \in K[X]
                        \\
                        n \in \N^*
                        \\
                        (\alpha_k)_{k \in \nset 1 n} \subset K\ |\ \forall (i, j) \in \nset{1}{n}^2,\ i \neq j \Rightarrow \alpha_i \neq \alpha_j
                    \end{array}
                    \right.
                $
                Alors :
                    \[ \forall k \in \nset 1 n,\ P(\alpha_k) = 0 \ssi \prod_{k = 1}^n (X - \alpha_k) \ |\ P \]
                
                \vspace{6pt}
                
                Soient
                $
                    \left|
                    \begin{array}{l}
                        P \in K[X]
                        \\
                        n = \deg(P)
                        \\
                        (\alpha_k)_{k \in \nset 1 n} \subset K\ |\ \forall k \in \nset 1 n,\ P(\alpha_k) = 0
                    \end{array}
                    \right.
                $
                Alors :
                    \[ \exists \lambda \in K^*\ |\ P = \lambda \prod_{k = 1}^n (X - \alpha_k) \]
            \end{indt}
        \end{indt}
        
        \vspace{12pt}
        
        \begin{indt}{\subsection{Racines multiples}}
            \begin{indt}{\subsubsection{Propriétés}}
                Soient $P \in K[X],\ a \in K,\ n \in \N^*$. Alors :
                    \[ (X - a)^n | P \ssi \forall k \in \nset{0}{n - 1},\ P^{(k)}(a) = 0 \]
                
                \vspace{6pt}
                
                Soient $P \in K[X],\ a \in K,\ n \in \N$. Alors :
                    \[\begin{array}{rcl}
                        \exists Q \in K[X]\ \eqsys{P = (X - a)^n Q}{Q(a) \neq 0}
                        &\ssi&
                        \eqsys{(X - a)^n | P}{(X - a)^{n + 1} \nmid P}
                        \vspace{6pt}
                        \\
                        &\ssi&
                        \eqsys{\forall k \in \nset{0}{n - 1},\ P^{(k)}(a) = 0}{P^{(n)}(a) \neq 0}
                    \end{array}\]
            \end{indt}
            
            \vspace{6pt}
            
            \begin{indt}{\subsubsection{Définition (Ordre de multiplicité)}}
                Soient $P \in K[X],\ a \in K,\ n \in \N^*$.
                
                $a$ est une racine d'ordre (exactement) $n$ de $P$ $\ssi \eqsys{(X - a)^n | P}{(X - a)^{n + 1} \nmid P}$
                
                L'entier $n$ est alors appelé \textit{l'ordre de multiplicité} de la racine $a$.
                
                \vspace{12pt}
                
                $P$ admet $a$ comme racine d'ordre au moins $n$ $\ssi (X - a)^n | P$.
            \end{indt}
            
            \vspace{12pt}
            
            \begin{indt}{\subsubsection{Définition (racines simples, multiples)}}
                Soit $P \in K[X]$
                
                Une racine simple de $P$ est une racine d'ordre exactement 1.
                
                Une racine multiple de $P$ est une racine d'ordre au moins 2.
            \end{indt}
            
            \vspace{12pt}
            
            \begin{indt}{\subsubsection{Propriétés}}
                Soient $P \in K[X],\ a \in K,\ n \in \N^*$. Si $a$ est racine d'ordre (exactement) $n$ de $P$, alors $a$ est racine d'ordre (exactement) $n - 1$ de $P'$.
                
                \vspace{6pt}
                
                Soient
                $
                    \left|
                    \begin{array}{l}
                        P \in K[X]
                        \\
                        n \in \N^*
                        \\
                        (r_k)_{k \in \nset 1 n} \subset N^*
                        \\
                        (a_k)_{k \in \nset 1 n} \subset K\ |\ \forall (i, j) \in \nset{1}{n}^2,\ i \neq j \Rightarrow a_i \neq a_j
                    \end{array}
                    \right.
                $
                Alors :
                    \[ \forall k \in \nset 1 n,\ (X - a_k)^{r_k} | P \ssi \prod_{k = 1}^n (X - a_k)^{r_k} | P \]
                
                De plus, si $\displaystyle \deg(P) = \sum_{k = 1}^n r_k$, on a :
                    \[
                        \forall k \in \nset 1 n,\
                        \eqsys{(X - a_k)^{r_k} | P}{(X - a)^{r_k + 1} \nmid P}
                        \ssi
                        \exists \lambda \in K^*\ |\ P = \lambda \prod_{k = 1}^n (X - a_k)^{r_k}
                    \]
            \end{indt}
        \end{indt}
        
        \vspace{12pt}
        
        \begin{indt}{\subsection{Polynômes scindés}}
            \begin{indt}{\subsubsection{Définition (polynôme scindé)}}
                Soit $P \in K[X]$.
                
                $P$ est \textit{scindé} sur $K$
                $
                    \ssi \eqsys
                        {\deg(P) > 0}
                        {\exists n \in \N^*,\ \exists (x_k)_{k \in \nset 1 n} \subset K,\ \exists \lambda \in K^*\ |\ P = \displaystyle \lambda \prod_{k = 1}^n (X - x_k)}
                $
            \end{indt}
            
            \vspace{12pt}
            
            \begin{indt}{\subsubsection{Propriétés}}
                Soient $A, B \in K[X] \setminus \set 0$, avec $A$ scindé.
                
                Soient
                $
                    \left|
                    \begin{array}{l}
                        n \in \N^*,
                        \\
                        (r_k)_{k \in \nset 1 n} \subset N^*
                        \\
                        (a_k)_{k \in \nset 1 n} \subset K\ |\ \forall (i, j) \in \nset 1 n ^2,\ i \neq j \Rightarrow a_i \neq a_j
                    \end{array}
                    \right.
                $
                
                \vspace{6pt}
                
                tels que
                $
                    \displaystyle
                    \exists \lambda \in K^*\ |\ A = \lambda \prod_{k = 1}^n (X - a_k)^{r_k}
                $
                (\textit{i.e} les $a_k$ sont les racines d'ordre $r_k$ de A). Alors :
                    \[ A|B \ssi \forall k \in \nset 1 n,\ (X - a_k)^{r_k} | B \]
            \end{indt}
        \end{indt}
        
    \end{indt}
    
    \vspace{12pt}
    
    \begin{indt}{\section{\'Etude de $\C[X]$ et $\R[X]$}}
        
        \begin{indt}{\subsection{Définition (polynôme conjugé)}}
            Soit $\displaystyle P = \sum_{k = 0}^n a_k X^k \in \C[X]$.
            
            On définit $\cj P$ le polynôme conjugé de $P$ par :
                \[ \cj P = \sum_{k = 0}^n \cj{a_k} X^k \]
        \end{indt}
        
        \vspace{6pt}
        
        \begin{indt}{\subsection{Propriétés}}
            Soient $P \in \C[X],\ z \in \C,\ k \in \N$. Alors :
                \[\begin{array}{l}
                    \cj{P(z)} = \cj{P}(\cj z)
                    \vspace{6pt}
                    \\
                    \cj{P^{(k)}} = \cj{P}^{(k)}
                    \vspace{6pt}
                    \\
                    P \in \R[X] \ssi P = \cj P
                    \vspace{6pt}
                    \\
                    \eqsys{(X - z)^k | P}{(X - z)^{k + 1} \nmid P} \ssi \eqsys{(X - \cj z)^k | \cj P}{(X - \cj z)^{k + 1} \nmid \cj P}
                \end{array}\]
            
            \vspace{6pt}
            
            Soit $z \in \C$. Alors :
                \[ (X - z)(X - \cj z) = X^2 + 2\Re(z)X + \abs{z}^2 \]
            
            \vspace{6pt}
            
            Soit $P \in \R[X]$. Alors on peut décomposer $P$ dans $\R[X]$ :
                
                $
                    \displaystyle
                    \left|
                    \begin{array}{l}
                        \exists \lambda \in \R^*
                        \vspace{3pt}
                        \\
                        \exists n, m \in \N^*
                        \vspace{3pt}
                        \\
                        \exists (a_k) \in \R^{\nset 1 n}
                        \vspace{3pt}
                        \\
                        \exists (b_k),\ (c_k) \in \R^{\nset 1 m}
                        \vspace{3pt}
                        \\
                        \exists (r_k)_{k \in \nset 1 n},\ (s_k)_{k \in \nset 1 m} \subset \N^*
                    \end{array}
                    \right.
                    |\ P = \lambda \lr{\prod_{k = 1}^n (X - a_k)^{r_k}} \lr{\prod_{k = 1}^m (X^2 + b_k X + c_k)^{s_k}}
                $
                
                \vspace{6pt}
                
                où $\forall k \in \nset 1 m,\ \nexists x \in \R\ |\ x^2 + b_k x + c_k = 0$.
        \end{indt}
        
    \end{indt}





    
    
    
\end{document}
%--------------------------------------------End
