\documentclass[a4paper, 12pt, twoside]{article}


%------------------------------------------------------------------------
%
% Author                :   Lasercata
% Last modification     :   2022.07.03
%
%------------------------------------------------------------------------


%------ini
\usepackage[utf8]{inputenc}
\usepackage[T1]{fontenc}
%\usepackage[french]{babel}
%\usepackage[english]{babel}


%------geometry
\usepackage[textheight=700pt, textwidth=500pt]{geometry}


%------color
\usepackage{xcolor}
\definecolor{ff4500}{HTML}{ff4500}
\definecolor{00f}{HTML}{0000ff}
\definecolor{0ff}{HTML}{00ffff}
\definecolor{656565}{HTML}{656565}

\renewcommand{\emph}{\textcolor{ff4500}}
\renewcommand{\em}{\color{ff4500}}

\newcommand{\strong}[1]{\textcolor{ff4500}{\bf #1}}
\newcommand{\st}{\color{ff4500}\bf}


%------Code highlighting
%---listings
\usepackage{listings}

\definecolor{cbg}{HTML}{272822}
\definecolor{cfg}{HTML}{ececec}
\definecolor{ccomment}{HTML}{686c58}
\definecolor{ckw}{HTML}{f92672}
\definecolor{cstring}{HTML}{e6db72}
\definecolor{cstringlight}{HTML}{98980f}
\definecolor{lightwhite}{HTML}{fafafa}

\lstdefinestyle{DarkCodeStyle}{
    backgroundcolor=\color{cbg},
    commentstyle=\itshape\color{ccomment},
    keywordstyle=\color{ckw},
    numberstyle=\tiny\color{cbg},
    stringstyle=\color{cstring},
    basicstyle=\ttfamily\footnotesize\color{cfg},
    breakatwhitespace=false,
    breaklines=true,
    captionpos=b,
    keepspaces=true,
    numbers=left,
    numbersep=5pt,
    showspaces=false,
    showstringspaces=false,
    showtabs=false,
    tabsize=4,
    xleftmargin=\leftskip
}

\lstdefinestyle{LightCodeStyle}{
    backgroundcolor=\color{lightwhite},
    commentstyle=\itshape\color{ccomment},
    keywordstyle=\color{ckw},
    numberstyle=\tiny\color{cbg},
    stringstyle=\color{cstringlight},
    basicstyle=\ttfamily\footnotesize\color{cbg},
    breakatwhitespace=false,
    breaklines=true,
    captionpos=b,
    keepspaces=true,
    numbers=left,
    numbersep=10pt,
    showspaces=false,
    showstringspaces=false,
    showtabs=false,
    tabsize=4,
    frame=L,
    xleftmargin=\leftskip
}

%\lstset{style=DarkCodeStyle}
\lstset{style=LightCodeStyle}
%Usage : \begin{lstlisting}[language=Caml] ... \end{lstlisting}

%---tcolorbox
\usepackage[many]{tcolorbox}
\DeclareTColorBox{pseudocode}{O{black}O{lightwhite}}{
    breakable,
    outer arc=0pt,
    arc=0pt,
    top=0pt,
    toprule=-.5pt,
    right=0pt,
    rightrule=-.5pt,
    bottom=0pt,
    bottomrule=-.5pt,
    colframe=#1,
    colback=#2,
    enlarge left by=10pt,
    width=\linewidth-\leftskip-10pt,
}


%-------make the table of content clickable
\usepackage{hyperref}
\hypersetup{
    colorlinks,
    citecolor=black,
    filecolor=black,
    linkcolor=black,
    urlcolor=black
}
%Uncomment this and comment above for dark mode
% \hypersetup{
%     colorlinks,
%     citecolor=white,
%     filecolor=white,
%     linkcolor=white,
%     urlcolor=white
% }


%------pictures
\usepackage{graphicx}
%\usepackage{wrapfig}

\usepackage{tikz}
%\usetikzlibrary{babel}             %Uncomment this to use circuitikz
%\usetikzlibrary{shapes.geometric}  % To draw triangles in trees
%\usepackage{circuitikz}            %Electrical circuits drawing


%------tabular
%\usepackage{color}
%\usepackage{colortbl}
%\usepackage{multirow}


%------Physics
%---Packages
%\usepackage[version=4]{mhchem} %$\ce{NO4^2-}$

%---Commands
\newcommand{\link}[2]{\mathrm{#1} \! - \! \mathrm{#2}}
\newcommand{\pt}[1]{\cdot 10^{#1}} % Power of ten
\newcommand{\dt}[2][t]{\dfrac{\mathrm d #2}{\mathrm d #1}} % Derivative


%------math
%---Packages
%\usepackage{textcomp}
%\usepackage{amsmath}
\usepackage{amssymb}
\usepackage{mathtools} % For abs
\usepackage{stmaryrd} %for \llbracket and \rrbracket
\usepackage{mathrsfs} %for \mathscr{x} (different from \mathcal{x})

%---Commands
%-Sets
\newcommand{\N}{\mathbb{N}} %set N
\newcommand{\Z}{\mathbb{Z}} %set Z
\newcommand{\Q}{\mathbb{Q}} %set Q
\newcommand{\R}{\mathbb{R}} %set R
\newcommand{\C}{\mathbb{C}} %set C
\newcommand{\U}{\mathbb{U}} %set U
\newcommand{\seg}[2]{\left[ #1\ ;\ #2 \right]}
\newcommand{\nset}[2]{\left\llbracket #1\ ;\ #2 \right\rrbracket}

%-Exponantial / complexs
\newcommand{\e}{\mathrm{e}}
\newcommand{\cj}[1]{\overline{#1}} %overline for the conjugate.

%-Vectors
\newcommand{\vect}{\overrightarrow}
\newcommand{\veco}[3]{\displaystyle \vect{#1}\binom{#2}{#3}} %vector + coord

%-Limits
\newcommand{\lm}[2][{}]{\lim\limits_{\substack{#2 \\ #1}}} %$\lm{x \to a} f$ or $\lm[x < a]{x \to a} f$
\newcommand{\Lm}[3][{}]{\lm[#1]{#2} \left( #3 \right)} %$\Lm{x \to a}{f}$ or $\Lm[x < a]{x \to a}{f}$
\newcommand{\tendsto}[1]{\xrightarrow[#1]{}}

%-Integral
\newcommand{\dint}[4][x]{\displaystyle \int_{#2}^{#3} #4 \mathrm{d} #1} %$\dint{a}{b}{f(x)}$ or $\dint[t]{a}{b}{f(t)}$

%-left right
\newcommand{\lr}[1]{\left( #1 \right)}
\newcommand{\lrb}[1]{\left[ #1 \right]}
\newcommand{\lrbb}[1]{\left\llbracket #1 \right\rrbracket}
\newcommand{\set}[1]{\left\{ #1 \right\}}
\newcommand{\abs}[1]{\left\lvert #1 \right\rvert}
\newcommand{\ceil}[1]{\left\lceil #1 \right\rceil}
\newcommand{\floor}[1]{\left\lfloor #1 \right\rfloor}
\newcommand{\lrangle}[1]{\left\langle #1 \right\rangle}

%-Others
\newcommand{\para}{\ /\!/\ } %//
\newcommand{\ssi}{\ \Leftrightarrow \ }
\newcommand{\eqsys}[2]{\begin{cases} #1 \\ #2 \end{cases}}

\newcommand{\med}[2]{\mathrm{med} \left[ #1\ ;\ #2 \right]}  %$\med{A}{B} -> med[A ; B]$
\newcommand{\Circ}[2]{\mathscr{C}_{#1, #2}}

\renewcommand{\le}{\leqslant}
\renewcommand{\ge}{\geqslant}

\newcommand{\oboxed}[1]{\textcolor{ff4500}{\boxed{\textcolor{black}{#1}}}} %orange boxed

\newcommand{\ep}{\varepsilon}


%------commands
%---to quote french text
\newcommand{\simplecit}[1]{\guillemotleft$\;$#1$\;$\guillemotright}
\newcommand{\cit}[1]{\simplecit{\textcolor{656565}{#1}}}
\newcommand{\quo}[1]{\cit{\it #1}}

%---to indent
\newcommand{\ind}[1][20pt]{\advance\leftskip + #1}
\newcommand{\deind}[1][20pt]{\advance\leftskip - #1}

%---to indent a text
\newcommand{\indented}[2][20pt]{\par \ind[#1] #2 \par \deind[#1]}
\newenvironment{indt}[2][20pt]{#2 \par \ind[#1]}{\par \deind} %Titled indented env

%---title
\newcommand{\thetitle}[2]{\begin{center}\textbf{{\LARGE \underline{\emph{#1} :}} {\Large #2}}\end{center}}


%------Sections
% To change section numbering :
% \renewcommand\thesection{\Roman{section}}
% \renewcommand\thesubsection{\arabic{subsection}}
% \renewcommand\thesubsubsection{\textit \alph{subsubsection}}

% To start numbering from 0
% \setcounter{section}{-1}


%------page style
\usepackage{fancyhdr}
\usepackage{lastpage}

\setlength{\headheight}{18pt}
\setlength{\footskip}{50pt}

\pagestyle{fancy}
\fancyhf{}
\fancyhead[LE, RO]{\textit{\textcolor{black}{\today}}}
\fancyhead[RE, LO]{\large{\textsl{\emph{\texttt{\jobname}}}}}

\fancyfoot[RO, LE]{\textit{\texttt{\textcolor{black}{Page \thepage /}\pageref{LastPage}}}} %Change 'black' to 'white' for dark mode
\fancyfoot[LO, RE]{\includegraphics[scale=0.12]{/home/lasercata/Pictures/1.images_profil/logo/mieux/lasercata_logo_fly_fond_blanc.png}}

% For dark mode :
%/home/lasercata/Pictures/1.images_profil/logo/mieux/lasercata_logo_fly.png


%------init lengths
\setlength{\parindent}{0pt} %To avoid using \noindent everywhere.
\setlength{\parskip}{3pt}


%---------------------------------Begin Document
\begin{document}
    
    %For dark mode :
    % \pagecolor{black}
    % \color{white}
    
    \thetitle{Maths}{Analyse}
    
    \tableofcontents
    \newpage
    
    
    \begin{indt}{\section{Limites}}
        
        \begin{indt}{\subsection{Suites}}
            \begin{indt}{\subsubsection{Définition (\textit{suite convergente})}}
                Soit $(u_n)_{n \in \N} \subset \R$ une suite.

                On dit que la suite $(u_n)$ est \textit{convergente} (ou admet une limite finie), si :
                \[
                    \exists \ell \in \R\ |\ \forall \varepsilon \in \R^*_+,\ \exists n_0 \in \N\ |\ \forall n \in \N,\ n \ge n_0 \Rightarrow \abs{u_n - \ell} \le \varepsilon
                \]

                Dans ce cas, $\ell$ est unique, et est la limite de la suite $(u_n)$. On dit que la suite $(u_n)$ \textit{converge} vers $\ell$, et on note :
                \[
                    u_n \tendsto{n \to +\infty} \ell
                \]
            \end{indt}

            \vspace{12pt}
            
            \begin{indt}{\subsubsection{Définition (\textit{suite divergente})}}
                Soit $(u_n)_{n \in \N} \subset \R$ une suite.

                La suite $(u_n)$ est \textit{divergente} si elle ne converge pas, \textit{i.e} si :
                \[
                    \forall \ell \in \R,\ \exists \ep \in \R^*_+\ |\ \forall n_0 \in \N,\ \exists n \in \N, n \ge n_0\ |\ \abs{u_n - \ell} \ge \ep
                \]
            \end{indt}

            \vspace{12pt}
            
            \begin{indt}{\subsubsection{Divergence vers l'infini}}
                Soit $(u_n)_{n \in \N} \subset \R$.

                $\bullet$ La suite $(u_n)$ tend vers $+\infty$, noté $u_n \tendsto{n \to +\infty} +\infty$, si :
                \[
                    \forall A \in \R,\ \exists n_0 \in \N\ |\ \forall n \in \N,\ n \ge n_0 \Rightarrow u_n \ge A
                \]

                $\bullet$ La suite $(u_n)$ tend vers $-\infty$, noté $u_n \tendsto{n \to +\infty} -\infty$, si :
                \[
                    \forall A \in \R,\ \exists n_0 \in \N\ |\ \forall n \in \N,\ n \ge n_0 \Rightarrow u_n \le A
                \]
            \end{indt}

            \vspace{12pt}
            
            \begin{indt}{\subsubsection{Définition (\textit{suites adjacentes})}}
                Soient $(u_n)_{n \in \N}, (v_n)_{n \in \N} \subset \R$ deux suites.

                \begin{indt}{Les suites $(u_n)$ et $(v_n)$ sont \textit{adjacentes} si :}
                    (1) $v_n - u_n \tendsto{n \to +\infty} 0$ ;

                    (2) $(u_n)$ et $(v_n)$ sont monotones de sens opposé.
                \end{indt}
            \end{indt}

            \vspace{12pt}
            
            \begin{indt}{\subsubsection{Propriétés}}
                Soient $(u_n)_{n \in \N}, (v_n)_{n \in \N} \subset \R$ deux suites adjacentes.

                Alors ces suites sont convergentes de même limite $\ell$, et si $(u_n)$ est croissante, on a :
                \[
                    \forall n \in \N,\ u_n \le \ell \le v_n
                \]
            \end{indt}

            \vspace{12pt}
            
            \begin{indt}{\subsubsection{Théorème des segments emboités}}
                Soient $(u_n)_{n \in \N}, (v_n)_{n \in \N} \subset \R$ deux suites adjacentes, avec $(u_n)$ croissante.

                Soit $\forall n \in \N,\ I_n = [u_n ; v_n]$

                $(I_n)_{n \in \N}$ est une suite de segments de $\R$ dont la longueur tends vers $0$ et telle que $\forall n \in \N,\ I_{n + 1} \subset I_n$.

                \vspace{12pt}
                
                Alors
                \[
                    \displaystyle \exists \ell \in \R\ |\ \bigcup_{n \in \N} I_n = \set \ell
                \]

                et :
                \[
                    \forall (w_n)_{n \in \N}\ |\ \forall n \in \N,\ w_n \in I_n,\ w_n \tendsto{n \to +\infty} \ell
                \]
            \end{indt}

            \vspace{12pt}
            
            \begin{indt}{\subsubsection{Théorème de Bolzano-Weierstrass}}
                Toute suite réelle bornée admet une sous-suite convergente.
            \end{indt}
        \end{indt}

        \vspace{12pt}
        
        \begin{indt}{\subsection{Fonctions}}
            \begin{indt}{\subsubsection{Définition (\textit{voisinage})}}
                Soit $a \in \R$.

                $\bullet$ Un voisinage de $a$ est un ensemble qui contient l'intervalle $]a - \ep\ ;\ a + \ep[$, où $\ep \in \R_+^*$.

                $\bullet$ Un voisinage de $a^+$ (ou voisinage à droite de $a$) est un ensemble contenant l'intervalle $[a\ ;\ a + \ep[$, où $\ep \in \R_+^*$.

                $\bullet$ Un voisinage de $a^-$ (ou voisinage à gauche de $a$) est un ensemble contenant l'intervalle $]a - \ep\ ;\ a]$, où $\ep \in \R_+^*$.

                $\bullet$ Un voisinage de $+\infty$ est un ensemble contenant l'intervalle $]h\ ;\ +\infty[$, où $h \in \R$.

                $\bullet$ Un voisinage de $-\infty$ est un ensemble contenant l'intervalle $]-\infty\ ;\ h[$, où $h \in \R$.
            \end{indt}

            \vspace{12pt}
            
            \begin{indt}{\subsubsection{Définition (\textit{fonction définie au voisinage d'un point})}}
                Une fonction est définie au voisinage d'un point $a \in \cj \R$ si elle est définie sur un voisinage de $a$, sauf peut-être en $a$.
            \end{indt}

            \vspace{12pt}
            
            \begin{indt}{\subsubsection{Limite en un point réel}}
                \label{1.2.3}

                Soit $a \in \R$, et $f$ une fonction définie sur $\mathcal D_f$, un voisinage de $a$, et à valeurs réelles.

                \vspace{6pt}
                
                $\bullet$ La fonction $f$ admet une limite finie $\ell \in \R$ en $a$ si :
                \[
                    \exists \ell \in \R\ |\ \forall \ep \in \R_+^*,\ \exists \delta \in \R_+^*\ |\ \forall x \in \mathcal D_f,\ \abs{x - a} \le \delta \Rightarrow \abs{f(x) - \ell} \le \ep
                \]
                On note alors $f(x) \tendsto{x \to a} \ell$, ou $\displaystyle \lim_{x \to a}f(x) = \ell$, ou $\displaystyle \lim_a f = \ell$.

                \vspace{6pt}
                
                $\bullet$ La fonction $f$ admet $+\infty$ comme limite en $a$ si :
                \[
                    \forall A \in \R, \exists \delta \in \R_+^*\ |\ \forall x \in \mathcal D_f,\ \abs{x - a} \le \delta \Rightarrow f(x) \ge A
                \]
                On note alors $f(x) \tendsto{x \to a} +\infty$, ou $\displaystyle \lim_a f = +\infty$.

                \vspace{6pt}
                
                $\bullet$ La fonction $f$ admet $-\infty$ comme limite en $a$ si :
                \[
                    \forall A \in \R, \exists \delta \in \R_+^*\ |\ \forall x \in \mathcal D_f,\ \abs{x - a} \le \delta \Rightarrow f(x) \le A
                \]
                On note alors $f(x) \tendsto{x \to a} -\infty$, ou $\displaystyle \lim_a f = -\infty$.
            \end{indt}

            \vspace{12pt}
            
            \begin{indt}{\subsubsection{Limite en $+\infty$}}
                Soit $f$ une fonction définie sur $\mathcal D_f$, un voisinage de $+\infty$, et à valeurs dans $\R$.

                \vspace{6pt}
                
                $\bullet$ La fonction $f$ admet une limite finie $\ell \in \R$ en $+\infty$ si :
                \[
                    \exists \ell \in \R\ |\ \forall \ep \in \R_+^*,\ \exists B \in \R\ |\ \forall x \in \mathcal D_f,\ x \ge B \Rightarrow \abs{f(x) - \ell} \le \ep
                \]
                On note alors $f(x) \tendsto{x \to +\infty} \ell$, ou $\displaystyle \lim_{+\infty} f = \ell$.

                \vspace{6pt}
                
                $\bullet$ La fonction $f$ admet $+\infty$ comme limite en $+\infty$ si :
                \[
                    \forall A \in \R,\ \exists B \in \R\ |\ \forall x \in \mathcal D_f,\ x \ge B \Rightarrow f(x) \ge A
                \]
                On note alors $f(x) \tendsto{x \to +\infty} +\infty$, ou $\displaystyle \lim_{+\infty} f = +\infty$.

                \vspace{6pt}
                
                $\bullet$ La fonction $f$ admet $-\infty$ comme limite en $+\infty$ si :
                \[
                    \forall A \in \R,\ \exists B \in \R\ |\ \forall x \in \mathcal D_f,\ x \ge B \Rightarrow f(x) \le A
                \]
                On note alors $f(x) \tendsto{x \to +\infty} -\infty$, ou $\displaystyle \lim_{+\infty} f = -\infty$.
            \end{indt}

            \vspace{12pt}
            
            \begin{indt}{\subsubsection{Limite en $-\infty$}}
                Soit $f$ une fonction définie sur $\mathcal D_f$, un voisinage de $-\infty$, et à valeurs dans $\R$.

                \vspace{6pt}
                
                $\bullet$ La fonction $f$ admet une limite finie $\ell \in \R$ en $-\infty$ si :
                \[
                    \exists \ell \in \R\ |\ \forall \ep \in \R_+^*,\ \exists B \in \R\ |\ \forall x \in \mathcal D_f,\ x \le B \Rightarrow \abs{f(x) - \ell} \le \ep
                \]
                On note alors $f(x) \tendsto{x \to -\infty} \ell$, ou $\displaystyle \lim_{-\infty} f = \ell$.

                \vspace{6pt}
                
                $\bullet$ La fonction $f$ admet $+\infty$ comme limite en $-\infty$ si :
                \[
                    \forall A \in \R,\ \exists B \in \R\ |\ \forall x \in \mathcal D_f,\ x \le B \Rightarrow f(x) \ge A
                \]
                On note alors $f(x) \tendsto{x \to -\infty} +\infty$, ou $\displaystyle \lim_{-\infty} f = +\infty$.

                \vspace{6pt}
                
                $\bullet$ La fonction $f$ admet $-\infty$ comme limite en $-\infty$ si :
                \[
                    \forall A \in \R,\ \exists B \in \R\ |\ \forall x \in \mathcal D_f,\ x \le B \Rightarrow f(x) \le A
                \]
                On note alors $f(x) \tendsto{x \to -\infty} -\infty$, ou $\displaystyle \lim_{-\infty} f = -\infty$.
            \end{indt}

            \vspace{12pt}
            
            \begin{indt}{\subsubsection{Limite à droite et à gauche}}
                \label{1.2.6}
                
                Soit $a \in \R$, et $f$ une fonction définie sur un voisinage $\mathcal D_f$ de $a$, et à valeurs dans $\R$.

                \vspace{6pt}
                
                $\bullet$ La fonction $f$ admet une limite finie $\ell \in \R$ à droite si :
                \[
                    \exists \ell \in \R\ |\
                    \forall \ep \in \R_+^*,\ \exists \delta \in \R_+^*\ |\
                    \forall x \in \mathcal D_f\, \cap\; ]\!-\infty\ ;\ a],\
                    \abs{x - a} \le \delta \Rightarrow \abs{f(x) - \ell} \le \ep
                \]
                On note alors : $f(x) \tendsto{x \to a^+} \ell$, ou $\displaystyle \lim_{a^+} f = \ell$.

                \vspace{6pt}
                
                Resp. pour la limite à gauche :
                \[
                    \exists \ell \in \R\ |\
                    \forall \ep \in \R_+^*,\ \exists \delta \in \R_+^*\ |\
                    \forall x \in \mathcal D_f \cap [a\ ; +\infty[,\
                    \abs{x - a} \le \delta \Rightarrow \abs{f(x) - \ell} \le \ep
                \]
                On note alors : $f(x) \tendsto{x \to a^-} \ell$, ou $\displaystyle \lim_{a^-} f = \ell$.
            \end{indt}

            \vspace{12pt}
            
            \begin{indt}{\subsubsection{Propriété}}
                Soit $f$ une fonction définie sur $] a - h\ ;\ a + h[ \setminus \set a$, où $h \in \R_+^*$.

                Alors :
                \[
                    \exists \ell \in \R\ |\ f(x) \tendsto{x \to a} \ell
                    \ssi
                    \exists \ell' \in \R\
                    \begin{array}{|l}
                        f(x) \tendsto{x \to a^+} \ell'
                        \\
                        f(x) \tendsto{x \to a^-} \ell'
                    \end{array}
                \]

                Et dans ce cas, $\ell = \ell'$.
            \end{indt}

            \vspace{12pt}
            
            \begin{indt}{\subsubsection{Propriétés locales}}
                Soit $a \in \R$, et $f$ une fonction définie sur un voisinage $\mathcal D_f$ de $a$, à valeurs réelles.

                \vspace{6pt}
                
                $\bullet$ Si $\exists \ell \in \R\ |\ f(x) \tendsto{x \to a} \ell$, alors $f$ est bornée au voisinage de $a$, \textit{i.e} pour $h \in \R_+^*$ :
                \[
                    \exists m, M \in \R\ |\
                    \forall x \in ]a - h\ ;\ a + h[,\
                    m \le f(x) \le M
                \]

                \vspace{6pt}
                
                $\bullet$ Si $\exists \ell \in \R_+^*\ |\ f(x) \tendsto{x \to a} \ell$, alors $f$ est minorée par un réel strictement positif au voisinage de $a$, \textit{i.e} pour $h \in \R_+^*$, on a :
                \[
                    \exists m \in \R_+^*\ |\ \forall x \in ]a - h\ ;\ a + h[,\ m \le f(x)
                \]
            \end{indt}

            \vspace{12pt}
            
            \begin{indt}{\subsubsection{Opérations algébriques sur les limites}}
                Soit $a \in \cj \R$, $I$ un voisinage de $a$, et $f, g \in \R^I$ admettant des limites finies en $a$.

                Alors $\forall \lambda, \mu \in \R$,
                \[
                    \lim_a(\lambda f + \mu g) = \lambda \lim_a f + \mu \lim_a g
                \]

                Et :
                \[
                    \lim_a(fg) = (\lim_a f) (\lim_a g)
                \]

                De plus, si $\displaystyle \lim_a f \in \R^*$, alors $\dfrac 1 f$ est définie au voisinage de $a$, et :
                \[
                    \lim_a\!\lr{\dfrac 1 f} = \dfrac{1}{\lim_a f}
                \]

                \vspace{12pt}
                
                \textbf{Remarque} :
                \[
                    \begin{array}{ccccc}
                        \varphi_a & : & E_a & \longrightarrow & \R
                        \\
                        && f & \longmapsto & \displaystyle \lim_a f
                    \end{array}
                \]
                où $E_a = \set{f \in \R^I\ |\ \exists \ell \in \R\ |\ f(x) \tendsto{x \to a} \ell}$

                est une forme linéaire.
            \end{indt}

            \vspace{12pt}
            
            \begin{indt}{\subsubsection{Théorème de comparaison}}
                Soit $a \in \cj \R$, et soient $f, g$ deux fonctions définies au voisinage $I$ de $a$, telles que $\forall x \in I,\ f(x) \le g(x)$.

                \vspace{6pt}
                
                $\bullet$ Si $f(x) \tendsto{x \to a} +\infty$, alors $g(x) \tendsto{x \to a} +\infty$.

                $\bullet$ Si $g(x) \tendsto{x \to a} -\infty$, alors $f(x) \tendsto{x \to a} -\infty$.
            \end{indt}

            \vspace{12pt}
            
            \begin{indt}{\subsubsection{Passage à la limite dans une inégalité large}}
                Soit $a \in \cj \R$, et soient $f, g$ deux fonctions définies au voisinage $I$ de $a$ telles que
                \[
                    \exists \ell, \ell' \in \R\
                    \begin{array}{|l}
                        f(x) \tendsto{x \to a} \ell
                        \\
                        g(x) \tendsto{x \to a} \ell'
                    \end{array}
                \]
                et telles que $\forall x \in I,\ f(x) \le g(x)$.

                \vspace{6pt}
                
                Alors :
                \[
                    \ell \le \ell'
                \]
            \end{indt}

            \vspace{12pt}
            
            \begin{indt}{\subsubsection{Théorème d'encadrement}}
                Soit $a \in \cj \R$, et soient $f, g, h$ trois fonctions définies au voisinage $I$ de $a$, telles que $\forall x \in I,\ f(x) \le g(x) \le h(x)$.

                \vspace{6pt}
                
                Si
                $
                    \exists \ell \in \R\
                    \begin{array}{|l}
                        f(x) \tendsto{x \to a} \ell
                        \\
                        h(x) \tendsto{x \to a} \ell
                    \end{array}
                $, alors $g$ admet une limite en $a$, et
                \[
                    g(x) \tendsto{x \to a} \ell
                \]
            \end{indt}

            \vspace{6pt}
            
            \begin{indt}{\subsubsection{Limites d'une fonction composée}}
                Soient $f \in \R^{\mathcal D_f}$ et $g \in \R^{\mathcal D_g}$ tel que $f(\mathcal D_f) \subset \mathcal D_g$

                Si $\exists a, b, c \in \cj \R$ tel que
                \[
                    \begin{array}{|l}
                        f(x) \tendsto{x \to a} b
                        \\
                        g(y) \tendsto{y \to b} c
                    \end{array}
                \]

                Alors $(g \circ f)(x) \tendsto{x \to a} c$
            \end{indt}

            \vspace{12pt}
            
            \begin{indt}{\subsubsection{Définition (\textit{bornes inférieure et supérieure d'une fonction})}}
                Soit $D \subset \R$, et $f \in \R^D$.

                \begin{indt}{On définit :}
                    $\bullet$ $\displaystyle \sup_D(f) = \sup\set{f(x)\ |\ x \in D}$

                    $\bullet$ $\displaystyle \inf_D(f) = \inf\set{f(x)\ |\ x \in D}$
                \end{indt}
            \end{indt}

            \vspace{12pt}
            
            \begin{indt}{\subsubsection{Théorème de la limite monotone}}
                Soit $I$ un intervalle de $\R$,
                $
                    \begin{array}{|l}
                        m = \inf(I)
                        \\
                        M = \sup(I)
                    \end{array}
                $
                Soit $f \in \R^I$ une fonction monotone sur $I$.

                \vspace{12pt}
                
                Alors
                \[
                    \forall a \in I,\ \exists \ell, \ell' \in \cj \R\
                    \begin{array}{|l}
                        f(x) \tendsto{x \to a^-} \ell
                        \\
                        f(x) \tendsto{x \to a^+} \ell'
                    \end{array}
                \]

                \begin{indt}{De plus, si $f$ est croissante, on a :}
                    $\bullet$ $\displaystyle f(x) \tendsto{x \to m^+} \inf_{]m ; M[}(f)$

                    $\bullet$ $\displaystyle f(x) \tendsto{x \to M^-} \sup_{]m ; M[}(f)$

                    $\bullet$ $\displaystyle \exists c \in \R\ |\ \forall x \in I,\ f(x) \ge c \ssi \lim_{m^-}(f) \neq -\infty$

                    $\bullet$ $\displaystyle \exists C \in \R\ |\ \forall x \in I,\ f(x) \le C \ssi \lim_{M^+}(f) \neq +\infty$

                    $\bullet$ $\displaystyle \forall x \in ]m ; M[,\ \lim_{m^+}(f) \le f(x) \le \lim_{M^-}(f)$
                \end{indt}
            \end{indt}
        \end{indt}
        
    \end{indt}

    \vspace{12pt}
    
    \begin{indt}{\section{Continuité}}
        \begin{indt}{\subsection{En un point}}
            \begin{indt}{\subsubsection{Définition (\textit{continuité en un point})}}
                Soit $f \in \R^{\mathcal D_f}$, et $a \in \mathcal D_f$.

                La fonction $f$ est continue en $a$ si elle y admet une limite finie, \textit{i.e} si (\textit{cf} \ref{1.2.3}, page \pageref{1.2.3})
                \[
                    \exists \ell \in \R\ |\
                    \forall \ep \in \R_+^*,\
                    \exists \delta \in \R_+^*\ |\
                    \forall x \in \mathcal D_f,\
                    \abs{x - a} \le \delta
                    \Rightarrow
                    \abs{f(x) - \ell} \le \ep
                \]
            \end{indt}

            \vspace{12pt}
            
            \begin{indt}{\subsubsection{Propriétés}}
                Soit $f \in \R^{\mathcal D_f}$, et $a \in \mathcal D_f$.

                $\bullet$ Si $f$ est continue en $a$, alors
                \[
                    f(x) \tendsto{x \to a} f(a)
                \]

                $\bullet$ La fonction $f$ est continue en $a$ $\ssi$
                \[
                    f_{|\mathcal D_f \setminus \set a}(x) \tendsto{x \to a} f(a)
                \]

                Remarque : par abus d'écriture, on abrège souvent cela par : $f$ continue en $a$ $\ssi f(x) \tendsto{x \to a} f(a)$.
            \end{indt}

            \vspace{12pt}
            
            \begin{indt}{\subsubsection{Prolongement par continuité}}
                Soit $I$ un intervalle de $\R$, $a \in I$, et $f \in \R^{I \setminus \set a}$.

                La fonction $f$ est \textit{prolongeable par continuité} en $a$ si elle admet une limite finie en $a$, et dans ce cas, son \textit{prolongement par continuité} est la fonction
                \[
                    \begin{array}{ccc}
                        I & \longrightarrow & \R
                        \\
                        x & \longmapsto &
                        \left\{\!\!
                        \begin{array}{ll}
                            f(x) & \text{si}\ x \neq a
                            \vspace{6pt}
                            \\
                            \displaystyle \lim_a(f) & \text{sinon}
                            %\vspace{-3pt}
                        \end{array}
                        \right.
                    \end{array}
                \]
            \end{indt}

            \vspace{12pt}
            
            \begin{indt}{\subsubsection{Définition (\textit{continuité à droite et à gauche})}}
                Soit $D$ un sous-ensemble de $\R$, $a \in D$, et $f \in \R^D$.

                La fonction $f$ est continue à \textit{gauche} (resp. à \textit{droite}) en $a$ si elle admet une limite à gauche (resp. à droite) en $a$ (\textit{cf} \ref{1.2.6}, page \pageref{1.2.6}), \textbf{et} si cette limite est $f(a)$.
            \end{indt}

            \vspace{12pt}
            
            \begin{indt}{\subsubsection{Propriétés}}
                Soit $D$ un sous-ensemble de $\R$, $a \in D$, et $f \in \R^D$.

                $\bullet$ La fonction $f$ est continue en $a$ $\ssi$
                \[
                    \begin{cases}
                        f(x) \tendsto{x \to a^-} f(a)
                        \\
                        f(x) \tendsto{x \to a^+} f(a)
                    \end{cases}
                \]

                $\bullet$ Si $f$ est continue en $a$, alors elle est bornée au voisinage de $a$.
            \end{indt}
        \end{indt}

        \vspace{12pt}
        
        \begin{indt}{\subsection{Sur un intervalle}}
            \begin{indt}{\subsubsection{Définition (\textit{Continuité sur un intervalle})}}
                Soit $I$ un intervalle de $\R$, et $f \in \R^I$.

                La fonction $f$ est continue sur l'intervalle $I$ si elle est continue en tout point de $I$, \textit{i.e} si :
                \[
                    \forall a \in I,\ f_{|I \setminus \set a}(x) \tendsto{x \to a} f(a)
                \]
            \end{indt}

            \vspace{12pt}
            
            \begin{indt}{\subsubsection{Définition}}
                Soient $I, J$ des intervalles de $\R$.

                On note $\mathcal C^0(I, J)$ l'ensemble des fonctions continues de $J^I$ :
                \[
                    \mathcal C^0(I, J) = \set{f \in J^I\ |\ \forall a \in I,\ f_{| I \setminus \set a}(x) \tendsto{x \to a} f(a)}
                \]

                Dans le cas où $J = \R$, on note plus simplement $\mathcal C^0(I) = \mathcal C^0(I, \R)$.
            \end{indt}

            \vspace{12pt}
            
            \begin{indt}{\subsubsection{Propriétés}}
                $\bullet$ Soit $I$ un intervalle de $\R$, et $f, g \in \mathcal C^0(I)$.

                On a :
                \[
                    \forall \lambda, \mu \in \R,\ \lambda f + \mu g \in \mathcal C^0(I)
                \]
                \[
                    fg \in \mathcal C^0(I)
                \]

                \vspace{6pt}
                
                $\bullet$ Soient $I, J$ des intervalles de $\R$, $f \in \mathcal C^0(I, J)$, et $g \in \mathcal C^0(J, \R)$.

                Alors :
                \[
                    g \circ f \in \mathcal C^0(I)
                \]
            \end{indt}
        \end{indt}

        \vspace{12pt}
        
        \begin{indt}{\subsection{Théorème des valeurs intermédiaires}}
            \begin{indt}{\subsubsection{Cas particulier}}
                Soient $a, b \in \R\ |\ a < b$, $I = [a, b]$, et $f \in \mathcal C^0(I)$ telle que
                \[
                    f(a)f(b) \le 0
                \]

                Alors
                \[
                    \exists x \in I\ |\ f(x) = 0
                \]
            \end{indt}

            \vspace{12pt}
            
            \begin{indt}{\subsubsection{Théorème des valeurs intermédiaires (cas général)}}
                Soient $a, b \in \R\ |\ a < b$, $I = [a, b]$, et $f \in \mathcal C^0(I)$.

                On a :
                \[
                    \forall d \in \seg{\inf_I(f)}{\sup_I(f)},\
                    \exists x \in I\ |\
                    d = f(x)
                \]

                Autrement dit :
                \[
                    \seg{\inf_I(f)}{\sup_I(f)} \subseteq f(I)
                \]
            \end{indt}

            \vspace{12pt}
            
            \begin{indt}{\subsubsection{Corollaire}}
                Soit $I$ un intervalle de $\R$ et $f \in \mathcal C^0(I)\ |\ \forall x \in I,\ f(x) \neq 0$.

                Alors $f$ est de signe constant sur $I$.
            \end{indt}

            \vspace{12pt}
            
            \begin{indt}{\subsubsection{Corollaire}}
                Soit $I$ un intervalle de $\R$, $f \in \mathcal C^0(I)$, et
                $
                    \left|
                    \begin{array}{l}
                        \displaystyle a = \inf_I(f)
                        \\
                        \displaystyle b = \sup_I(f)
                        \vspace{-3pt}
                    \end{array}
                    \right.
                $.

                On a :
                \[
                    \exists \ell, \ell' \in \cj \R\
                    \begin{cases}
                        f(x) \tendsto{x \to a} \ell
                        \\
                        f(x) \tendsto{x \to b} \ell'
                        \\
                        \ell \ell' < 0
                    \end{cases}
                    \Rightarrow \exists x \in I\ |\ f(x) = 0
                \]
            \end{indt}

            \vspace{12pt}
            
            \begin{indt}{\subsubsection{Théorème}}
                Soit $I$ un intervalle de $\R$, et $f \in \mathcal C^0(I)$.

                Alors $f(I)$ est un intervalle.
            \end{indt}
        \end{indt}

        \vspace{12pt}
        
        \begin{indt}{\subsection{Sur un segment}}
            \begin{indt}{\subsubsection{Définition (\textit{segment})}}
                Un segment de $\R$ est un intervalle fermé, \textit{i.e} un intervalle du type $\seg a b$, où $a, b \in \R,\ a < b$.
            \end{indt}

            \vspace{12pt}
            
            \begin{indt}{\subsubsection{Théorème}}
                Soit $I$ un segment de $\R$, et $f \in \mathcal C^0(I)$.

                Alors :
                \[
                    \exists a, b \in I\ |\
                    \forall x \in I,\
                    f(a) \le f(x) \le f(b)
                \]

                \vspace{6pt}
                
                et par définition,
                $
                    \left|
                    \begin{array}{l}
                        f(a) = \displaystyle \inf_I(f) = \min_I(f)
                        \\
                        f(b) = \displaystyle \sup_I(f) = \max_I(f)
                        \vspace{-3pt}
                    \end{array}
                    \right.
                $.

                \vspace{6pt}
                
                On dit aussi qu'une fonction continue sur un segment est bornée et atteint ses bornes.
            \end{indt}

            \vspace{12pt}
            
            \begin{indt}{\subsubsection{Propriété}}
                Soient $a, b \in \R\ |\ a < b$, $I = \seg a b$, et $f \in \mathcal C^0(I)$.

                Alors :
                \[
                    f(I) = \seg{\min_I(f)}{\max_I(f)}
                \]
            \end{indt}
        \end{indt}

        \vspace{12pt}
        
        \begin{indt}{\subsection{Monotonie et bijection}}
            \begin{indt}{\subsubsection{Propriété}}
                Soit $I$ un intervalle de $\R$, et $f \in \R^I$ une fonction monotone telle que $f(I)$ soit un intervalle.

                Alors :
                \[
                    f \in \mathcal C^0(I)
                \]
            \end{indt}

            \vspace{12pt}
            
            \begin{indt}{\subsubsection{Propriété}}
                Soit $I$ un intervalle de $\R$, et $f \in \mathcal C^0(I)$.

                Alors :
                \[
                    f\ \text{injective} \ssi f\ \text{strictement monotone}
                \]
            \end{indt}

            \vspace{12pt}
            
            \begin{indt}{\subsubsection{Théorème (\textit{continuité de la fonction réciproque})}}
                Soient $I, J$ des intervalles de $\R$, et $f \in \mathcal C^0(I, J)$ une fonction bijective.

                On a alors :
                \[
                    f^{-1} \in \mathcal C^0(J, I)
                \]
            \end{indt}

            \vspace{12pt}
            
            \begin{indt}{\subsubsection{Théorème de la bijection}}
                Soient $a, b \in \R\ |\ a < b$, $I = \seg a b$, et $f \in \mathcal C^0(I)$ une fonction strictement monotone.

                \begin{indt}{Alors :}
                    $\bullet$ $f$ réalise une bijection de $I$ sur
                    $
                        J = 
                        \begin{cases}
                            \seg{f(a)}{f(b)}
                            & \text{si $f$ croissante}
                            \\
                            \seg{f(b)}{f(a)}
                            & \text{sinon}
                        \end{cases}
                    $

                    \vspace{6pt}
                    
                    $\bullet$ Sa fonction réciproque $f^{-1} : J \longrightarrow I$ est continue, strictement monotone, et de même sens de variations que $f$.
                \end{indt}

                \vspace{12pt}
                
                Remarque : on peut ouvrir un / les crochets de $I$, et dans ce cas, on ouvre le / les crochets correspondant dans $J$, en remplaçant $f(x)$  par $\displaystyle \lim_x f$ (où $x \in \set{a, b}$).
            \end{indt}
        \end{indt}
        
    \end{indt}

    \vspace{12pt}
    
    \begin{indt}{\section{Dérivation}}
        On fixe un intervalle $I$ de $\R$ non trivial.

        \begin{indt}{\subsection{Dérivée en un point}}
            \begin{indt}{\subsubsection{Définition (\textit{dérivée en un point})}}
                Soit $f \in \R^I$, et $a \in I$

                Soit $\tau_a$ le taux d'accroissement de $f$ en $a$ :
                \[
                    \begin{array}{rcccc}
                        \tau_a & : & I \setminus \set a & \longrightarrow & \R
                        \\
                        && x & \longmapsto & \dfrac{f(x) - f(a)}{x - a}
                    \end{array}
                \]

                \vspace{6pt}
                
                La fonction $f$ est \textit{dérivable} (resp. \textit{dérivable à gauche}, \textit{à droite}) en $a$ si $\tau_a$ admet une limite (resp. limite à gauche, droite) finie en $a$, et on note :
                \[
                    f'(a) = \lim_{x \to a} \tau_a(x)
                    \quad \lr{\text{resp.}\ \
                        \begin{array}{|l}
                            f_{\rm g}'(a) = \displaystyle \lim_{x \to a^-} \tau_a(x)
                            \\
                            f_{\rm r}'(a) = \displaystyle \lim_{x \to a^+} \tau_a(x)
                        \end{array}
                    }
                \]

                Remarque : on note aussi $\dfrac{\mathrm d f}{\mathrm dx}(a)$ la dérivée de $f$ en $a$ par rapport à $x$.
            \end{indt}

            \vspace{12pt}
            
            \begin{indt}{\subsubsection{Propriété (\textit{équation de la tangente})}}
                Soit $a \in I$, et $f \in \R^I$ une fonction dérivable en $a$.

                La tangente au graphe de $f$ en $a$ a pour équation :
                \[
                    y = f'(a)(x - a) + f(a)
                \]
            \end{indt}

            \vspace{12pt}
            
            \begin{indt}{\subsubsection{Propriété (\textit{CNS de dérivabilité en un point})}}
                Soit $a \in I \setminus \set{\inf(I), \sup(I)}$, et $f \in \R^I$.

                La fonction $f$ est dérivable en $a$
                $
                    \ssi
                    \begin{cases}
                        \text{$f$ est dérivable à droite et à gauche en $a$}
                        \\
                        f_{\rm g}'(a) = f_{\rm d}'(a)
                    \end{cases}
                $
            \end{indt}

            \vspace{12pt}
            
            \begin{indt}{\subsubsection{Propriété (\textit{continuité des fonctions dérivables})}}
                Soit $f \in \R^I$, et $a \in I$.

                Si $f$ est dérivable (resp. dérivable à gauche, droite) en $a$, alors $f$ est continue (resp. continue à gauche, droite) en $a$.
            \end{indt}
        \end{indt}

        \vspace{12pt}
        
        \begin{indt}{\subsection{Sur un intervalle}}
            \begin{indt}{\subsubsection{Définition (\textit{fonction dérivée})}}
                Soit $f \in \R^I$.

                La fonction $f$ est dérivable sur $I$ si elle est dérivable en tout point de $I$, et la \textit{dérivée} de $f$ est la fonction
                \[
                    \begin{array}{rcccc}
                        f' & : & I & \longrightarrow & \R
                        \\
                           && x & \longmapsto & f'(x)
                    \end{array}
                \]

                On note $\mathcal D(I, J)$ l'ensemble des fonctions dérivables sur $I$ à valeurs dans $\R$. On omet $J$ lorsque celui-ci est $\R$.

                Remarque : on note aussi $\dfrac{\mathrm df}{\mathrm dx} = f'$.
            \end{indt}

            \vspace{12pt}
            
            \begin{indt}{\subsubsection{Propriété (\textit{continuité des fonctions dérivables})}}
                Soit $f \in \R^I$.

                On a
                \[
                    f \in \mathcal D(I)
                    \Rightarrow
                    f \in \mathcal C^0(I)
                \]

                \textit{i.e}
                \[
                    \mathcal C^0(I) \subset \mathcal D(I)
                \]
            \end{indt}

            \vspace{12pt}
            
            \begin{indt}{\subsubsection{Propriété (\textit{linéarité de la dérivation})}}
                Soit l'application
                \[
                    \begin{array}{rcccc}
                        \varphi & : & \mathcal D(I) & \longrightarrow & \R^I
                        \\
                        && f & \longmapsto & f'
                    \end{array}
                \]

                Alors $\varphi$ est une application linéaire, \textit{i.e} :
                \[
                    \forall \lambda, \mu \in \R, \forall f, g \in \mathcal D(I),\
                    \varphi(\lambda f + \mu g) = \lambda \varphi(f) + \mu \varphi(g)
                \]
                \textit{i.e} :
                \[
                    \forall \lambda, \mu \in \R, \forall f, g \in \mathcal D(I),\
                    (\lambda f + \mu g)' = \lambda f' + \mu g'
                \]

                \vspace{12pt}
                
                Remarque : $\forall a \in I, \forall f \in \mathcal D(I)$, soit $\varphi_a(f) = \varphi(f)(a)$.
                Alors $\varphi_a$ est une forme linéaire.
            \end{indt}

            \vspace{12pt}
            
            \begin{indt}{\subsubsection{Propriété (\textit{produit de dérivées})}}
                Soient $f, g \in \mathcal D(I)$.

                Alors $fg \in \mathcal D(I)$, et :
                \[
                    (fg)' = f'g + g'f
                \]
            \end{indt}
        \end{indt}

        \vspace{12pt}
        
        \begin{indt}{\subsection{Dérivées d'ordre supérieur}}
            \begin{indt}{\subsubsection{Définition (\textit{Dérivée $n$-ème})}}
                Soit $f \in \R^I$.

                On définit par récurrence, pour $n \in \N$, la \textit{dérivée $n$-ème} $f^{(n)}$ de $f$, si elle existe, par :
                \[
                    \begin{cases}
                        f^{(0)} = f
                        \\
                        f^{(n)} = \lr{f^{(n - 1)}}'
                    \end{cases}
                \]

                En cas d'existence, on dit que $f$ est $n$-fois dérivable.

                \vspace{6pt}
                
                Remarque : on note aussi $\dfrac{\mathrm d^n f}{\mathrm dx^n} = f^{(n)}$.
            \end{indt}

            \vspace{12pt}
            
            \begin{indt}{\subsubsection{Définition (\textit{Fonctions de classe $\mathcal C^n$})}}
                $\bullet$ On note $\mathcal C^n(I, J)$ l'ensemble des fonctions $n$-fois dérivables sur $I$ à valeurs dans $J$, et dont la dérivée $n$-ème est continue.
                On omet $J$ si celui-ci est $\R$.

                Si $f \in \mathcal C^n(I)$, on dit que $f$ est de classe $\mathcal C^n$ sur $I$.

                \vspace{6pt}
                
                $\bullet$ On note
                $
                    \mathcal C^\infty(I, J)
                    = \set{f \in J^I\ |\ \forall n \in \N,\ f \in \mathcal C^n(I, J)}
                $,
                \textit{i.e} l'ensemble des fonctions dérivables à tout ordre.
                De même que précédemment, on omet $J$ dans le cas $J = \R$.

                Si $f \in \mathcal C^\infty(I)$, on dit que $f$ est de classe $\mathcal C^\infty$ sur $I$.
            \end{indt}

            \vspace{12pt}
            
            \begin{indt}{\subsubsection{Propriétés}}
                $\bullet$ Soient $n, p \in \N\ |\ p \le n$.

                Alors
                \[
                    \mathcal C^n(I) \subset \mathcal C^p(I)
                \]

                \vspace{6pt}
                
                $\bullet$ Soit $n \in \N^*$, et $f \in \R^I$ une fonction $n$-fois dérivable.

                Alors
                \[
                    \forall p \in \nset{0}{n - 1},\ f \in \mathcal C^p(I)
                \]

                \vspace{6pt}
                
                $\bullet$ On a
                \[
                    \mathcal C^\infty(I) = \bigcap_{n \in \N} \mathcal C^n(I)
                \]

                \vspace{6pt}
                
                $\bullet$ Soit $n \in \N$, et $f \in \mathcal D(I)$.

                Alors
                \[
                     f \in \mathcal C^n(I) \ssi f' \in \mathcal C^{n - 1}(I)
                \]
            \end{indt}

            \vspace{12pt}
            
            \begin{indt}{\subsubsection{Propriété (\textit{linéarité de la dérivation $n$-ème})}}
                Soit $n \in \N$.

                L'application
                \[
                    \begin{array}{rcccc}
                        \varphi_n & : & \mathcal C^n(I) & \longrightarrow & \R^I
                        \\
                        && f & \longmapsto & f^{(n)}
                    \end{array}
                \]
                est linéaire, \textit{i.e} :
                \[
                    \forall \lambda, \mu \in \R, \forall f, g \in \mathcal C^n(I),\
                    (\lambda f + \mu g)^{(n)} = \lambda f^{(n)} + \mu g^{(n)}
                \]
            \end{indt}

            \vspace{12pt}
            
            \begin{indt}{\subsubsection{Propriété (\textit{Produit de dérivées $n$-èmes})}}
                Soient $n \in \N$, et $f, g \in \mathcal C^n(I)$.

                Alors $fg \in \mathcal C^n(I)$, et :
                \[
                    (fg)^{(n)} =
                    \sum_{k = 0}^n \binom n k f^{(k)} g^{(n - k)}
                \]
                (formule de \textsc{Leibniz}).

                \vspace{12pt}
                
                Remarque : $\mathcal C^n(I)$ est un sous-anneau de $(\R^I, +, \times)$.
            \end{indt}

            \vspace{12pt}
            
            \begin{indt}{\subsubsection{Propriété (\textit{dérivée $n$-ème des monômes})}}
                $\bullet$ Soit $p \in \N$.

                On a :
                $x \longmapsto x^p \in \mathcal C^\infty(\R)$, et :
                \[
                    \forall n \in \N,\
                    \dfrac{\mathrm d^n \! \left[ x^p \right]}{\mathrm dx^n} =
                    \begin{cases}
                        \dfrac{p!}{(p - n)!}x^{p - n}
                        & \text{si}\ n \le p
                        \vspace{6pt}
                        \\
                        0 & \text{sinon}
                    \end{cases}
                \]

                \vspace{12pt}
                
                $\bullet$ On a : $x \longmapsto \dfrac 1 x \in \mathcal C^\infty(\R^*)$, et :
                \[
                    \forall n \in \N,\
                    \dfrac{\mathrm d^n \! \left[ \dfrac 1 x \right]}{\mathrm dx^n} =
                    (-1)^n \dfrac{n!}{x^{n + 1}}
                \]
            \end{indt}
        \end{indt}

        \vspace{12pt}
        
        \begin{indt}{\subsection{Quotient}}
            \begin{indt}{\subsubsection{Propriété (\textit{dérivée de l'inverse en un point})}}
                Soit $a \in I$, et $f \in \R^I$ une fonction dérivable en $a$ et telle que $\forall x \in I,\ f(x) \neq 0$.

                Alors $\dfrac 1 f$ est dérivable en $a$, et :
                \[
                    \lr{\dfrac 1 f}'(a) = -\dfrac{f'(a)}{\lr{f(a)}^2}
                \]
            \end{indt}

            \vspace{12pt}
            
            \begin{indt}{\subsubsection{Propriété (\textit{dérivée de l'inverse sur un intervalle})}}
                $\bullet$ Soit $f \in \mathcal D(I)\ |\ \forall x \in I,\ f(x) \neq 0$.

                Alors $\dfrac 1 f \in \mathcal D(I)$, et
                \[
                    \lr{\dfrac 1 f}' = - \dfrac{f'}{f^2}
                \]

                \vspace{6pt}
                
                $\bullet$ Soit $f \in \mathcal C^n(I, \R^*)$.

                Alors $\dfrac 1 f \in \mathcal C^n(I)$.
            \end{indt}

            \vspace{12pt}
            
            \begin{indt}{\subsubsection{Corollaire (\textit{dérivé d'un quotient})}}
                $\bullet$ Soient $f, g \in \mathcal D(I)\ |\ \forall x \in I,\ g(x) \neq 0$.

                Alors $\dfrac f g \in \mathcal D(I)$, et
                \[
                    \lr{\dfrac f g}' = \dfrac{f'g - g'f}{g^2}
                \]

                \vspace{12pt}
                
                $\bullet$ Soient $f, g \in \mathcal C^n(I)\ |\ \forall x \in I,\ g(x) \neq 0$.

                Alors $\dfrac f g \in \mathcal C^n(I)$.
            \end{indt}
        \end{indt}

        \vspace{12pt}
        
        \begin{indt}{\subsection{Composée}}
            \begin{indt}{\subsubsection{Propriété (\textit{dérivée d'une fonction composée en un point})}}
                Soient $I, J$ des intervalles de $\R$, $a \in I$, et
                $
                    \left| \!
                    \begin{array}{l}
                        f \in J^I
                        \\
                        g \in \R^J
                    \end{array}
                    \right.
                $
                telles que $f$ soit dérivable en $a$ et $g$ en $f(a)$.

                Alors $g \circ f$ est dérivable en $a$, et :
                \[
                    (g \circ f)'(a) = f'(a) \cdot g'\!\lr{f(a)}
                \]
            \end{indt}

            \vspace{12pt}
            
            \begin{indt}{\subsubsection{Propriété (\textit{dérivée d'une fonction composée sur un intervalle})}}
                Soient $I, J$ des intervalles de $\R$, et
                $
                    \begin{array}{|l}
                        f \in \mathcal D(I, J)
                        \\
                        g \in \mathcal D(J, \R)
                    \end{array}
                $.

                Alors $g \circ f \in \mathcal D(I)$, et
                \[
                    (g \circ f)' = f' \cdot (g' \circ f)
                \]
            \end{indt}

            \vspace{12pt}
            
            \begin{indt}{\subsubsection{Propriété (\textit{Classe $\mathcal C^n$})}}
                Soient $I, J$ des intervalles de $\R$, $n \in \N$, et
                $
                    \begin{array}{|l}
                        f \in \mathcal C^n(I, J)
                        \\
                        g \in \mathcal C^n(J, \R)
                    \end{array}
                $.

                Alors $g \circ f \in \mathcal C^n(I)$.
            \end{indt}
        \end{indt}

        \vspace{12pt}
        
        \begin{indt}{\subsection{Fonction réciproque}}
            \begin{indt}{\subsubsection{Propriété (\textit{dérivée de la fonction réciproque en un point})}}
                Soit $a \in I$, et $f \in \mathcal C^0(I)$ une fonction bijective de $I$ sur $J = f(I)$, dérivable en $a$, telle que $f'(a) \neq 0$.

                Alors $f^{-1}$ est dérivable en $f(a)$, et
                \[
                    \lr{f^{-1}}'\!\lr{f(a)} = \dfrac{1}{f'(a)}
                \]
            \end{indt}

            \vspace{12pt}
            
            \begin{indt}{\subsubsection{Propriété (\textit{dérivée de la fonction réciproque sur un intervalle})}}
                Soit $f \in \mathcal D(I)$ une fonction bijective de $I$ sur $J = f(I)$, telle que $\forall x \in I,\ f'(x) \neq 0$.

                Alors $f^{-1} \in \mathcal D(J)$, et
                \[
                    \lr{f^{-1}}' = \dfrac{1}{f' \circ f^{-1}}
                \]
            \end{indt}

            \vspace{12pt}
            
            \begin{indt}{\subsubsection{Corollaire}}
                Soit $f \in \mathcal D(I)$ une fonction bijective de $I$ sur $J = f(I)$.

                Alors
                \[
                    f^{-1} \in \mathcal D\!\lr{\set{
                        f(x) \in J\
                        \left|
                        \begin{array}{l}
                            x \in I
                            \\
                            f'(x) \neq 0
                        \end{array}
                        \right.
                    }}
                \]
            \end{indt}

            \vspace{12pt}
            
            \begin{indt}{\subsubsection{Propriété}}
                Soit $n \in \N^*$, et $f \in \mathcal C^n(I)\ |\ \forall x \in I,\ f'(x) \neq 0$.

                Alors $f$ réalise une bijection de $I$ sur $f(I)$, et $f^{-1} \in \mathcal C^n(f(I))$.

                \vspace{6pt}
                
                Remarque : un telle fonction s'appelle un \textit{$\mathcal C^n$-difféomorphisme}.
            \end{indt}
        \end{indt}

        \vspace{12pt}
        
        \begin{indt}{\subsection{Théorème de Rolle}}
            \begin{indt}{\subsubsection{Propriété}}
                Soit $a \in I \setminus \set{\inf(I), \sup(I)}$, et $f \in \mathcal D(I)$.

                Si $f$ admet un extremum en $a$, alors $f'(a) = 0$.

                \vspace{12pt}
                
                Remarques : un tel point est un point critique.
                La réciproque est fausse ($x \longmapsto x^3$ en 0).
            \end{indt}

            \vspace{12pt}
            
            \begin{indt}{\subsubsection{Théorème de Rolle}}
                Soient $a, b \in \R\ |\ a < b$, et $f \in \mathcal C^0\!\lr{\seg a b} \cap \mathcal D\!\lr{]a\ ;\ b[}\ |\ f(a) = f(b)$.

                Alors
                \[
                    \exists c \in ]a\ ;\ b[\ |\ f'(c) = 0
                \]
            \end{indt}
        \end{indt}

        \vspace{12pt}
        
        \begin{indt}{\subsection{\'Egalité des accroissements finis}}
            \begin{indt}{\subsubsection{Théorème (\textit{égalité des accroissements finis})}}
                Soient $a, b \in \R\ |\ a < b$, et soit $f \in \mathcal C^0\!\lr{\seg a b} \cap \mathcal D\!\lr{]a\ ;\ b[}$.

                Alors
                \[
                    \exists c \in ]a\ ;\ b[\ |\ f'(c) = \dfrac{f(b) - f(a)}{b - a}
                \]
            \end{indt}

            \vspace{12pt}
            
            \begin{indt}{\subsubsection{Définition (\textit{fonction lipschitzienne})}}
                Soit $f \in \R^I$.

                La fonction $f$ est \textit{lipschitzienne} si
                \[
                    \exists k \in \R\ |\
                    \forall x, y \in I,\
                    \abs{f(x) - f(y)} \le k\abs{x - y}
                \]

                Dans ce cas, on dit que $f$ est $k$-lipschitzienne.
            \end{indt}

            \vspace{12pt}
            
            \begin{indt}{\subsubsection{Propriété}}
                Une fonction lipschitzienne est continue.
            \end{indt}

            \vspace{12pt}
            
            \begin{indt}{\subsubsection{Corollaire (\textit{Inégalité des accroissements finis})}}
                Soient $a, b \in \R\ |\ a < b$, et $f \in \mathcal C^0\!\lr{\seg a b} \cap \mathcal D\!\lr{]a\ ;\ b[}$ telle que
                \[
                    \exists m, M \in \R\ |\
                    \forall x \in ]a\ ;\ b[,\
                    m \le f'(x) \le M.
                \]

                Alors
                \[
                    \forall x, y \in \seg a b,\
                    x \le y \Rightarrow
                    m(y - x) \le f(y) - f(x) \le M(y - x)
                \]

                \textit{i.e} $f$ est $\max\!\lr{\abs m,\ \abs M}$-lipschitzienne.
            \end{indt}

            \vspace{12pt}
            
            \begin{indt}{\subsubsection{Propriétés}}
                $\bullet$ Soit $f \in \mathcal D(I)$ une fonction lipschitzienne sur $I$.

                Alors $f'$ est bornée.

                \vspace{12pt}
                
                $\bullet$ Soient $a, b \in \R\ |\ a < b$, et $f \in \mathcal C^1(\seg a b)$.

                Alors $f$ est lipschitzienne.
            \end{indt}
        \end{indt}

        \vspace{12pt}
        
        \begin{indt}{\subsection{Monotonie et dérivabilité}}
            \begin{indt}{\subsubsection{Propriété (\textit{CNS de monotonie avec la dérivabilité})}}
                Soit $I$ un intervalle de $\R$, $\widetilde I$ l'intérieur de $I$, et $f \in \mathcal C^0(I) \cap \mathcal D\!\lr{\widetilde I}$.

                Alors :
                \[
                    \begin{array}{c}
                        f \nearrow \ssi f' \ge 0
                        \\
                        f \searrow \ssi f' \le 0
                        \\
                        \exists a \in \R\ |\ \forall x \in I,\ f(x) = a \ssi f' = 0
                    \end{array}
                \]
            \end{indt}

            \vspace{12pt}
            
            \begin{indt}{\subsubsection{Propriété (\textit{conditions suffisantes de stricte monotonie})}}
                Soit $I$ un intervalle de $\R$, $\widetilde I$ l'intérieur de $I$, et $f \in \mathcal C^0(I) \cap \mathcal D\!\lr{\widetilde I}$.

                \vspace{6pt}
                
                $\bullet$ Si
                \[
                    \forall x \in \widetilde I,\ f'(x) > 0\ (\text{resp.}\ f'(x) < 0),
                \]
                alors $f$ est strictement croissante (resp. strictement décroissante) sur $I$.

                \vspace{12pt}
                
                $\bullet$ Si
                \[
                    \begin{cases}
                        \forall x \in \widetilde I,\ f'(x) \ge 0\ (\text{resp.}\ f'(x) \le 0)
                        \vspace{6pt}
                        \\
                        \mathrm{card}\set{x \in \widetilde I\ |\ f'(x) = 0} < +\infty
                    \end{cases}
                \]
                alors $f$ est strictement croissante (resp. strictement décroissante) sur $I$
                (corollaire du théorème suivant).
            \end{indt}

            \vspace{12pt}
            
            \begin{indt}{\subsubsection{Théorème (\textit{CNS de stricte monotonie})}}
                Soit $I$ un intervalle de $\R$, $\widetilde I$ l'intérieur de $I$, et $f \in \mathcal C^0(I) \cap \mathcal D\!\lr{\widetilde I}$.

                Alors
                $f$ est strictement croissante sur $I$ $\ssi$
                \[
                    \begin{cases}
                        \forall x \in \widetilde I,\ f(x) \ge 0
                        \\
                        \forall x, y \in \widetilde I\ |\ x < y,\ f'(x) = 0 = f'(y)
                        \Rightarrow \exists x_0 \in\ ]x\ ;\ y[\ \ |\ f'(x_0) \neq 0
                    \end{cases}
                \]

                Et $f$ est strictement décroissante sur $I$ $\ssi$
                \[
                    \begin{cases}
                        \forall x \in \widetilde I,\ f(x) \le 0
                        \\
                        \forall x, y \in \widetilde I\ |\ x < y,\ f'(x) = 0 = f'(y)
                        \Rightarrow \exists x_0 \in\ ]x\ ;\ y[\ \ |\ f'(x_0) \neq 0
                    \end{cases}
                \]
            \end{indt}
        \end{indt}

        \vspace{12pt}
        
        \begin{indt}{\subsection{Théorème de la limite de la dérivée}}
            \begin{indt}{\subsubsection{Théorème de la limite de la dérivée}}
                Soit $c \in I$, et $f \in \mathcal C^0(I) \cap \mathcal D\!\lr{I \setminus \set c}$.

                \vspace{6pt}
                
                $\bullet$ Si $\exists \ell \in \R\ |\ f'(x) \tendsto{x \to c} \ell$, alors $f \in \mathcal D(I)$,
                \[
                    f'(c) = \ell
                \]
                et $f'$ est continue en $c$.

                De plus, si $f \in \mathcal C^1\!\lr{I \setminus \set c}$, alors $f \in \mathcal C^1(I)$.

                \vspace{12pt}
                
                $\bullet$ Si $f'(x) \tendsto{x \to c} \pm \infty$, alors la courbe représentative de $f$ admet une tangente verticale au point d'abscisse $c$.
            \end{indt}

            \vspace{12pt}
            
            \begin{indt}{\subsubsection{Théorème de classe $\mathcal C^k$ par prolongement}}
                Soient $a \in I$, $n \in \N$, et $f \in \mathcal C^n\!\lr{I \setminus \set a}$ telle que
                \[
                    \forall k \in \nset 0 n,\
                    \exists \ell_k \in \R\ |\
                    f^{(k)}(x) \tendsto{x \to a} \ell_k.
                \]

                Alors $f$ est prolongeable par continuité en $a$, et en notant $\widetilde f$ son prolongement, on a :
                \[
                    \widetilde f \in \mathcal C^n(I)
                \]
                et :
                \[
                    \forall k \in \nset 0 n,\
                    {\widetilde f}^{(k)}(a) = \ell_k.
                \]
            \end{indt}
        \end{indt}
    \end{indt}
    
    
    
\end{document}
%--------------------------------------------End
