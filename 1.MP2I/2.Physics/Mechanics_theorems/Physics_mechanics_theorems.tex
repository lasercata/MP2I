\documentclass[a4paper, 12pt, twoside]{article}


%------------------------------------------------------------------------
%
% Author                :   Lasercata
% Last modification     :   2022.09.17
%
%------------------------------------------------------------------------


%------ini
\usepackage[utf8]{inputenc}
\usepackage[T1]{fontenc}
%\usepackage[french]{babel}
%\usepackage[english]{babel}


%------geometry
\usepackage[textheight=700pt, textwidth=500pt]{geometry}


%------color
\usepackage{xcolor}
\definecolor{ff4500}{HTML}{ff4500}
\definecolor{00f}{HTML}{0000ff}
\definecolor{0ff}{HTML}{00ffff}
\definecolor{656565}{HTML}{656565}

\renewcommand{\emph}{\textcolor{ff4500}}
\renewcommand{\em}{\color{ff4500}}

\newcommand{\strong}[1]{\textcolor{ff4500}{\bf #1}}
\newcommand{\st}{\color{ff4500}\bf}


%------Code highlighting
%---listings
\usepackage{listings}

\definecolor{cbg}{HTML}{272822}
\definecolor{cfg}{HTML}{ececec}
\definecolor{ccomment}{HTML}{686c58}
\definecolor{ckw}{HTML}{f92672}
\definecolor{cstring}{HTML}{e6db72}
\definecolor{cstringlight}{HTML}{98980f}
\definecolor{lightwhite}{HTML}{fafafa}

\lstdefinestyle{DarkCodeStyle}{
    backgroundcolor=\color{cbg},
    commentstyle=\itshape\color{ccomment},
    keywordstyle=\color{ckw},
    numberstyle=\tiny\color{cbg},
    stringstyle=\color{cstring},
    basicstyle=\ttfamily\footnotesize\color{cfg},
    breakatwhitespace=false,
    breaklines=true,
    captionpos=b,
    keepspaces=true,
    numbers=left,
    numbersep=5pt,
    showspaces=false,
    showstringspaces=false,
    showtabs=false,
    tabsize=4,
    xleftmargin=\leftskip
}

\lstdefinestyle{LightCodeStyle}{
    backgroundcolor=\color{lightwhite},
    commentstyle=\itshape\color{ccomment},
    keywordstyle=\color{ckw},
    numberstyle=\tiny\color{cbg},
    stringstyle=\color{cstringlight},
    basicstyle=\ttfamily\footnotesize\color{cbg},
    breakatwhitespace=false,
    breaklines=true,
    captionpos=b,
    keepspaces=true,
    numbers=left,
    numbersep=10pt,
    showspaces=false,
    showstringspaces=false,
    showtabs=false,
    tabsize=4,
    frame=L,
    xleftmargin=\leftskip
}

%\lstset{style=DarkCodeStyle}
\lstset{style=LightCodeStyle}
%Usage : \begin{lstlisting}[language=Caml] ... \end{lstlisting}

%---tcolorbox
\usepackage[many]{tcolorbox}
\DeclareTColorBox{pseudocode}{O{black}O{lightwhite}}{
    breakable,
    outer arc=0pt,
    arc=0pt,
    top=0pt,
    toprule=-.5pt,
    right=0pt,
    rightrule=-.5pt,
    bottom=0pt,
    bottomrule=-.5pt,
    colframe=#1,
    colback=#2,
    enlarge left by=10pt,
    width=\linewidth-\leftskip-10pt,
}


%-------make the table of content clickable
\usepackage{hyperref}
\hypersetup{
    colorlinks,
    citecolor=black,
    filecolor=black,
    linkcolor=black,
    urlcolor=black
}
%Uncomment this and comment above for dark mode
% \hypersetup{
%     colorlinks,
%     citecolor=white,
%     filecolor=white,
%     linkcolor=white,
%     urlcolor=white
% }


%------pictures
\usepackage{graphicx}
%\usepackage{wrapfig}

\usepackage{tikz}
%\usetikzlibrary{babel}             %Uncomment this to use circuitikz
%\usetikzlibrary{shapes.geometric}  % To draw triangles in trees
%\usepackage{circuitikz}            %Electrical circuits drawing


%------tabular
%\usepackage{color}
%\usepackage{colortbl}
%\usepackage{multirow}


%------Physics
%---Packages
%\usepackage[version=4]{mhchem} %$\ce{NO4^2-}$

%---Commands
\newcommand{\link}[2]{\mathrm{#1} \! - \! \mathrm{#2}}
\newcommand{\pt}[1]{\cdot 10^{#1}} % Power of ten
\newcommand{\dt}[2][t]{\dfrac{\mathrm d #2}{\mathrm d #1}} % Derivative


%------math
%---Packages
%\usepackage{textcomp}
%\usepackage{amsmath}
\usepackage{amssymb}
\usepackage{mathtools} % For abs
\usepackage{stmaryrd} %for \llbracket and \rrbracket
\usepackage{mathrsfs} %for \mathscr{x} (different from \mathcal{x})

%---Commands
%-Sets
\newcommand{\N}{\mathbb{N}} %set N
\newcommand{\Z}{\mathbb{Z}} %set Z
\newcommand{\Q}{\mathbb{Q}} %set Q
\newcommand{\R}{\mathbb{R}} %set R
\newcommand{\C}{\mathbb{C}} %set C
\newcommand{\U}{\mathbb{U}} %set U
\newcommand{\seg}[2]{\left[ #1\ ;\ #2 \right]}
\newcommand{\nset}[2]{\left\llbracket #1\ ;\ #2 \right\rrbracket}

%-Exponantial / complexs
\newcommand{\e}{\mathrm{e}}
\newcommand{\cj}[1]{\overline{#1}} %overline for the conjugate.

%-Vectors
\newcommand{\vect}{\overrightarrow}
\newcommand{\veco}[3]{\displaystyle \vect{#1}\binom{#2}{#3}} %vector + coord

%-Limits
\newcommand{\lm}[2][{}]{\lim\limits_{\substack{#2 \\ #1}}} %$\lm{x \to a} f$ or $\lm[x < a]{x \to a} f$
\newcommand{\Lm}[3][{}]{\lm[#1]{#2} \left( #3 \right)} %$\Lm{x \to a}{f}$ or $\Lm[x < a]{x \to a}{f}$
\newcommand{\tendsto}[1]{\xrightarrow[#1]{}}

%-Integral
\newcommand{\dint}[4][x]{\displaystyle \int_{#2}^{#3} #4 \mathrm{d} #1} %$\dint{a}{b}{f(x)}$ or $\dint[t]{a}{b}{f(t)}$

%-left right
\newcommand{\lr}[1]{\left( #1 \right)}
\newcommand{\lrb}[1]{\left[ #1 \right]}
\newcommand{\lrbb}[1]{\left\llbracket #1 \right\rrbracket}
\newcommand{\set}[1]{\left\{ #1 \right\}}
\newcommand{\abs}[1]{\left\lvert #1 \right\rvert}
\newcommand{\norm}[1]{\left\lVert #1 \right\rVert}
\newcommand{\ceil}[1]{\left\lceil #1 \right\rceil}
\newcommand{\floor}[1]{\left\lfloor #1 \right\rfloor}
\newcommand{\lrangle}[1]{\left\langle #1 \right\rangle}

%-Others
\newcommand{\para}{\ /\!/\ } %//
\newcommand{\ssi}{\ \Leftrightarrow \ }
\newcommand{\eqsys}[2]{\begin{cases} #1 \\ #2 \end{cases}}

\newcommand{\med}[2]{\mathrm{med} \left[ #1\ ;\ #2 \right]}  %$\med{A}{B} -> med[A ; B]$
\newcommand{\Circ}[2]{\mathscr{C}_{#1, #2}}

\renewcommand{\le}{\leqslant}
\renewcommand{\ge}{\geqslant}

\newcommand{\oboxed}[1]{\textcolor{ff4500}{\boxed{\textcolor{black}{#1}}}} %orange boxed


%------commands
%---to quote french text
\newcommand{\simplecit}[1]{\guillemotleft$\;$#1$\;$\guillemotright}
\newcommand{\cit}[1]{\simplecit{\textcolor{656565}{#1}}}
\newcommand{\quo}[1]{\cit{\it #1}}

%---to indent
\newcommand{\ind}[1][20pt]{\advance\leftskip + #1}
\newcommand{\deind}[1][20pt]{\advance\leftskip - #1}

%---to indent a text
\newcommand{\indented}[2][20pt]{\par \ind[#1] #2 \par \deind[#1]}
\newenvironment{indt}[2][20pt]{#2 \par \ind[#1]}{\par \deind} %Titled indented env

%---title
\newcommand{\thetitle}[2]{\begin{center}\textbf{{\LARGE \underline{\emph{#1} :}} {\Large #2}}\end{center}}

%---Maths environments
%-Proofs
\newenvironment{proof}[1][{}]{\begin{indt}{$\square$ #1}}{$\blacksquare$ \end{indt}}

%-Maths parts (proposition, definition, ...)
\newenvironment{mathspart}[1]{\begin{indt}{\boxed{\text{\textbf{#1}}}}}{\end{indt}}
\newenvironment{mathbox}[1]{\boxed{\text{\textbf{#1}}}\begin{pseudocode}}{\end{pseudocode}}
\newenvironment{mathul}[1]{\begin{indt}{\underline{\textbf{#1}}}}{\end{indt}}

\newenvironment{theo}{\begin{mathspart}{Théorème}}{\end{mathspart}}
\newenvironment{Theo}{\begin{mathbox}{Théorème}}{\end{mathbox}}
\newenvironment{theodef}{\begin{mathspart}{Théorème-définition}}{\end{mathspart}}

\newenvironment{prop}{\begin{mathspart}{Proposition}}{\end{mathspart}}
\newenvironment{Prop}{\begin{mathbox}{Proposition}}{\end{mathbox}}
\newenvironment{props}{\begin{mathspart}{Propriétés}}{\end{mathspart}}

\newenvironment{defi}{\begin{mathspart}{Définition}}{\end{mathspart}}
\newenvironment{meth}{\begin{mathspart}{Méthode}}{\end{mathspart}}

\newenvironment{Rq}{\begin{mathul}{Remarque :}}{\end{mathul}}
\newenvironment{Rqs}{\begin{mathul}{Remarques :}}{\end{mathul}}

\newenvironment{Ex}{\begin{mathul}{Exemple :}}{\end{mathul}}
\newenvironment{Exs}{\begin{mathul}{Exemples :}}{\end{mathul}}


%------Sections
% To change section numbering :
% \renewcommand\thesection{\Roman{section}}
% \renewcommand\thesubsection{\arabic{subsection})}
% \renewcommand\thesubsubsection{\textit \alph{subsubsection})}

% To start numbering from 0
% \setcounter{section}{-1}


%------page style
\usepackage{fancyhdr}
\usepackage{lastpage}

\setlength{\headheight}{18pt}
\setlength{\footskip}{50pt}

\pagestyle{fancy}
\fancyhf{}
\fancyhead[LE, RO]{\textit{\textcolor{black}{\today}}}
\fancyhead[RE, LO]{\large{\textsl{\emph{\texttt{\jobname}}}}}

\fancyfoot[RO, LE]{\textit{\texttt{\textcolor{black}{Page \thepage /}\pageref{LastPage}}}} %Change 'black' to 'white' for dark mode
\fancyfoot[LO, RE]{\includegraphics[scale=0.12]{/home/lasercata/Pictures/1.images_profil/logo/mieux/lasercata_logo_fly_fond_blanc.png}}

% For dark mode :
%/home/lasercata/Pictures/1.images_profil/logo/mieux/lasercata_logo_fly.png


%------init lengths
\setlength{\parindent}{0pt} %To avoid using \noindent everywhere.
\setlength{\parskip}{3pt}


%---------------------------------Begin Document
\begin{document}
    
    \thetitle{Physics}{Mechanics theorems}
    
    \tableofcontents
    \newpage
    
    \begin{indt}{\section{Principe fondamental de la dynamique}}
        \begin{indt}{\subsection{Définitions}}
            \begin{indt}{\subsubsection{Définition (\textit{quantité de mouvement})}}
                Soit $M$ un point de masse $m$, de vitesse $\vec v$ dans un référentiel $\mathscr R$.

                La \textit{quantité de mouvement} du point $M$ est la grandeur suivante :
                \[
                    \vec p = m \vec v
                \]
            \end{indt}

            \vspace{12pt}
            
            \begin{indt}{\subsubsection{Propriété}}
                Soit $\Sigma = \set{M_k\ |\ k \in I}$ un système, où $\forall k \in I$ $M_k$ est de masse $m_k$.

                La quantité de mouvement de $\Sigma$ est :
                \[
                    \vec p = \sum_{k \in I} m_k \vec v
                \]
            \end{indt}
        \end{indt}

        \vspace{12pt}
        
        \begin{indt}{\subsection{Lois de \textsc{Newton}}}
            \begin{indt}{\subsubsection{Première loi de \textsc{Newton}}}
                Il existe des référentiels privilégiés dans lesquels le mouvement d'un point matériel isolé est rectiligne uniforme.

                Ces référentiels sont appelés \textit{référentiels galiléens}.
            \end{indt}

            \vspace{12pt}
            
            \begin{indt}{\subsubsection{Deuxième loi de \textsc{Newton}}}
                Soit $M$ un point matériel de masse $m$ dans un référentiel galiléen $\mathscr R$.

                Soit $\vec F$ la résultante des forces s'exerçant sur $M$, et $\vec p$ la quantité de mouvement de $M$.

                Alors :
                \[
                    \vec F = \dt{\vec p}
                \]

                et si $m$ est constante, on a :
                \[
                    \vec F = m\vec a
                \]
                avec $\vec a$ l'accélération de $M$ dans $\mathscr R$.

                \vspace{12pt}
                
                \textit{Remarques :}

                $\bullet$ Cette loi est aussi appelée \textit{principe fondamental de la dynamique}.

                $\bullet$ Cette loi est un postulat, elle ne se démontre pas, mais se vérifie expérimentalement.
            \end{indt}

            \vspace{12pt}
            
            \begin{indt}{\subsubsection{Troisième loi de \textsc{Newton}}}
                Soient $A, B$ deux points matériels.

                On a :
                \[
                    \vect{f_{A/B}} + \vect{f_{B/A}} = \vect 0
                \]

                De plus, ces deux forces sont portées par la droite $(AB)$.

                \vspace{12pt}
                
                \textit{Remarque :}

                $\bullet$ On appelle aussi cette loi le \textit{principe des actions réciproques}.
            \end{indt}
        \end{indt}
    \end{indt}

    \vspace{12pt}
    
    \begin{indt}{\section{\'Energie}}
        \begin{indt}{\subsection{Travail}}
            \begin{indt}{\subsubsection{Définition (\textit{travail d'une force constante})}}
                Soient $A, B$ deux points, et soit $\vec f$ une force constante.

                \vspace{6pt}
                
                Le \textit{travail} de $\vec f$ est :
                \[
                    W\!\lr{\vec f} = \vec f \cdot \vect{AB}
                \]
            \end{indt}

            \vspace{12pt}
            
            \begin{indt}{\subsubsection{Définition (\textit{travail élémentaire})}}
                Soit $M$ un point mobile, et $\vec f$ une force.

                \vspace{6pt}
                
                On définit le \textit{travail élémentaire} de $\vec f$ dans le déplacement $\vect{\mathrm d M}$ par :
                \[
                    \delta W\!\lr{\vec f} = \vec f \cdot \vect{\mathrm d M}
                \]

                \vspace{12pt}
                
                \textit{Remarque :}

                Si on se place dans un référentiel $\mathscr R$, avec $O$ un point fixe de $\mathscr R$, on a $\vec v = \dt{\vect{OM}}$, donc $\vect{\mathrm d M} = \mathrm d \vect{OM} = \vec v \mathrm dt$, et :
                \[
                    \delta W\!\lr{\vec f} = \vec f \cdot \vec v \mathrm dt
                \]
            \end{indt}

            \vspace{12pt}
            
            \begin{indt}{\subsubsection{Définition (\textit{travail le long d'une courbe})}}
                Soient $A, B$ deux points d'un référentiel $\mathscr R$, $M$ un point mobile se déplaçant de $A$ à $B$ le long d'une courbe $(\mathscr C)$, et soit $\vec f$ une force s'exerçant sur $M$.

                Le \textit{travail} de $\vec f$ entre $A$ et $B$ sur $(\mathscr C)$ est :
                \[
                    W_{A \rightarrow B, (\mathscr C)}\!\lr{\vec f}
                    = \int_{A\ (\mathscr C)}^B \delta W\!\lr{\vec f}
                \]

                \vspace{12pt}
                
                \textit{Remarque :}

                En cas de parcours fermé (\textit{i.e} $A = B$), on note $\displaystyle W = \oint_{(\mathscr C)} \delta W$.
            \end{indt}
        \end{indt}

        \vspace{12pt}
        
        \begin{indt}{\subsection{Puissance}}
            \begin{indt}{\subsubsection{Définition (\textit{puissance d'une force})}}
                Soit $M$ un point de vitesse $\vec v$ dans un référentiel $\mathscr R$, et soit $\vec f$ une force s'exerçant sur $M$.

                Alors la \textit{puissance} de $\vec f$ est :
                \[
                    P = \vec f \cdot \vec v
                \]
            \end{indt}

            \vspace{12pt}
            
            \begin{indt}{\subsubsection{Propriétés}}
                Soit $M$ un point de vitesse $\vec v$ dans un référentiel $\mathscr R$, et soit $\vec f$ une force s'exerçant sur $M$.

                Comme $\delta W = \vec f \cdot \vec v \mathrm dt$, on a
                \[
                    \delta W = P\mathrm dt
                \]
            \end{indt}
        \end{indt}

        \vspace{12pt}
        
        \begin{indt}{\subsection{Énergie cinétique}}
            \begin{indt}{\subsubsection{Définition (\textit{énergie cinétique})}}
                Soit $M$ un point matériel de masse $m$ et de vitesse $\vec v$ (de norme $v$) dans un référentiel $\mathscr R$.

                Alors l'\textit{énergie cinétique} de $M$ est :
                \[
                    E_{\rm c} = \dfrac 1 2 mv^2
                \]
            \end{indt}

            \vspace{12pt}
            
            \begin{indt}{\subsubsection{Théorème de la puissance cinétique}}
                Soit $M$ un point matériel de masse $m$ et de vitesse $\vec v$ (de norme $v$) dans un référentiel galiléen $\mathscr R$, et $\vec F$ la résultantes des forces s'exerçant sur $M$.

                Alors on a :
                \[
                    P = \dt{E_{\rm c}}
                \]

                Conséquence : si $\vec F$ est normale à $\vec v$, alors $v = \norm{\vec v}$ est constante.
            \end{indt}

            \vspace{12pt}
            
            \begin{indt}{\subsubsection{Théorème de l'énergie cinétique}}
                Soit $M$ un point matériel de masse $m$ dans un référentiel galiléen $\mathscr R$, deux points $A, B$, et $\vec F$ la résultante de toutes les forces s'exerçant sur $M$.

                Alors on a :
                \[
                    \Delta_{A \to B} E_{\rm c} = W_{A \to B}\!\lr{\vec F}
                \]

                où $\Delta_{A \to B} E_{\rm c} = E_{\rm c}(B) - E_{\rm c}(A)$.
            \end{indt}
        \end{indt}

        \vspace{12pt}
        
        \begin{indt}{\subsection{Énergie potentielle}}
            \begin{indt}{\subsubsection{Définition (\textit{\'Energie potentielle})}}
                Soit $M$ un point matériel soumis à un champ de forces $\vec f(M)$.

                \begin{indt}{Si $\exists U\ |\ \delta W\!\!\lr{\vec f} = -\mathrm dU$, alors on dit que :}
                    $\bullet$ La force $\vec f$ est \textit{conservative} ;

                    $\bullet$ La force $\vec f$ \textit{dérive} du potentiel $U$ ;

                    $\bullet$ $U$ est le potentiel de la force $\vec f$ ;

                    $\bullet$ $U(M)$ est l'\textit{énergie potentielle} du point $M$ dans le champ de forces $\vec f(M)$.
                \end{indt}
            \end{indt}

            \vspace{12pt}
            
            \begin{indt}{\subsubsection{Propriétés}}
                Soit $M$ un point matériel soumis à un champ de forces $\vec f(M)$ conservatif de potentiel $U$.

                \vspace{6pt}
                
                $\bullet$ $U$ n'est connue que par $\mathrm dU$, donc $U$ est définie à une constante additive près.

                \vspace{6pt}
                
                $\bullet$ On a $\delta W = -\mathrm dU$, donc :
                \[
                    \int_A^B \delta W = -\int_A^B \mathrm dU
                \]
                Donc :
                \[
                    W_{A \to B} = U(A) - U(B)
                    = - \Delta_{A \to B} U
                \]
                \textit{I.e} le travail ne dépend pas du chemin suivi par $M$, et si le parcours est fermé, le travail est nul.

                \vspace{6pt}
                
                $\bullet$ Si le point $M$ a une vitesse $\vec v$ dans le référentiel $\mathscr R$, on a
                \[
                    -\mathrm dU = \delta W
                    = \vec f \cdot \vect{\mathrm dM}
                    = \vec f \cdot \vec v \mathrm dt
                \]
                Donc :
                \[
                    \dt U = -\vec f \cdot \vec v
                \]
            \end{indt}
        \end{indt}

        \vspace{12pt}
        
        \begin{indt}{\subsection{Énergie mécanique}}
            \begin{indt}{\subsubsection{Définition (\textit{\'Energie mécanique})}}
                Soit un point $M$ soumis à un champ de forces $\vec f(M) = \vec{f_1}(M) + \vec{f_2}(M)$ dans un référentiel $\mathscr R$, où $\vec{f_1}$ est conservatif de potentiel $U$.

                \vspace{6pt}
                
                Alors l'\textit{énergie mécanique} du point $M$ est :
                \[
                    E = E_{\rm c} + U
                \]
            \end{indt}

            \vspace{12pt}
            
            \begin{indt}{\subsubsection{Propriété}}
                Soit $M$ un point matériel soumis à un champ de forces $\vec f(M)$ conservatif de potentiel $U$, dans un référentiel galiléen $\mathscr R$.

                \vspace{6pt}
                
                Alors l'énergie mécanique $E$ est constante.
            \end{indt}

            \vspace{12pt}
            
            \begin{indt}{\subsubsection{Théorème de l'énergie mécanique}}
                \label{2.5.3}

                Soit un point matériel $M$ dans un référentiel galiléen $\mathscr R$, et soit $\vec f = \vec{f_{\rm c}} + \vec{f_{\rm nc}}$ la résultante des forces s'exerçant sur $M$, où $\vec{f_{\rm c}}$ est conservative et $\vec{f_{\rm nc}}$ est non conservative.

                \vspace{6pt}
                
                Alors :
                \[
                    \Delta_{A \to B} E = W_{A \to B}\!\lr{\vec{f_{\rm nc}}}
                \]

                \vspace{6pt}
                
                Forme infinitésimale du théorème de l'énergie mécanique :
                \[
                    \mathrm dE = \delta W\!\lr{\vec{f_{\rm nc}}}
                \]
            \end{indt}

            \vspace{12pt}
            
            \begin{indt}{\subsubsection{Théorème de la puissance mécanique}}
                On reprend les notations du point précédant (\ref{2.5.3}).

                On a :
                \[
                    \dt E = P\!\lr{\vec{f_{\rm nc}}}
                \]
            \end{indt}
        \end{indt}
    \end{indt}

    \vspace{12pt}
    
    \begin{indt}{\section{Moment}}
        \begin{indt}{\subsection{Moment d'une force}}
            \begin{indt}{\subsubsection{Par rapport à un point}}
                Soit $M$ un point soumis à une force $\vec f$, et $A$ un point quelconque.

                \vspace{6pt}
                
                Le \textit{moment} de $\vec f$ par rapport à $A$ est :
                \[
                    \vect{\mathscr M_A}\!\lr{\vec f} = \vect{AM} \wedge \vec f
                \]
            \end{indt}

            \vspace{12pt}
            
            \begin{indt}{\subsubsection{Par rapport à un axe orienté}}
                Soit $M$ un point soumis à une force $\vec f$, $A$ un point quelconque, $\Delta$ un axe passant par $A$ et orienté par un vecteur unitaire $\vect{u_\Delta}$.

                \vspace{6pt}
                
                Le \textit{moment} de $\vec f$ par rapport à $\Delta$ est :
                \[
                    \mathscr M_\Delta\!\lr{\vec f} = \vect{\mathscr M_A}\!\lr{\vec f} \cdot \vect{u_\Delta}
                \]

                \vspace{6pt}
                
                \textit{Remarque} : cette grandeur est bien définie, car pour $A' \in \Delta$, on a :
                \[
                    \begin{array}{rcl}
                        \vect{\mathscr M_{A'}}\!\lr{\vec f} \cdot \vect{u_\Delta}
                        &=& \lr{\vect{A'M} \wedge \vec f} \cdot \vect{u_\Delta}
                        \vspace{3pt}
                        \\
                        &=& \lr{\vect{A'A} \wedge \vec f + \vect{AM} \wedge \vec f} \cdot \vect{u_\Delta}
                        \vspace{3pt}
                        \\
                        &=& \underbrace{\lr{\vect{A'A} \wedge \vec f}}_{\perp \Delta} \cdot \vect{u_\Delta} + \lr{\vect{AM} \wedge \vec f} \cdot \vect{u_\Delta}
                        \vspace{3pt}
                        \\
                        &=& \vect{\mathscr M_A}\!\lr{\vec f} \cdot \vect{u_\Delta}
                    \end{array}
                \]
            \end{indt}

            \vspace{12pt}
            
            \begin{indt}{\subsubsection{Bras de levier}}
                Soit $M$ un point soumis à une force $\vec f$, $A$ un point quelconque, et $\Delta$ l'axe passant par $A$ perpendiculaire au plan formé par $\vect{AM}$ et $\vec f$, orienté par un vecteur unitaire $\vect{u_\Delta}$.
                
                Soit $H$ le projeté orthogonal de $A$ sur la droite d'action de $\vec f$.

                \vspace{6pt}
                
                On a alors :
                \[
                    \vect{\mathscr M_A}\!\lr{\vec f} = \mathscr M_\Delta\!\lr{\vec f} \cdot \vect{u_\Delta}
                \]
                et :
                \[
                    \norm{\vect{\mathscr M_A}\!\lr{\vec f}} = \abs{\mathscr M_\Delta\!\lr{\vec f}}
                    = \norm{\vect{AH}} \cdot \norm{\vec f}
                \]

                \vspace{6pt}
                
                \begin{center}
                    \begin{tikzpicture}[scale=1.3]
                        \node (A) at (0, 0) [label=above left:$A$] {$+$};
                        \node (M) at (4, 1) [label=above right:$M$] {$+$};

                        \draw [dashed] (2, 3) to (6, -1);

                        \draw[-latex] (0, 0) to (4, 1);
                        \draw[color=ff4500, -latex] (4, 1) to (5, 0);

                        \node (f) at (5, 0) [label=above right:${\vec f}$] {};

                        \node (u) at (-1, 2) [label=above right:${\vect{u_\Delta}}$] {$\odot$};

                        \node (H) at (2.5, 2.5) [label=above right:$H$] {};

                        \draw [dashed] (2.53, 2.53) to (0, 0);
                    \end{tikzpicture}
                \end{center}
            \end{indt}
        \end{indt}

        \vspace{12pt}
        
        \begin{indt}{\subsection{Moment cinétique d'un point}}
            \begin{indt}{\subsubsection{Par rapport à un point}}
                Soit $M$ un point matériel de masse $m$, de vitesse $\vec v$ dans un référentiel $\mathscr R$, de quantité de mouvement $\vec p = m\vec v$, et soit $A$ un point quelconque.

                \vspace{6pt}
                
                Le \textit{moment cinétique} de $M$ par rapport à $A$ est :
                \[
                    \begin{array}{rcl}
                        \vect{\sigma_A}(M)
                        &=& \vect{\mathscr M_A}\!\lr{\vec p}
                        \vspace{3pt}
                        \\
                        &=& \vect{AM} \wedge \vec p
                        \vspace{3pt}
                        \\
                        &=& m \vect{AM} \wedge \vec v
                    \end{array}
                \]

                \textit{Remarque} : on note aussi $\vect{L_A}(M) = \vect{\sigma_A}(M)$
            \end{indt}

            \vspace{12pt}
            
            \begin{indt}{\subsubsection{Par rapport à un axe orienté}}
                Soit $M$ un point matériel, $A$ un point quelconque, et $\Delta$ un axe passant par $A$ et orienté par un vecteur unitaire $\vect{u_\Delta}$.

                \vspace{6pt}
                
                Le \textit{moment cinétique} de $M$ par rapport à $\Delta$ est :
                \[
                    \sigma_\Delta(M) = \vect{\sigma_A}(M) \cdot \vect{u_\Delta}
                \]
            \end{indt}
        \end{indt}

        \vspace{12pt}
        
        \begin{indt}{\subsection{Théorème du moment cinétique}}
            \begin{indt}{\subsubsection{En un point fixe}}
                Soit $M$ un point de masse $m$, de vitesse $\vec v$ dans un référentiel galiléen $\mathscr R$, $A$ un point \textbf{fixe} de $\mathscr R$, et $\displaystyle \vec F = \sum_{k \in I} \vec{f_k}$ la résultante des forces $\vec{f_k}$ s'exerçant sur $M$.

                \vspace{6pt}
                
                Alors :
                \[
                    \dt{\vect{\sigma_A}} = \vect{\mathscr M_A}\!\lr{\vec F}
                    = \sum_{k \in I} \vect{\mathscr M_A}\!\lr{\vec{f_k}}
                \]

                \textit{Remarque} : on appelle aussi ce théorème le \textit{théorème vectoriel du moment cinétique}
            \end{indt}

            \vspace{12pt}
            
            \begin{indt}{\subsubsection{Par rapport à un axe fixe}}
                Soit $M$ un point de masse $m$, de vitesse $\vec v$ dans un référentiel galiléen $\mathscr R$, $A$ un point \textbf{fixe} de $\mathscr R$, et $\Delta$ un axe \textbf{fixe} passant par $A$, et orienté par un vecteur unitaire $\vect{u_\Delta}$.
                On note $\vec F$ la résultante des forces s'exerçant sur $M$.

                \vspace{6pt}
                
                Alors :
                \[
                    \dt{\sigma_\Delta} = \mathscr M_\Delta\!\lr{\vec F}
                \]

                \textit{Remarque} : On appelle ce théorème le \textit{théorème scalaire du moment cinétique}
            \end{indt}
        \end{indt}
    \end{indt}
    
    
    
\end{document}
%--------------------------------------------End
