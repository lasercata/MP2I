\documentclass[a4paper, 12pt, twoside]{article}


%------------------------------------------------------------------------
%
% Author                :   Lasercata
% Last modification     :   2022.01.21
%
%------------------------------------------------------------------------


%------ini
\usepackage[utf8]{inputenc}
\usepackage[T1]{fontenc}
%\usepackage[french]{babel}
%\usepackage[english]{babel}


%------geometry
\usepackage[textheight=700pt, textwidth=500pt]{geometry}


%------color
\usepackage{xcolor}
\definecolor{ff4500}{HTML}{ff4500}
\definecolor{00f}{HTML}{0000ff}
\definecolor{0ff}{HTML}{00ffff}
\definecolor{656565}{HTML}{656565}

\renewcommand{\emph}{\textcolor{ff4500}}
\renewcommand{\em}{\color{ff4500}}

\newcommand{\strong}[1]{\textcolor{ff4500}{\bf #1}}
\newcommand{\st}{\color{ff4500}\bf}


%------Code highlighting
\usepackage{listings}

\definecolor{cbg}{HTML}{272822}
\definecolor{cfg}{HTML}{ececec}
\definecolor{ccomment}{HTML}{686c58}
\definecolor{ckw}{HTML}{f92672}
\definecolor{cstring}{HTML}{e6db72}
\definecolor{cstringlight}{HTML}{98980f}
\definecolor{lightwhite}{HTML}{fafafa}

\lstdefinestyle{DarkCodeStyle}{
    backgroundcolor=\color{cbg},
    commentstyle=\itshape\color{ccomment},
    keywordstyle=\color{ckw},
    numberstyle=\tiny\color{cbg},
    stringstyle=\color{cstring},
    basicstyle=\ttfamily\footnotesize\color{cfg},
    breakatwhitespace=false,
    breaklines=true,
    captionpos=b,
    keepspaces=true,
    numbers=left,
    numbersep=5pt,
    showspaces=false,
    showstringspaces=false,
    showtabs=false,
    tabsize=4,
    xleftmargin=\leftskip
}

\lstdefinestyle{LightCodeStyle}{
    backgroundcolor=\color{lightwhite},
    commentstyle=\itshape\color{ccomment},
    keywordstyle=\color{ckw},
    numberstyle=\tiny\color{cbg},
    stringstyle=\color{cstringlight},
    basicstyle=\ttfamily\footnotesize\color{cbg},
    breakatwhitespace=false,
    breaklines=true,
    captionpos=b,
    keepspaces=true,
    numbers=left,
    numbersep=10pt,
    showspaces=false,
    showstringspaces=false,
    showtabs=false,
    tabsize=4,
    frame=L,
    xleftmargin=\leftskip
}

%\lstset{style=DarkCodeStyle}
\lstset{style=LightCodeStyle}

%Usage : \begin{lstlisting}[language=Caml] ... \end{lstlisting}


%-------make the table of content clickable
\usepackage{hyperref}
\hypersetup{
    colorlinks,
    citecolor=black,
    filecolor=black,
    linkcolor=black,
    urlcolor=black
}


%------pictures
\usepackage{graphicx}
%\usepackage{wrapfig}


%------tabular
%\usepackage{color}
%\usepackage{colortbl}
%\usepackage{multirow}


%------Physics
%---Packages
%\usepackage[version=4]{mhchem} %$\ce{NO4^2-}$

%---Commands
\newcommand{\link}[2]{\mathrm{#1} \! - \! \mathrm{#2}}
\newcommand{\pt}[1]{\cdot 10^{#1}} % Power of ten
\newcommand{\dt}[2][t]{\dfrac{d#2}{d#1}} % Derivative


%------math
%---Packages
%\usepackage{textcomp}
%\usepackage{amsmath}
\usepackage{amssymb}
\usepackage{mathtools} % For abs
\usepackage{stmaryrd} %for \llbracket and \rrbracket
\usepackage{mathrsfs} %for \mathscr{x} (different from \mathcal{x})

%---Commands
%-Sets
\newcommand{\N}{\mathbb{N}} %set N
\newcommand{\Z}{\mathbb{Z}} %set Z
\newcommand{\Q}{\mathbb{Q}} %set Q
\newcommand{\R}{\mathbb{R}} %set R
\newcommand{\C}{\mathbb{C}} %set C
\newcommand{\U}{\mathbb{U}} %set U
\newcommand{\set}[2]{\left[ #1\ ;\ #2 \right]}
\newcommand{\nset}[2]{\left\llbracket #1\ ;\ #2 \right\rrbracket}

%-Exponantial / complexs
\newcommand{\e}{\mathrm{e}}
\newcommand{\cj}[1]{\overline{#1}} %overline for the conjugate.

%-Vectors
\newcommand{\vect}{\overrightarrow}
\newcommand{\veco}[3]{\displaystyle \vect{#1}\binom{#2}{#3}} %vector + coord

%-Limits
\newcommand{\lm}[2][{}]{\lim\limits_{\substack{#2 \\ #1}}} %$\lm{x \to a} f$ or $\lm[x < a]{x \to a} f$
\newcommand{\Lm}[3][{}]{\lm[#1]{#2} \left( #3 \right)} %$\Lm{x \to a}{f}$ or $\Lm[x < a]{x \to a}{f}$
\newcommand{\tendsto}[1]{\xrightarrow[#1]{}}

%-Integral
\newcommand{\dint}[4][x]{\displaystyle \int_{#2}^{#3} #4 \mathrm{d} #1} %$\dint{a}{b}{f(x)}$ or $\dint[t]{a}{b}{f(t)}$

%-Others
\newcommand{\para}{\ /\!/\ } %//
\newcommand{\ssi}{\ \Leftrightarrow \ }
\newcommand{\abs}[1]{\left\lvert #1 \right\rvert} % abs{x} -> |x|
\newcommand{\eqsys}[2]{\begin{cases} #1 \\ #2 \end{cases}}

\newcommand{\med}[2]{\mathrm{med} \left[ #1\ ;\ #2 \right]}  %$\med{A}{B} -> med[A ; B]$
\newcommand{\Circ}[2]{\mathscr{C}_{#1, #2}}

\newcommand{\lr}[1]{\left( #1 \right)}
\newcommand{\lrb}[1]{\left[ #1 \right]}

\newcommand{\lrangle}[1]{\left\langle #1 \right\rangle}

\renewcommand{\le}{\leqslant}
\renewcommand{\ge}{\geqslant}


%------commands
%---to quote french text
\newcommand{\simplecit}[1]{\guillemotleft$\;$#1$\;$\guillemotright}
\newcommand{\cit}[1]{\simplecit{\textcolor{656565}{#1}}}
\newcommand{\quo}[1]{\cit{\it #1}}

%---to indent
\newcommand{\ind}[1][20pt]{\advance\leftskip + #1}
\newcommand{\deind}[1][20pt]{\advance\leftskip - #1}

%---to indent a text
\newcommand{\indented}[2][20pt]{\par \ind[#1] #2 \par \deind[#1]}
\newenvironment{indt}[2][20pt]{#2 \par \ind[#1]}{\par \deind} %Titled indented env

%---title
\newcommand{\thetitle}[2]{\begin{center}\textbf{{\LARGE \underline{\emph{#1} :}} {\Large #2}}\end{center}}

%---parts
%-I
\newcommand{\mainpart}[2][$\!\!$]{\underline{\large \textbf{\emph{\textit{#1} #2}}}}
\newcommand{\bmainpart}[2][$\!\!$]{\underline{\large \textbf{\textit{#1} #2}}}
%-A
\newcommand{\subpart}[2][$\!\!$]{\underline{\bf \textit{#1} #2}}
%-1
\newcommand{\subsubpart}[2][$\!\!$]{\underline{\textsl{#1} #2}}
%-a
\newcommand{\subsubsubpart}[2][$\!\!$]{\underline{\it #1 #2}}


%------page style
\usepackage{fancyhdr}
\usepackage{lastpage}

\setlength{\headheight}{18pt}
\setlength{\footskip}{50pt}

\pagestyle{fancy}
\fancyhf{}
\fancyhead[LE, RO]{\textit{\today}}
\fancyhead[RE, LO]{\large{\textsl{\emph{\texttt{\jobname}}}}}

\fancyfoot[RO, LE]{\textit{\texttt{Page \thepage /\pageref{LastPage}}}}
\fancyfoot[LO, RE]{\includegraphics[scale=0.12]{/home/lasercata/Pictures/1.images_profil/logo/mieux/lasercata_logo_fly_fond_blanc.png}}


%------init lengths
\setlength{\parindent}{0pt} %To avoid using \noindent everywhere.
\setlength{\parskip}{3pt}


%---------------------------------Begin Document
\begin{document}

    \thetitle{Maths}{Arithmetic}
    
    
    \tableofcontents
    \newpage
    
    
    \begin{indt}{\section{Divisibilité}}
        
        \begin{indt}{\subsection{Définitions}}
            Divisibilité : $\forall a, b \in \Z,\ a|b \ssi \exists k \in \Z\ |\ b = ak$
            
            \vspace{6pt}
            
            Ensemble des diviseurs : $\forall a \in \Z,\ \mathcal D(a) = \{ x \in \Z,\ x|a \} = \{ x \in \Z\ |\ \exists k \in \Z\ |\ a = kx \}$
            
            \vspace{6pt}
            
            Multiples : $\forall a \in \Z,\ a\Z = \{ na, n \in \Z \}$
        \end{indt}
        
        \vspace{12pt}
        
        \begin{indt}{\subsection{Propriétés}}
            $
                \forall (a, b) \in \Z^2,\
                \eqsys{a|b}{b|a}
                \ssi \abs a = \abs b
            $
            
            \vspace{6pt}
            
            $
                \forall (a, b, c) \in \Z^3,\
                \eqsys{c|a}{c|b}
                \Rightarrow \forall (u, v) \in \Z,\ c|(au + bv)
            $
            
            \vspace{6pt}
            
            $
                \forall (a, b, c) \in \Z^3,\
                \eqsys{c \neq 0}{ac|bc}
                \Rightarrow a|b
            $
        \end{indt}
        
        \vspace{6pt}
        
        \begin{indt}{\subsection{Division euclidienne}}
            $
                \forall (a, b) \in \Z \times \Z^*,\
                \exists! (q, r) \in \Z \times \N\
                \left|
                \begin{array}{l}
                    a = bq + r
                    \\
                    0 \le r < b
                \end{array}
                \right.
            $
        \end{indt}
        
    \end{indt}
    
    \vspace{12pt}
    
    \begin{indt}{\section{Congruences}}
        
        \begin{indt}{\subsection{Définition}}
            $\forall (a, b, n) \in \Z^3,\ a \equiv b\ [n] \ssi \exists k \in \Z\ |\ a - b = kn$
        \end{indt}
        
        \vspace{6pt}
        
        \begin{indt}{\subsection{Propriétés}}
            $\forall a, a', b, b', \alpha \in \Z\ |\ \eqsys{a \equiv b\ [\alpha]}{a' \equiv b'\ [\alpha]}$, on a :
            
            \[ a + a' \equiv b + b'\ [\alpha] \]
            \[ aa' \equiv bb'\ [\alpha] \]
        \end{indt}
        
    \end{indt}
    
    \vspace{12pt}
    
    \begin{indt}{\section{PGCD}}
        \begin{indt}{\subsection{Définition}}
            $\forall (a, b) \in \N^2,\ \text{avec}\ a \neq 0\ \text{ou}\ b \neq 0,\quad a \wedge b = \max(\mathcal D(a) \cap \mathcal D(b))$
            
            $\forall (a, b) \in \Z^2,\ a \wedge b = \abs a \wedge \abs b$
        \end{indt}
        
        \vspace{6pt}
        
        \begin{indt}{\subsection{Propriétés}}
            $\forall (a, b) \in \N^* \times \N,\ a|b \ssi a \wedge b = a$
            
            \vspace{6pt}
            
            $\forall (a, b, q, r) \in \N^4\ |\ a = bq + r$, on a :
                \[ \mathcal D(a) \cap \mathcal D(b) = \mathcal D(b) \cap \mathcal D(r) \]
                \[ \text{si}\ a \neq 0\ \text{ou}\ b \neq 0,\quad a \wedge b = b \wedge r \]
            
            \vspace{6pt}
            
            $\forall (a, b, d) \in \N^2 \times \N^*\ |\ a \neq 0$ ou $b \neq 0$, on a :
                \[\begin{array}{rcl}
                    d = a \wedge b
                    &\ssi& \mathcal D(d) = \mathcal D(a) \cap \mathcal D(b)
                    \\
                    \\
                    &\ssi&
                    \begin{cases}
                        d|a
                        \\
                        d|b
                        \\
                        \forall n \in \N,\ \eqsys{n|a}{n|b} \Rightarrow n|d
                    \end{cases}
                \end{array}\]
            
            \vspace{6pt}
            
            $
                \forall (a, b) \in \Z^2,\
                \exists (a_1, b_1) \in \Z^2,\
                \begin{cases}
                    a_1 \wedge b_1 = 1
                    \\
                    a = a_1(a \wedge b)
                    \\
                    b = b_1(a \wedge b)
                \end{cases}
            $
        \end{indt}
        
        \vspace{12pt}
        
        \begin{indt}{\subsection{Algorithme d'Euclide}}
            \begin{indt}{Soient $a, b \in \N^*$. On construit tant que possible une suite $(r_n)_{n \ge -1} \subset \N$ par récurrence :}
                $-$ On pose $r_{-1} = a$ et $r_0 = b$ ;
                
                $-$ Si $r_n \neq 0$, on pose $r_{n + 1} \equiv r_{n - 1}\ [r_n]$, avec $0 \le r_{n + 1} < r_n$ ;
                
                $-$ Sinon, $r_{n + 1}$ n'est pas défini.
            \end{indt}
            
            \vspace{12pt}
            
            \begin{indt}{Alors :}
                $-$ $\exists p \in \N\ |\ \forall n \le p,\ r_n \neq 0$ et est bien défini, et $r_{p + 1} = 0$ ;
                
                $-$ $r_p = a \wedge b$.
            \end{indt}
        \end{indt}
    \end{indt}
    
    \vspace{12pt}
    
    \begin{indt}{\section{PPCM}}
        
        \begin{indt}{\subsection{Définition}}
            $\forall (a, b) \in \N^*,\ a \vee b = \min(a\Z \cap b\Z \cap \N^*)$
            
            $\forall (a, b) \in (\Z^*)^2,\ a \vee b = \abs a \vee \abs b$
        \end{indt}
        
        \vspace{6pt}
        
        \begin{indt}{\subsection{Propriétés}}
            $\forall a, b, m \in \N^*$, on a :
                \[
                    \begin{array}{rcl}
                        m = a \vee b
                        &\ssi& m\Z = a\Z \cap b\Z
                        \\
                        &\ssi&
                        \begin{cases}
                            a|m
                            \\
                            b|m
                            \\
                            \forall n \in \N, \eqsys{a|n}{b|n} \Rightarrow m|n
                        \end{cases}
                    \end{array}
                \]
            
            \vspace{6pt}
            
            $\forall (a, b) \in \Z^2\ |\ a \wedge b = 1,\ a \vee b = \abs{ab}$
            
            \vspace{6pt}
            
            $\forall (a, b) \in \Z,\ \abs{ab} = (a \wedge b)(a \vee b)$
        \end{indt}
        
    \end{indt}

    
    \vspace{12pt}
    
    \begin{indt}{\section{Bézout, Gauss}}
        
        \begin{indt}{\subsection{Relation de Bézout}}
            $\forall (a, b) \in \Z^2,\ \exists (u, v) \in \Z^2\ |\ au + bv = a \wedge b$
        \end{indt}
        
        \vspace{6pt}
        
        \begin{indt}{\subsection{Théorème de Bézout}}
            $\forall (a, b) \in \Z^2,\ a \wedge b = 1 \ssi \exists (u, v) \in \Z^2\ |\ au + bv = 1$
        \end{indt}
        
        \vspace{6pt}
        
        \begin{indt}{\subsection{Lemme de Gauss}}
            $\forall (a, b, c) \in \Z^3,\ \eqsys{a|bc}{a \wedge b = 1} \Rightarrow a|c$
        \end{indt}
        
    \end{indt}
    
    \vspace{12pt}
    
    \begin{indt}{\section{Nombres premiers}}
        
        \begin{indt}{\subsection{Définition}}
            $\forall p \in \N^*,\ p \in \mathbb{P} \ssi \exists! (a, b) \in \N^2,\ a \neq b\ |\ p = ab\ \text{ou}\ p = ba$.
            
            Un entier naturel est premier si et seulement si il admet exactement deux diviseurs entiers distincts (qui sont alors 1 et lui même).
        \end{indt}
        
        \vspace{6pt}
        
        \begin{indt}{\subsection{Propriétés}}
            $\forall p \in \mathbb P$, on a :
                \[ \forall k \in \nset{1}{p - 1},\ p \left| \binom{p}{k} \right. \]
                \[ \forall (a, b, n) \in \Z^2 \times \N,\ (a + b)^n \equiv a + b\ [p] \]
            
            \vspace{6pt}
            
            $\forall (a, b, c) \in \Z^3$, on a :
                \[ a \wedge bc = 1 \ssi \eqsys{a \wedge b = 1}{a \wedge c = 1} \]
                \[
                    \begin{cases}
                        a|c
                        \\
                        b|c
                        \\
                        a \wedge b = 1
                    \end{cases}
                    \Rightarrow ab|c
                \]
        \end{indt}
        
        \vspace{6pt}
        
        \begin{indt}{\subsection{Petit théorème de Fermat}}
            $\forall (p, a) \in \mathbb P \times \Z,\ a^p \equiv a\ [p]$
            
            Et si $a \wedge p = 1$, alors $a^{p - 1} \equiv 1\ [p]$
        \end{indt}
        
    \end{indt}




    

    
    
\end{document}
%--------------------------------------------End
