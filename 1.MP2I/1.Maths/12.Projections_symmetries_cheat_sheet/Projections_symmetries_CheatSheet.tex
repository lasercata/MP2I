\documentclass[a4paper, 12pt, twoside]{article}


%------------------------------------------------------------------------
%
% Author                :   Lasercata
% Last modification     :   2022.08.28
%
%------------------------------------------------------------------------


%------ini
\usepackage[utf8]{inputenc}
\usepackage[T1]{fontenc}
%\usepackage[french]{babel}
%\usepackage[english]{babel}


%------geometry
\usepackage[textheight=700pt, textwidth=500pt]{geometry}


%------color
\usepackage{xcolor}
\definecolor{ff4500}{HTML}{ff4500}
\definecolor{00f}{HTML}{0000ff}
\definecolor{0ff}{HTML}{00ffff}
\definecolor{656565}{HTML}{656565}

\renewcommand{\emph}{\textcolor{ff4500}}
\renewcommand{\em}{\color{ff4500}}

\newcommand{\strong}[1]{\textcolor{ff4500}{\bf #1}}
\newcommand{\st}{\color{ff4500}\bf}


%------Code highlighting
%---listings
\usepackage{listings}

\definecolor{cbg}{HTML}{272822}
\definecolor{cfg}{HTML}{ececec}
\definecolor{ccomment}{HTML}{686c58}
\definecolor{ckw}{HTML}{f92672}
\definecolor{cstring}{HTML}{e6db72}
\definecolor{cstringlight}{HTML}{98980f}
\definecolor{lightwhite}{HTML}{fafafa}

\lstdefinestyle{DarkCodeStyle}{
    backgroundcolor=\color{cbg},
    commentstyle=\itshape\color{ccomment},
    keywordstyle=\color{ckw},
    numberstyle=\tiny\color{cbg},
    stringstyle=\color{cstring},
    basicstyle=\ttfamily\footnotesize\color{cfg},
    breakatwhitespace=false,
    breaklines=true,
    captionpos=b,
    keepspaces=true,
    numbers=left,
    numbersep=5pt,
    showspaces=false,
    showstringspaces=false,
    showtabs=false,
    tabsize=4,
    xleftmargin=\leftskip
}

\lstdefinestyle{LightCodeStyle}{
    backgroundcolor=\color{lightwhite},
    commentstyle=\itshape\color{ccomment},
    keywordstyle=\color{ckw},
    numberstyle=\tiny\color{cbg},
    stringstyle=\color{cstringlight},
    basicstyle=\ttfamily\footnotesize\color{cbg},
    breakatwhitespace=false,
    breaklines=true,
    captionpos=b,
    keepspaces=true,
    numbers=left,
    numbersep=10pt,
    showspaces=false,
    showstringspaces=false,
    showtabs=false,
    tabsize=4,
    frame=L,
    xleftmargin=\leftskip
}

%\lstset{style=DarkCodeStyle}
\lstset{style=LightCodeStyle}
%Usage : \begin{lstlisting}[language=Caml] ... \end{lstlisting}

%---tcolorbox
\usepackage[many]{tcolorbox}
\DeclareTColorBox{pseudocode}{O{black}O{lightwhite}}{
    breakable,
    outer arc=0pt,
    arc=0pt,
    top=0pt,
    toprule=-.5pt,
    right=0pt,
    rightrule=-.5pt,
    bottom=0pt,
    bottomrule=-.5pt,
    colframe=#1,
    colback=#2,
    enlarge left by=10pt,
    width=\linewidth-\leftskip-10pt,
}


%-------make the table of content clickable
\usepackage{hyperref}
\hypersetup{
    colorlinks,
    citecolor=black,
    filecolor=black,
    linkcolor=black,
    urlcolor=black
}
%Uncomment this and comment above for dark mode
% \hypersetup{
%     colorlinks,
%     citecolor=white,
%     filecolor=white,
%     linkcolor=white,
%     urlcolor=white
% }


%------pictures
\usepackage{graphicx}
%\usepackage{wrapfig}

\usepackage{tikz}
%\usetikzlibrary{babel}             %Uncomment this to use circuitikz
%\usetikzlibrary{shapes.geometric}  % To draw triangles in trees
%\usepackage{circuitikz}            %Electrical circuits drawing


%------tabular
%\usepackage{color}
%\usepackage{colortbl}
%\usepackage{multirow}


%------Physics
%---Packages
%\usepackage[version=4]{mhchem} %$\ce{NO4^2-}$

%---Commands
\newcommand{\link}[2]{\mathrm{#1} \! - \! \mathrm{#2}}
\newcommand{\pt}[1]{\cdot 10^{#1}} % Power of ten
\newcommand{\dt}[2][t]{\dfrac{\mathrm d #2}{\mathrm d #1}} % Derivative


%------math
%---Packages
%\usepackage{textcomp}
%\usepackage{amsmath}
\usepackage{amssymb}
\usepackage{mathtools} % For abs
\usepackage{stmaryrd} %for \llbracket and \rrbracket
\usepackage{mathrsfs} %for \mathscr{x} (different from \mathcal{x})

%---Commands
%-Sets
\newcommand{\N}{\mathbb{N}} %set N
\newcommand{\Z}{\mathbb{Z}} %set Z
\newcommand{\Q}{\mathbb{Q}} %set Q
\newcommand{\R}{\mathbb{R}} %set R
\newcommand{\C}{\mathbb{C}} %set C
\newcommand{\U}{\mathbb{U}} %set U
\newcommand{\seg}[2]{\left[ #1\ ;\ #2 \right]}
\newcommand{\nset}[2]{\left\llbracket #1\ ;\ #2 \right\rrbracket}

%-Exponantial / complexs
\newcommand{\e}{\mathrm{e}}
\newcommand{\cj}[1]{\overline{#1}} %overline for the conjugate.

%-Vectors
\newcommand{\vect}{\overrightarrow}
\newcommand{\veco}[3]{\displaystyle \vect{#1}\binom{#2}{#3}} %vector + coord

%-Limits
\newcommand{\lm}[2][{}]{\lim\limits_{\substack{#2 \\ #1}}} %$\lm{x \to a} f$ or $\lm[x < a]{x \to a} f$
\newcommand{\Lm}[3][{}]{\lm[#1]{#2} \left( #3 \right)} %$\Lm{x \to a}{f}$ or $\Lm[x < a]{x \to a}{f}$
\newcommand{\tendsto}[1]{\xrightarrow[#1]{}}

%-Integral
\newcommand{\dint}[4][x]{\displaystyle \int_{#2}^{#3} #4 \mathrm{d} #1} %$\dint{a}{b}{f(x)}$ or $\dint[t]{a}{b}{f(t)}$

%-left right
\newcommand{\lr}[1]{\left( #1 \right)}
\newcommand{\lrb}[1]{\left[ #1 \right]}
\newcommand{\lrbb}[1]{\left\llbracket #1 \right\rrbracket}
\newcommand{\set}[1]{\left\{ #1 \right\}}
\newcommand{\abs}[1]{\left\lvert #1 \right\rvert}
\newcommand{\ceil}[1]{\left\lceil #1 \right\rceil}
\newcommand{\floor}[1]{\left\lfloor #1 \right\rfloor}
\newcommand{\lrangle}[1]{\left\langle #1 \right\rangle}

%-Others
\newcommand{\para}{\ /\!/\ } %//
\newcommand{\ssi}{\ \Leftrightarrow \ }
\newcommand{\eqsys}[2]{\begin{cases} #1 \\ #2 \end{cases}}

\newcommand{\med}[2]{\mathrm{med} \left[ #1\ ;\ #2 \right]}  %$\med{A}{B} -> med[A ; B]$
\newcommand{\Circ}[2]{\mathscr{C}_{#1, #2}}

\renewcommand{\le}{\leqslant}
\renewcommand{\ge}{\geqslant}

\newcommand{\oboxed}[1]{\textcolor{ff4500}{\boxed{\textcolor{black}{#1}}}} %orange boxed


%------commands
%---to quote french text
\newcommand{\simplecit}[1]{\guillemotleft$\;$#1$\;$\guillemotright}
\newcommand{\cit}[1]{\simplecit{\textcolor{656565}{#1}}}
\newcommand{\quo}[1]{\cit{\it #1}}

%---to indent
\newcommand{\ind}[1][20pt]{\advance\leftskip + #1}
\newcommand{\deind}[1][20pt]{\advance\leftskip - #1}

%---to indent a text
\newcommand{\indented}[2][20pt]{\par \ind[#1] #2 \par \deind[#1]}
\newenvironment{indt}[2][20pt]{#2 \par \ind[#1]}{\par \deind} %Titled indented env

%---title
\newcommand{\thetitle}[2]{\begin{center}\textbf{{\LARGE \underline{\emph{#1} :}} {\Large #2}}\end{center}}


%------Sections
% To change section numbering :
% \renewcommand\thesection{\Roman{section}}
% \renewcommand\thesubsection{\arabic{subsection}}
% \renewcommand\thesubsubsection{\textit \alph{subsubsection}}

% To start numbering from 0
% \setcounter{section}{-1}


%------page style
\usepackage{fancyhdr}
\usepackage{lastpage}

\setlength{\headheight}{18pt}
\setlength{\footskip}{50pt}

\pagestyle{fancy}
\fancyhf{}
\fancyhead[LE, RO]{\textit{\textcolor{black}{\today}}}
\fancyhead[RE, LO]{\large{\textsl{\emph{\texttt{\jobname}}}}}

\fancyfoot[RO, LE]{\textit{\texttt{\textcolor{black}{Page \thepage /}\pageref{LastPage}}}} %Change 'black' to 'white' for dark mode
\fancyfoot[LO, RE]{\includegraphics[scale=0.12]{/home/lasercata/Pictures/1.images_profil/logo/mieux/lasercata_logo_fly_fond_blanc.png}}

% For dark mode :
%/home/lasercata/Pictures/1.images_profil/logo/mieux/lasercata_logo_fly.png


%------init lengths
\setlength{\parindent}{0pt} %To avoid using \noindent everywhere.
\setlength{\parskip}{3pt}


%---------------------------------Begin Document
\begin{document}
    
    %For dark mode :
    % \pagecolor{black}
    % \color{white}
    
    \thetitle{Maths}{Projections et symétries}
    
    \tableofcontents
    \newpage
    
    
    Dans tout ce qui suit, $K$ désigne un corps, et $E$ un $K$-espace vectoriel.


    \begin{indt}{\section{Projections}}
        
        \begin{indt}{\subsection{Définition (\textit{projection})}}
            \label{1.1}

            Soient $A, B$ deux sous-espaces vectoriels supplémentaires de $E$ (\textit{i.e} $E = A \oplus B$).

            La \textit{projection} (ou le \textit{projecteur}) sur $A$ parallèlement à $B$ est l'application
            \[
                \begin{array}{ccccc}
                    p & : & E & \longrightarrow & E
                    \\
                    && a + b & \longmapsto & a
                \end{array}
            \]
            où $(a, b) \in A \times B$.

            On appelle $A$ et $B$ les \textit{éléments caractéristiques} de $p$.

            \vspace{12pt}
            
            Remarque :
            cette application est bien définie puisque $E = A \oplus B$ (donc $\forall x \in E,\ \exists! (a, b) \in A \times B\ |\ x = a + b$).
        \end{indt}

        \vspace{12pt}
        
        \begin{indt}{\subsection{Exemple}}
            Prenons $E = \R^2$, et $A, B$ deux droites vectorielles non parallèles (\textit{i.e} $E = A \oplus B$).

            Soit $p$ la projection sur $A$ parallèlement à $B$, et $x = a + b \in E$, où $(a, b) \in A \times B$.

            \begin{center}
                \begin{tikzpicture}
                    \draw (-5, 0) to (5, 0);
                    \draw (-3, -3) to (3, 3);

                    \node (A) at (5.3, -.3) {$A$};
                    \node (B) at (3.3, 3.3) {$B$};

                    \node [label=above right:$x$] (x) at (3.5, 1.5) {};
                    \node [label=below:${a = p(x)}$] (a) at (2, 0) {};
                    \node [label=left:$b$] (b) at (1.5, 1.5) {};

                    \draw[-latex] (0, 0) to (3.5, 1.5);
                    \draw[-latex] (0, 0) to (2, 0);

                    \draw [dashed] (2, 0) -- (x) -- (1.5, 1.5);
                \end{tikzpicture}
            \end{center}
        \end{indt}

        \vspace{12pt}
        
        \begin{indt}{\subsection{Propriété (\textit{linéarité, idempotence})}}
            Soit $p$ une projection de $E$.

            Alors :

            $\bullet$ $p \in \mathcal L(E)$

            $\bullet$ $p \circ p = p$ (idempotence)

            \vspace{12pt}
            
            Remarque :
            Comme on travaille dans l'anneau $(\mathcal L(E), +, \circ)$, on note $p^2 = p \circ p$.
        \end{indt}

        \vspace{12pt}
        
        \begin{indt}{\subsection{Propriété}}
            Soit $f \in \mathcal L(E)\ |\ f^2 = f$.

            Alors $f$ est une projection.
        \end{indt}

        \vspace{12pt}
        
        \begin{indt}{\subsection{Propriété (\textit{éléments caractéristiques})}}
            Soit $p$ une projection de $E$.

            Alors :

            $\bullet$ $E = \mathrm{Im}(p) \oplus \mathrm{Ker}(p)$

            $\bullet$ $p$ est la projection sur $\mathrm{Im}(p)$ et parallèlement à $\mathrm{Ker}(p)$.

            Autrement dit, avec les notations de la définition \ref{1.1}, on a :
            \[
                \begin{cases}
                    A = \mathrm{Im}(p)
                    \\
                    B = \mathrm{Ker}(p)
                \end{cases}
            \]

            $\bullet$ $\mathrm{Im}(p) = \set{x \in E\ |\ x = p(x)} = \mathrm{Ker}(p - \mathrm{id}_E)$
        \end{indt}
        
    \end{indt}

    \vspace{12pt}
    
    \begin{indt}{\section{Symétries}}
        \begin{indt}{\subsection{Définition (\textit{symétrie})}}
            \label{2.1}

            Soient $A, B$ deux sous-espaces vectoriels supplémentaires de $E$ (\textit{i.e} $E = A \oplus B$).

            La \textit{symétrie} par rapport à $A$ et parallèlement à $B$ est l'application
            \[
                \begin{array}{ccccc}
                    s & : & E & \longrightarrow & E
                    \\
                    && a + b & \longmapsto & a - b
                \end{array}
            \]
            où $(a, b) \in A \times B$.

            On appelle $A$ et $B$ les \textit{éléments caractéristiques} de $s$.
        \end{indt}

        \vspace{12pt}
        
        \begin{indt}{\subsection{Exemple}}
            Prenons $E = \R^2$, et $A, B$ deux droites vectorielles non parallèles (\textit{i.e} $E = A \oplus B$).

            Soit $s$ la symétrie par rapport à $A$ et parallèlement à $B$, et $x = a + b \in E$, où $(a, b) \in A \times B$.

            \begin{center}
                \begin{tikzpicture}
                    \draw (-5, 0) to (5, 0);
                    \draw (-3, -3) to (3, 3);

                    \node (A) at (5.3, -.3) {$A$};
                    \node (B) at (3.3, 3.3) {$B$};

                    \node [label=above right:$x$] (x) at (3.5, 1.5) {};
                    \node [label=below right:$a$] (a) at (2, 0) {};
                    \node [label=left:$b$] (b) at (1.5, 1.5) {};
                    \node [label=left:$-b$] (b2) at (-1.5, -1.5) {};

                    \node [label=below right:$s(x)$] (s) at (.5, -1.5) {};

                    \draw[-latex] (0, 0) to (3.5, 1.5);
                    \draw[-latex] (0, 0) to (.5, -1.5);

                    \draw [dashed] (-1.5, -1.5) -- (s) -- (a) -- (x) -- (1.5, 1.5);
                \end{tikzpicture}
            \end{center}
        \end{indt}

        \vspace{12pt}
        
        \begin{indt}{\subsection{Propriété (\textit{automorphisme, involution})}}
            Soit $s$ une symétrie de $E$.

            Alors :

            $\bullet$ $s \in GL(E)$ (automorphisme)

            $\bullet$ $s^2 = \mathrm{id}_E$ (involution)
        \end{indt}

        \vspace{12pt}
        
        \begin{indt}{\subsection{Propriété}}
            Soit $f \in \mathcal L(E)\ |\ f^2 = \mathrm{id}_E$.

            Alors $f$ est une symétrie de $E$.
        \end{indt}

        \vspace{12pt}
        
        \begin{indt}{\subsection{Propriété (\textit{éléments caractéristiques})}}
            Soit $s$ une symétrie de $E$.

            Alors :

            $\bullet$ $E = \mathrm{Ker}(s - \mathrm{id}_E) \oplus \mathrm{Ker}(s + \mathrm{id}_E)$

            $\bullet$ $s$ est la symétrie par rapport à $\mathrm{Ker}(s - \mathrm{id}_E)$ et parallèlement à $\mathrm{Ker}(s + \mathrm{id}_E)$.

            Autrement dit, avec les notations de la définition \ref{2.1}, on a :
            \[
                \begin{cases}
                    A = \mathrm{Ker}(s - \mathrm{id}_E)
                    \\
                    B = \mathrm{Ker}(s + \mathrm{id}_E)
                \end{cases}
            \]

            \vspace{12pt}
            
            Remarque :
            \[
                \mathrm{Ker}(s - \mathrm{id}_E) = \set{x \in E\ |\ s(x) = x}
            \]
            \[
                \mathrm{Ker}(s + \mathrm{id}_E) = \set{x \in E\ |\ s(x) = -x}
            \]
        \end{indt}

        \vspace{12pt}
        
        \begin{indt}{\subsection{Propriété (\textit{relation projection / symétrie})}}
            Soient $p, s \in \mathcal L(E)\ |\ s = 2p - \mathrm{id}_E$.

            On a alors :
            \[
                p\ \text{projection}
                \ssi
                s\ \text{symétrie}
            \]

            Et dans ce cas, on a :
            \[
                \begin{cases}
                    \mathrm{Im}(p) = \mathrm{Ker}(s - \mathrm{id}_E)
                    \\
                    \mathrm{Ker}(p) = \mathrm{Ker}(s + \mathrm{id}_E)
                \end{cases}
            \]
            \textit{i.e} $p$ et $s$ ont les mêmes éléments caractéristiques.
        \end{indt}
    \end{indt}
    
    
    
\end{document}
%--------------------------------------------End
